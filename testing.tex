\documentclass[
	12pt,
	a4paper,
	% draft
	]
	{article}

\usepackage{mysty, xparse, GExam, fancyhdr}

\usepackage[
	top = 0cm,
	bottom = 0cm,
	left = 1cm,
	right = 1cm,
	headheight = 15pt,
	]{geometry}

% \usepackage{showframe}


\pagestyle{fancy}


\lhead{
	\begin{tikzpicture}[remember picture, overlay]
		\foreach \y in {-99mm, -198mm}
			\draw[dashed] (current page.north west) ++ (0, \y) -- +(210mm, 0);
	\end{tikzpicture}
}
\cfoot{}


\renewcommand{\bak}{Направления}
\renewcommand{\spec}{080100.62, 100100.62}
\renewcommand{\disc}{Теория автоматических систем сервиса и туризма в экономике}
\renewcommand{\kaf}{УУУиЭЭЭ}
\renewcommand{\logo}{logo_new_psk} 
% \renewcommand{\protocol}{\makebox[2.8cm]{\hrulefill}}
\teacher {доцент А. В. Подъяблонский}
\zavkaf {профессор С. Ю. Вышгородский}

\begin{document}


\ExamCard {
		Признак оптимальности $C$-ядра. Примеры вектора Шепли; мотивация.
		;;
		Равновесие Нэша. Уточнение; примеры. Равновесие в повторяющихся играх, триггерные стратегии. Дисконт-фактор. Совершенное в подыграх равновесие Нэша.
		;;
		Дополнительный вопрос (№\,4)
	}

\ExamCard {
		Признак оптимальности $C$-ядра. Примеры вектора Шепли; мотивация.
		;;
		Равновесие Нэша. Уточнение; примеры. Равновесие в повторяющихся играх, триггерные стратегии. Дисконт-фактор. Совершенное в подыграх равновесие Нэша.
		;;
		Коалиционные игры с бесконечным квадратом выигрыша. Парето-оптимальность. Критерий фон Неймана. Модель Спенса-Эджворта.
	}

\ExamCard {
		Защита от дыма и огня. Противогаз. Защитные очки.
		;;
		Защита от отравляющих веществ. Зарин, мазин и туразин-34. Марлевая повязка и ее устройство. Классификация противоядий и антидотов.
		;;
		Искусственная вентиляция легких и непрямой массаж сердца. Работа с манекеном на время.
	}



\end{document}