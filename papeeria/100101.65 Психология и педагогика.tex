\documentclass[
	14pt,
	a4paper,
	]
	{scrartcl}

\usepackage{mysty}

\newcommand{\spec}{100101.65}
\newcommand{\disc}{Психология\\ \multicolumn{2}{r}{ и педагогика }}
\newcommand{\kafedra}{ГиСД}

\usepackage[
	papersize={210mm,99mm},
	top=.5cm,
	left=1.5cm,
	right=1.5cm,
	bottom=.5cm
	]{geometry}

\pagestyle{empty}

\begin{document}


\shapk
\bilet{Экзаменационный билет}
\setcounter{zad}{0}

\vfill
\z 	Психология как наука. Предмет и объект психологии. Явления, которые изучает психология.
 \vfill
\z 	Семья как социальный институт.
 \vfill

\vfill

\newpage


\shapk
\bilet{Экзаменационный билет}
\setcounter{zad}{0}

\vfill
\z 	Изменение и расширение предмета психологии с древнейших времен до наших дней.
 \vfill
\z 	Семья как социокультурная среда воспитания и развития личности.
 \vfill

\vfill

\newpage


\shapk
\bilet{Экзаменационный билет}
\setcounter{zad}{0}

\vfill
\z 	Отрасли психологии.
 \vfill
\z 	Семья как способ педагогического взаимодействия.
 \vfill

\vfill

\newpage


\shapk
\bilet{Экзаменационный билет}
\setcounter{zad}{0}

\vfill
\z 	Методы психологии. Связь психологии с другими науками.
 \vfill
\z 	Государственные органы управления системой образования.
 \vfill

\vfill

\newpage


\shapk
\bilet{Экзаменационный билет}
\setcounter{zad}{0}

\vfill
\z 	Психика человека и ее развитие. Психика и организм. Особенности психического отражения.
 \vfill
\z 	Управление системой образования.
 \vfill

\vfill

\newpage


\shapk
\bilet{Экзаменационный билет}
\setcounter{zad}{0}

\vfill
\z 	Объективная и субъективная (психическая) реальность.
 \vfill
\z 	Методы, приемы, средства осуществления педагогического процесса.
 \vfill

\vfill

\newpage


\shapk
\bilet{Экзаменационный билет}
\setcounter{zad}{0}

\vfill
\z 	Сознание как форма отражения человеком действительности. Основные характеристики сознания.
 \vfill
\z 	Лекция, ее характеристика.
 \vfill

\vfill

\newpage


\shapk
\bilet{Экзаменационный билет}
\setcounter{zad}{0}

\vfill
\z 	Сознание и бессознательное. Основные механизмы психологической защиты.
 \vfill
\z 	Урок – основная форма организации обучения в современной школе.
 \vfill

\vfill

\newpage


\shapk
\bilet{Экзаменационный билет}
\setcounter{zad}{0}

\vfill
\z 	Познавательные психические процессы и особенности их протекания.
 \vfill
\z 	Понятие о формах организации обучения.
 \vfill

\vfill

\newpage


\shapk
\bilet{Экзаменационный билет}
\setcounter{zad}{0}

\vfill
\z 	Ощущение как простейший психический процесс. Классификация и основные характеристики ощущений.
 \vfill
\z 	Воспитание как педагогическое явление.
 \vfill

\vfill

\newpage


\shapk
\bilet{Экзаменационный билет}
\setcounter{zad}{0}

\vfill
\z 	Восприятие. Свойства, виды, формы восприятия. Восприятие и представление.
 \vfill
\z 	Цели, задачи, тенденции и принципы гуманистического воспитания.
 \vfill

\vfill

\newpage


\shapk
\bilet{Экзаменационный билет}
\setcounter{zad}{0}

\vfill
\z 	Воображение: виды, формы, приемы и функции воображения. Связь воображения с творчеством.
 \vfill
\z 	Характеристика развивающей функции обучения.
 \vfill

\vfill

\newpage


\shapk
\bilet{Экзаменационный билет}
\setcounter{zad}{0}

\vfill
\z 	Мышление. Формы и виды мышления.
 \vfill
\z 	Характеристика воспитательной функции обучения.
 \vfill

\vfill

\newpage


\shapk
\bilet{Экзаменационный билет}
\setcounter{zad}{0}

\vfill
\z 	Мышление, его методы  и операции.
 \vfill
\z 	Характеристика образовательной функции обучения.
 \vfill

\vfill

\newpage


\shapk
\bilet{Экзаменационный билет}
\setcounter{zad}{0}

\vfill
\z 	Речь и ее функции. Речь и язык. Взаимосвязь мышления и речи.
 \vfill
\z  
 \vfill

\vfill

\newpage


\shapk
\bilet{Экзаменационный билет}
\setcounter{zad}{0}

\vfill
\z 	Внимание как психический процесс. Основные свойства и функции внимания.
 \vfill
\z 	Образовательная система России.
 \vfill

\vfill

\newpage


\shapk
\bilet{Экзаменационный билет}
\setcounter{zad}{0}

\vfill
\z 	Память. Основные процессы памяти. Виды и формы памяти. Законы памяти.
 \vfill
\z 	Новейшие педагогические технологии.
 \vfill

\vfill

\newpage


\shapk
\bilet{Экзаменационный билет}
\setcounter{zad}{0}

\vfill
\z 	Определение личности в психологии. Соотношение понятий «человек», «индивид», «индивидуальность» с понятием «личность».
 \vfill
\z 	Педагогическая технология и педагогическая задача.
 \vfill

\vfill

\newpage


\shapk
\bilet{Экзаменационный билет}
\setcounter{zad}{0}

\vfill
\z 	Структура личности. Различные подходы в исследовании личности.
 \vfill
\z 	Понятия «воспитание» и «обучение».
 \vfill

\vfill

\newpage


\shapk
\bilet{Экзаменационный билет}
\setcounter{zad}{0}

\vfill
\z 	Социализация личности. Механизмы и стадии социализации
 \vfill
\z 	Образование и самообразование.
 \vfill

\vfill

\newpage


\shapk
\bilet{Экзаменационный билет}
\setcounter{zad}{0}

\vfill
\z 	Темперамент: типы и свойства.
 \vfill
\z 	Непрерывное образование, его значение для человека.
 \vfill

\vfill

\newpage


\shapk
\bilet{Экзаменационный билет}
\setcounter{zad}{0}

\vfill
\z 	Характер. Типология характера. Акцентуации характера.
 \vfill
\z 	Образование как педагогический процесс.
 \vfill

\vfill

\newpage


\shapk
\bilet{Экзаменационный билет}
\setcounter{zad}{0}

\vfill
\z 	Воля как психическая регуляция человеческого поведения. Волевые качества характера.
 \vfill
\z 	Образование как социокультурный феномен.
 \vfill

\vfill

\newpage


\shapk
\bilet{Экзаменационный билет}
\setcounter{zad}{0}

\vfill
\z 	Эмоции. Структура эмоций. Виды эмоций. Эмоции и мотивация.
 \vfill
\z 	Образование как общечеловеческая ценность.
 \vfill

\vfill

\newpage


\shapk
\bilet{Экзаменационный билет}
\setcounter{zad}{0}

\vfill
\z 	Категория деятельности в психологии. Виды деятельности. Структура деятельности.
 \vfill
\z 	Характеристика понятия «педагогическая система».
 \vfill

\vfill

\newpage


\shapk
\bilet{Экзаменационный билет}
\setcounter{zad}{0}

\vfill
\z 	Общение как вид психической деятельности. Компоненты общения. Виды общения. Цели и средства общения.
 \vfill
\z 	Связь педагогики и психологии. Значение психологических знаний для развития науки педагогики.
 \vfill

\vfill

\newpage


\shapk
\bilet{Экзаменационный билет}
\setcounter{zad}{0}

\vfill
\z 	Механизмы понимания и восприятия людьми друг друга в процессе общения.	Конфликт. Виды конфликтов.
 \vfill
\z 	Педагогика в системе наук.
 \vfill

\vfill

\newpage


\shapk
\bilet{Экзаменационный билет}
\setcounter{zad}{0}

\vfill
\z 	Способы профилактики и разрешения конфликтов.
 \vfill
\z 	Связь педагогики с другими науками.
 \vfill

\vfill

\newpage


\shapk
\bilet{Экзаменационный билет}
\setcounter{zad}{0}

\vfill
\z 	Малая группа и ее характеристики. Виды малых групп. Структура внутригрупповых отношений. Эффективность групповой деятельности.
 \vfill
\z 	Функции и задачи педагогики.
 \vfill

\vfill

\newpage


\shapk
\bilet{Экзаменационный билет}
\setcounter{zad}{0}

\vfill
\z 	Лидерство и руководство в малых группах. Теории лидерства. Стили лидерства.
 \vfill
\z 	Предмет и объект педагогики.
 \vfill

\vfill

\newpage



\end{document}