\documentclass[
	14pt,
	a4paper,
	]
	{scrartcl}

\usepackage{mysty}

\newcommand{\spec}{100101.65}
\newcommand{\disc}{ОФСС\\ \multicolumn{2}{r}{ (1 семестр) }}
\newcommand{\kafedra}{сервиса}

\usepackage[
	papersize={210mm,99mm},
	top=.5cm,
	left=1.5cm,
	right=1.5cm,
	bottom=.5cm
	]{geometry}

\pagestyle{empty}

\begin{document}


\shapk
\bilet{Экзаменационный билет}
\setcounter{zad}{0}

\vfill
\z Направление научно-технического прогресса.
 \vfill
\z Понятие об уравновешивании механизмов. \vfill

\vfill

\newpage


\shapk
\bilet{Экзаменационный билет}
\setcounter{zad}{0}

\vfill
\z Типы схем сборки изделия.
 \vfill
\z Основные требования при разработке технологического процесса.
 \vfill

\vfill

\newpage


\shapk
\bilet{Экзаменационный билет}
\setcounter{zad}{0}

\vfill
\z Технологические и технические факторы научно-технического прогресса.
 \vfill
\z Понятие об электрических цепях переменного тока.
 \vfill

\vfill

\newpage


\shapk
\bilet{Экзаменационный билет}
\setcounter{zad}{0}

\vfill
\z Технологический процесс в швейном производстве (в автосервисе).
 \vfill
\z Расчет частот собственных (свободных) колебаний.
 \vfill

\vfill

\newpage


\shapk
\bilet{Экзаменационный билет}
\setcounter{zad}{0}

\vfill
\z Многоуровневая схема целей и задач техники, технических и технологических- систем.
 \vfill
\z Понятие о внутреннем трении в материале.
 \vfill

\vfill

\newpage


\shapk
\bilet{Экзаменационный билет}
\setcounter{zad}{0}

\vfill
\z Примерная производственная структура предприятия автосервиса (ателье пошива одежды по индивидуальным заказам).
 \vfill
\z Основные цели и этапы технологического расчета машин (оборудования систем сервиса).
 \vfill

\vfill

\newpage


\shapk
\bilet{Экзаменационный билет}
\setcounter{zad}{0}

\vfill
\z Основные пути повышения показателей эффективности и функциональных характеристик технических систем.
 \vfill
\z Понятие о расчете электрических цепей постоянного тока.
 \vfill

\vfill

\newpage


\shapk
\bilet{Экзаменационный билет}
\setcounter{zad}{0}

\vfill
\z Виды и причины неисправностей деталей и механизмов.
 \vfill
\z Прочностной расчет деталей машин.
 \vfill

\vfill

\newpage


\shapk
\bilet{Экзаменационный билет}
\setcounter{zad}{0}

\vfill
\z Основные определения и понятия по эксплуатации техники.
 \vfill
\z Понятие о виброзащите машин.
 \vfill

\vfill

\newpage


\shapk
\bilet{Экзаменационный билет}
\setcounter{zad}{0}

\vfill
\z Гидростатические приводы.
 \vfill
\z Классификация характеристик механизмов.
 \vfill

\vfill

\newpage


\shapk
\bilet{Экзаменационный билет}
\setcounter{zad}{0}

\vfill
\z Показатели и признаки качества техники.
 \vfill
\z Понятие о динамической и статической жесткости.
 \vfill

\vfill

\newpage


\shapk
\bilet{Экзаменационный билет}
\setcounter{zad}{0}

\vfill
\z Технологическая последовательность процесса сборки.
 \vfill
\z Модель эффективности производственной системы.
 \vfill

\vfill

\newpage


\shapk
\bilet{Экзаменационный билет}
\setcounter{zad}{0}

\vfill
\z Количественная оценка уровня технологичности конструкции.
 \vfill
\z Понятия об электрических цепях постоянного тока.
 \vfill

\vfill

\newpage


\shapk
\bilet{Экзаменационный билет}
\setcounter{zad}{0}

\vfill
\z Гидродинамические передачи.
 \vfill
\z Понятие о статической и динамической балансировке.
 \vfill

\vfill

\newpage


\shapk
\bilet{Экзаменационный билет}
\setcounter{zad}{0}

\vfill
\z Основные понятия и определения надежности.
 \vfill
\z Общие сведения о магнитных материалах.
 \vfill

\vfill

\newpage


\shapk
\bilet{Экзаменационный билет}
\setcounter{zad}{0}

\vfill
\z Классификация приводов.
 \vfill
\z Силовой расчет механических трансмиссий.
 \vfill

\vfill

\newpage


\shapk
\bilet{Экзаменационный билет}
\setcounter{zad}{0}

\vfill
\z Оценка интегрального показателя качества и эффективности техники (изделия).
 \vfill
\z Общие сведения об электроизоляционных материалах.
 \vfill

\vfill

\newpage


\shapk
\bilet{Экзаменационный билет}
\setcounter{zad}{0}

\vfill
\z Состав и принцип действия механического привода.
 \vfill
\z Воздействие колебаний на технологический процесс.
 \vfill

\vfill

\newpage


\shapk
\bilet{Экзаменационный билет}
\setcounter{zad}{0}

\vfill
\z Оценка надежности технических систем по типовым моделям.
 \vfill
\z Общие сведения электротехнических материалов.
 \vfill

\vfill

\newpage


\shapk
\bilet{Экзаменационный билет}
\setcounter{zad}{0}

\vfill
\z Понятие об электрическом поле и его использование в технике.
 \vfill
\z Перспективы направления повышения надежности и качества функционирования систем сервиса.
 \vfill

\vfill

\newpage


\shapk
\bilet{Экзаменационный билет}
\setcounter{zad}{0}

\vfill
\z Схема обеспечения надежности.
 \vfill
\z Понятие о валах и осях.
 \vfill

\vfill

\newpage


\shapk
\bilet{Экзаменационный билет}
\setcounter{zad}{0}

\vfill
\z Кинетический анализ кривошипно-ползунного механизма.
 \vfill
\z Понятия о технологическом процессе сборки (монтажа) изделий.
 \vfill

\vfill

\newpage


\shapk
\bilet{Экзаменационный билет}
\setcounter{zad}{0}

\vfill
\z Основные принципы обеспечения высокой конструктивной надежности.
 \vfill
\z Типы механических передач.
 \vfill

\vfill

\newpage


\shapk
\bilet{Экзаменационный билет}
\setcounter{zad}{0}

\vfill
\z Усилия, действующие в кривошипно-ползунном механизме.
 \vfill
\z Силы инерции вращающихся масс.
 \vfill

\vfill

\newpage


\shapk
\bilet{Экзаменационный билет}
\setcounter{zad}{0}

\vfill
\z Типы производств в сфере сервиса.
 \vfill
\z Подшипники качения и скольжения.
 \vfill

\vfill

\newpage


\shapk
\bilet{Экзаменационный билет}
\setcounter{zad}{0}

\vfill
\z Понятие о крутильных колебаниях валов.
 \vfill
\z Описание технологических процессов по способу и характеру воздействия на изделие.
 \vfill

\vfill

\newpage


\shapk
\bilet{Экзаменационный билет}
\setcounter{zad}{0}

\vfill
\z Понятия о производственном и технологическом процессе сферы сервиса.
 \vfill
\z Соединительные муфты приводов.
 \vfill

\vfill

\newpage


\shapk
\bilet{Экзаменационный билет}
\setcounter{zad}{0}

\vfill
\z Расчет деталей машин на жесткость.
 \vfill
\z Описание технологических процессов по степени детализации.
 \vfill

\vfill

\newpage


\shapk
\bilet{Экзаменационный билет}
\setcounter{zad}{0}

\vfill
\z Виды и состав технологических процессов.
 \vfill
\z Типы соединений деталей и узлов.
 \vfill

\vfill

\newpage


\shapk
\bilet{Экзаменационный билет}
\setcounter{zad}{0}

\vfill
\z Расчет пружин.
 \vfill
\z Понятие о динамических нагрузках.
 \vfill

\vfill

\newpage



\end{document}