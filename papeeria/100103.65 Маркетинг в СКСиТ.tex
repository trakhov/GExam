\documentclass[
	14pt,
	a4paper,
	]
	{scrartcl}

\usepackage{mysty}

\newcommand{\spec}{100103.65}
\newcommand{\disc}{Маркетинг\\ \multicolumn{2}{r}{ в СКСиТ }}
\newcommand{\kafedra}{МиЭ}

\usepackage[
	papersize={210mm,99mm},
	top=.5cm,
	left=1.5cm,
	right=1.5cm,
	bottom=.5cm
	]{geometry}

\pagestyle{empty}

\begin{document}


\shapk
\bilet{Экзаменационный билет}
\setcounter{zad}{0}

\vfill
\z Маркетинг, основные понятия маркетинга (нужда, желание, товар, сделка, обмен, рынок, цена, покупательский спрос) и их содержание.
 \vfill
\z Особенности внешнеэкономического маркетинга в сфере услуг. \vfill

\vfill

\newpage


\shapk
\bilet{Экзаменационный билет}
\setcounter{zad}{0}

\vfill
\z Маркетинг услуг, характеристика услуги как экономической категории.
 \vfill
\z Международный маркетинг.
 \vfill

\vfill

\newpage


\shapk
\bilet{Экзаменационный билет}
\setcounter{zad}{0}

\vfill
\z Сущность, цели и задачи маркетинга.
 \vfill
\z Понятие контроллинга в системе маркетинга
 \vfill

\vfill

\newpage


\shapk
\bilet{Экзаменационный билет}
\setcounter{zad}{0}

\vfill
\z Функции и их содержание, принципы маркетинга.
 \vfill
\z Контроль в системе маркетинга.
 \vfill

\vfill

\newpage


\shapk
\bilet{Экзаменационный билет}
\setcounter{zad}{0}

\vfill
\z Виды маркетинга и их содержание.
 \vfill
\z Организационные структуры службы маркетинга и принципы их построения.
 \vfill

\vfill

\newpage


\shapk
\bilet{Экзаменационный билет}
\setcounter{zad}{0}

\vfill
\z Типы маркетинга в зависимости от состояния спроса, задачи маркетинга, соответствующие этим состояниям.
 \vfill
\z Планирование маркетинговой деятельности, формирование плана маркетинга.
 \vfill

\vfill

\newpage


\shapk
\bilet{Экзаменационный билет}
\setcounter{zad}{0}

\vfill
\z Концепции маркетинга и их эволюция.
 \vfill
\z «Паблик рилейшинс». Содержание, цели, задачи.
 \vfill

\vfill

\newpage


\shapk
\bilet{Экзаменационный билет}
\setcounter{zad}{0}

\vfill
\z Особенности маркетинга в сфере услуг.
 \vfill
\z Формы и виды рекламных сообщений, принципы их разработки. 
 \vfill

\vfill

\newpage


\shapk
\bilet{Экзаменационный билет}
\setcounter{zad}{0}

\vfill
\z Маркетинговая среда предприятия, понятие общая характеристика.
 \vfill
\z Средств распространения рекламы, их характеристика, преимущества и недостатки.
 \vfill

\vfill

\newpage


\shapk
\bilet{Экзаменационный билет}
\setcounter{zad}{0}

\vfill
\z Основные факторы макросреды предприятия сферы услуг. 
 \vfill
\z Виды рекламы и правовые основы рекламной деятельности.
 \vfill

\vfill

\newpage


\shapk
\bilet{Экзаменационный билет}
\setcounter{zad}{0}

\vfill
\z Основные факторы микросреды предприятия сферы услуг.
 \vfill
\z Понятие «рекламы» как средства коммуникации в системе продвижения услуги.
 \vfill

\vfill

\newpage


\shapk
\bilet{Экзаменационный билет}
\setcounter{zad}{0}

\vfill
\z Комплекс маркетинга на предприятиях в сфере услуг.
 \vfill
\z Понятие и методы стимулирования сбыта на услуги.
 \vfill

\vfill

\newpage


\shapk
\bilet{Экзаменационный билет}
\setcounter{zad}{0}

\vfill
\z Маркетинговые исследования в сфере услуг, его содержание и сущность, виды маркетинговых исследований.
 \vfill
\z Продвижение услуг. 
 \vfill

\vfill

\newpage


\shapk
\bilet{Экзаменационный билет}
\setcounter{zad}{0}

\vfill
\z Этапы маркетингового исследования и их характеристика.
 \vfill
\z Понятие интегрированных маркетинговых коммуникаций в системе маркетинга.
 \vfill

\vfill

\newpage


\shapk
\bilet{Экзаменационный билет}
\setcounter{zad}{0}

\vfill
\z Система маркетинговой информации, характеристика ее основных элементов.
 \vfill
\z Прямые продажи, характеристика и особенности прямых продаж в сфере услуг.
 \vfill

\vfill

\newpage


\shapk
\bilet{Экзаменационный билет}
\setcounter{zad}{0}

\vfill
\z Маркетинговые исследования и выбор источников информации.
 \vfill
\z Методы продажи услуг.
 \vfill

\vfill

\newpage


\shapk
\bilet{Экзаменационный билет}
\setcounter{zad}{0}

\vfill
\z Методы сбора первичных данных и их характеристика
 \vfill
\z Ценовые стратегии: виды, характеристика и их значение. 
 \vfill

\vfill

\newpage


\shapk
\bilet{Экзаменационный билет}
\setcounter{zad}{0}

\vfill
\z Методы получения и обработки маркетинговой вторичной информации.
 \vfill
\z Методы ценообразования на услуги.
 \vfill

\vfill

\newpage


\shapk
\bilet{Экзаменационный билет}
\setcounter{zad}{0}

\vfill
\z Понятие «рынок услуг». Классификация и организационные формы рынка услуг.
 \vfill
\z Теоретические основы ценообразования: понятие цены и ее значение в рыночной экономике, психологические границы цен.
 \vfill

\vfill

\newpage


\shapk
\bilet{Экзаменационный билет}
\setcounter{zad}{0}

\vfill
\z Анализ рынка услуг, методы и характеристика основных элементов рыночного исследования.
 \vfill
\z Характеристика основных этапов процесса разработки нового товара/услуги.
 \vfill

\vfill

\newpage


\shapk
\bilet{Экзаменационный билет}
\setcounter{zad}{0}

\vfill
\z Исследование конкурентов и потребителей, позиционирование товара и услуг на потребительском рынке.
 \vfill
\z Сервис и гарантийное обслуживание в системе маркетинга.
 \vfill

\vfill

\newpage


\shapk
\bilet{Экзаменационный билет}
\setcounter{zad}{0}

\vfill
\z Конъюнктурные и прогнозные исследования рынка услуг.
 \vfill
\z Понятие «торговой марки», значение использования торговой марки в маркетинге.
 \vfill

\vfill

\newpage


\shapk
\bilet{Экзаменационный билет}
\setcounter{zad}{0}

\vfill
\z Сегментирование рынка, понятие, виды и критерии сегментации.
 \vfill
\z Конкурентоспособность товара/услуги, конкурентное преимущество.
 \vfill

\vfill

\newpage


\shapk
\bilet{Экзаменационный билет}
\setcounter{zad}{0}

\vfill
\z Особенности сегментации рынка в сфере услуг.
 \vfill
\z Ассортиментная политика предприятия.
 \vfill

\vfill

\newpage


\shapk
\bilet{Экзаменационный билет}
\setcounter{zad}{0}

\vfill
\z Определение понятия «позиционирование» товара на рынке. Основные принципы позиционирования услуги.
 \vfill
\z Концепция «жизненного цикла товара/услуги» и содержание маркетинговых усилий на разных этапах ЖЦТ.
 \vfill

\vfill

\newpage


\shapk
\bilet{Экзаменационный билет}
\setcounter{zad}{0}

\vfill
\z Характеристика понятия «стратегия маркетинга».
 \vfill
\z Понятие качества товаров и услуг, классификация потребительских свойств товара и услуги.
 \vfill

\vfill

\newpage


\shapk
\bilet{Экзаменационный билет}
\setcounter{zad}{0}

\vfill
\z Понятие «товарная стратегия» и ее содержание.
 \vfill
\z Услуга в системе маркетинга. Основные виды классификации услуг.
 \vfill

\vfill

\newpage



\end{document}