\documentclass[
	14pt,
	a4paper,
	]
	{scrartcl}

\usepackage{mysty}

\newcommand{\spec}{100103.65}
\newcommand{\disc}{Информатика \\}
\newcommand{\kafedra}{ЕМиТД}

\usepackage[
	papersize={210mm,99mm},
	top=.5cm,
	left=1.5cm,
	right=1.5cm,
	bottom=.5cm
	]{geometry}

\pagestyle{empty}

\begin{document}


\shapk
\bilet{Экзаменационный билет}
\setcounter{zad}{0}

\vfill
\z Основные компоненты ПЭВМ.
 \vfill
\z Подпрограмма, обращение к ней, возврат из подпрограммы. \vfill

\vfill

\newpage


\shapk
\bilet{Экзаменационный билет}
\setcounter{zad}{0}

\vfill
\z Структура программного обеспечения ЭВМ.
 \vfill
\z Массивы (размерность, типы, заполнение). 
 \vfill

\vfill

\newpage


\shapk
\bilet{Экзаменационный билет}
\setcounter{zad}{0}

\vfill
\z Функции операционной системы. Интерфейс пользователя в MS DOS.
 \vfill
\z Операторы графики.
 \vfill

\vfill

\newpage


\shapk
\bilet{Экзаменационный билет}
\setcounter{zad}{0}

\vfill
\z Управление файловой системой в MS DOS.
 \vfill
\z Стандартные функции. Определение функции пользователя. 
 \vfill

\vfill

\newpage


\shapk
\bilet{Экзаменационный билет}
\setcounter{zad}{0}

\vfill
\z Загрузка MS DOS в оперативную память с диска.
 \vfill
\z Операторы ввода/вывода информации. 
 \vfill

\vfill

\newpage


\shapk
\bilet{Экзаменационный билет}
\setcounter{zad}{0}

\vfill
\z Операционная оболочка Norton Commander.
 \vfill
\z Цикл, тело цикла, параметры цикла. 
 \vfill

\vfill

\newpage


\shapk
\bilet{Экзаменационный билет}
\setcounter{zad}{0}

\vfill
\z Антивирусные программы и программы-архиваторы.
 \vfill
\z Условный оператор, полный и неполный варианты. 
 \vfill

\vfill

\newpage


\shapk
\bilet{Экзаменационный билет}
\setcounter{zad}{0}

\vfill
\z Линейные и разветвляющиеся алгоритмы.
 \vfill
\z Оператор присваивания. Арифметические выражения. 
 \vfill

\vfill

\newpage


\shapk
\bilet{Экзаменационный билет}
\setcounter{zad}{0}

\vfill
\z Циклические алгоритмы.
 \vfill
\z Функции обработки символьных (литерных) переменных. 
 \vfill

\vfill

\newpage


\shapk
\bilet{Экзаменационный билет}
\setcounter{zad}{0}

\vfill
\z Языки программирования.
 \vfill
\z Типы переменных (целый, вещественный, символьный, логический). 
 \vfill

\vfill

\newpage


\shapk
\bilet{Экзаменационный билет}
\setcounter{zad}{0}

\vfill
\z Программирование разветвляющихся алгоритмов на языке Бэйсик. Условный оператор.
 \vfill
\z Переменная: имя и назначение. Идентификатор. 
 \vfill

\vfill

\newpage


\shapk
\bilet{Экзаменационный билет}
\setcounter{zad}{0}

\vfill
\z Программирование циклических алгоритмов на языке Бэйсик. Циклический оператор.
 \vfill
\z Этапы решения задач на ЭВМ. 
 \vfill

\vfill

\newpage


\shapk
\bilet{Экзаменационный билет}
\setcounter{zad}{0}

\vfill
\z Системы счисления.
 \vfill
\z Конструирование алгоритмов методом последовательной детализации. Вспомогательный алгоритм. 
 \vfill

\vfill

\newpage


\shapk
\bilet{Экзаменационный билет}
\setcounter{zad}{0}

\vfill
\z Перевод чисел из одной системы счисления в другую.
 \vfill
\z Циклический алгоритм. 
 \vfill

\vfill

\newpage


\shapk
\bilet{Экзаменационный билет}
\setcounter{zad}{0}

\vfill
\z Логические основы ЭВМ.
 \vfill
\z Разветвляющийся алгоритм. 
 \vfill

\vfill

\newpage


\shapk
\bilet{Экзаменационный билет}
\setcounter{zad}{0}

\vfill
\z Видеосистема. ПЭВМ.
 \vfill
\z Понятия алгоритма. Исполнители алгоритмов. 
 \vfill

\vfill

\newpage


\shapk
\bilet{Экзаменационный билет}
\setcounter{zad}{0}

\vfill
\z Накопители на жестких и гибких магнитных дисках.
 \vfill
\z Информация. Единицы измерения информации. 
 \vfill

\vfill

\newpage


\shapk
\bilet{Экзаменационный билет}
\setcounter{zad}{0}

\vfill
\z Принтеры. Типы. Способы печати. Работа в сети.
 \vfill
\z Информатизация общества. Развитие вычислительной техники. 
 \vfill

\vfill

\newpage


\shapk
\bilet{Экзаменационный билет}
\setcounter{zad}{0}

\vfill
\z Интерфейс и объекты табличного процессора MS Excel.
 \vfill
\z Локальные и телекоммуникационные компьютерные сети. Компьютерная почта. 
 \vfill

\vfill

\newpage


\shapk
\bilet{Экзаменационный билет}
\setcounter{zad}{0}

\vfill
\z Данные, хранимые в ячейках табличного процессора MS Excel.
 \vfill
\z Текстовый редактор. Назначение и основные функции. 
 \vfill

\vfill

\newpage


\shapk
\bilet{Экзаменационный билет}
\setcounter{zad}{0}

\vfill
\z Режимы работы табличного процессора MS Excel.
 \vfill
\z Электронные таблицы. Назначение и основные функции. 
 \vfill

\vfill

\newpage


\shapk
\bilet{Экзаменационный билет}
\setcounter{zad}{0}

\vfill
\z Копирование и перемещение информации в табличном процессоре Excel. Понятие абсолютной и относительной ссылок.
 \vfill
\z Системы управления базами данных (СУБД). Назначение и основные функции. 
 \vfill

\vfill

\newpage


\shapk
\bilet{Экзаменационный билет}
\setcounter{zad}{0}

\vfill
\z Способы копирования и перемещения информации в табличном процессоре MS Excel. Объединение электронных таблиц.
 \vfill
\z Машинная графика. Назначение и основные функции. 
 \vfill

\vfill

\newpage


\shapk
\bilet{Экзаменационный билет}
\setcounter{zad}{0}

\vfill
\z Сортировка и фильтрация данных в электронных таблицах.
 \vfill
\z Файл. Файловая система. 
 \vfill

\vfill

\newpage


\shapk
\bilet{Экзаменационный билет}
\setcounter{zad}{0}

\vfill
\z Сводные таблицы в табличном процессоре MS Excel.
 \vfill
\z Операционная система (назначение, состав, загрузка). 
 \vfill

\vfill

\newpage


\shapk
\bilet{Экзаменационный билет}
\setcounter{zad}{0}

\vfill
\z Основные понятия и классификация баз данных.
 \vfill
\z Матричный принцип печати. Принтер. 
 \vfill

\vfill

\newpage


\shapk
\bilet{Экзаменационный билет}
\setcounter{zad}{0}

\vfill
\z Модели данных.
 \vfill
\z Внешняя память (дисководы, диски). 
 \vfill

\vfill

\newpage


\shapk
\bilet{Экзаменационный билет}
\setcounter{zad}{0}

\vfill
\z Создание структуры таблиц в СУБД MS Access.
 \vfill
\z Структура памяти. Оперативная память (область программ пользователя и видеопамять). Постоянное запоминающее устройство. 
 \vfill

\vfill

\newpage


\shapk
\bilet{Экзаменационный билет}
\setcounter{zad}{0}

\vfill
\z Создание схемы данных в СУБД MS Access.
 \vfill
\z Разрядность процессора. Адресное пространство процессора. 
 \vfill

\vfill

\newpage


\shapk
\bilet{Экзаменационный билет}
\setcounter{zad}{0}

\vfill
\z Загрузка, просмотр и корректировка базы данных в СУБД MS Access.
 \vfill
\z Компьютер. Магистрально-модульный принцип построения. 
 \vfill

\vfill

\newpage


\shapk
\bilet{Экзаменационный билет}
\setcounter{zad}{0}

\vfill
\z Запросы к базе данных и технология их конструирования в СУБД Access.
 \vfill
\z Запросы на сортировку, с критериями поиска, с параметрами в СУБД MS Access.
 \vfill

\vfill

\newpage



\end{document}