\documentclass[
	11pt,
	a4paper,
	]
	{article}

\usepackage{mysty, xparse, GExam, fancyhdr}

\usepackage[
	top = .7cm,
	right = 1cm,
	bottom = 1.2cm,
	left = 1cm,
	headheight = 14pt
	]{geometry}


\pagestyle{fancy}
\renewcommand{\headrulewidth}{0pt}
\renewcommand{\footrulewidth}{0pt}


\lhead{\begin{tikzpicture}[remember picture, overlay]
	\draw[dashed] (current page.north west) ++ (0, -99mm) -- +(210mm, 0);
	\draw[dashed] (current page.north west) ++ (0, -198mm) -- +(210mm, 0);
\end{tikzpicture}}
\cfoot{}



\begin{document}

	

\begin{minipage}[t][\miniH]{\miniL}\centering
	\topGE * {100103.65} {Социально-культурный сервис и туризм}
	\bilGE {1} 
		{
			Культура края в сфере СКСиТ.
		}{
			Этапы развития международного туризма.
		}{
			Туристские ресурсы Австралии и Океании.
		}
	\lowGE
\end{minipage}

\vfill



\begin{minipage}[t][\miniH]{\miniL}\centering
	\topGE * {100103.65} {Социально-культурный сервис и туризм}
	\bilGE {2} 
		{
			Актуальные проблемы современного сервиса.
		}{
			Этапы развития туризма в России.
		}{
			Туристские ресурсы Латинской Америки.
		}
	\lowGE
\end{minipage}

\vfill



\begin{minipage}[t][\miniH]{\miniL}\centering
	\topGE * {100103.65} {Социально-культурный сервис и туризм}
	\bilGE {3} 
		{
			Речевая коммуникация в сфере СКСиТ.
		}{
			Социокультурная мотивация путешествий: история и современность.
		}{
			Туристские ресурсы Северной Америки.
		}
	\lowGE
\end{minipage}





\begin{minipage}[t][\miniH]{\miniL}\centering
	\topGE * {100103.65} {Социально-культурный сервис и туризм}
	\bilGE {4} 
		{
			Типология конфликтов в сфере СКСиТ и способы их регулирования.
		}{
			Великие географические открытия и великие путешественники.
		}{
			Туристские ресурсы Китая.
		}
	\lowGE
\end{minipage}

\vfill



\begin{minipage}[t][\miniH]{\miniL}\centering
	\topGE * {100103.65} {Социально-культурный сервис и туризм}
	\bilGE {5} 
		{
			Профессиональная этика сферы СКСиТ.
		}{
			Теория туризма: основные понятия.
		}{
			Туристские ресурсы Японии.
		}
	\lowGE
\end{minipage}

\vfill



\begin{minipage}[t][\miniH]{\miniL}\centering
	\topGE * {100103.65} {Социально-культурный сервис и туризм}
	\bilGE {6} 
		{
			Профессиональные требования к организатору в сфере СКСиТ.
		}{
			Туризм как социально-культурное явление.
		}{
			Туристские ресурсы Юго-Восточной Азии.
		}
	\lowGE
\end{minipage}





\begin{minipage}[t][\miniH]{\miniL}\centering
	\topGE * {100103.65} {Социально-культурный сервис и туризм}
	\bilGE {7} 
		{
			Документационное обеспечение деятельности предприятий сферы СКСиТ.
		}{
			Нормы и правовое регулирование в сфере туризма.
		}{
			Туристские ресурсы Южной Азии.
		}
	\lowGE
\end{minipage}

\vfill



\begin{minipage}[t][\miniH]{\miniL}\centering
	\topGE * {100103.65} {Социально-культурный сервис и туризм}
	\bilGE {8} 
		{
			Связи с общественностью в сфере СКСиТ.
		}{
			Государственная политика в области туризма.
		}{
			Туристские ресурсы Африки.
		}
	\lowGE
\end{minipage}

\vfill



\begin{minipage}[t][\miniH]{\miniL}\centering
	\topGE * {100103.65} {Социально-культурный сервис и туризм}
	\bilGE {9} 
		{
			Реклама в СКСиТ.
		}{
			Региональные целевые программы развития туризма и отдыха в РФ.
		}{
			Туристские ресурсы Израиля.
		}
	\lowGE
\end{minipage}





\begin{minipage}[t][\miniH]{\miniL}\centering
	\topGE * {100103.65} {Социально-культурный сервис и туризм}
	\bilGE {10} 
		{
			Инновации в СКСиТ: источники, проблемы внедрения.
		}{
			Договорные отношения в сфере туризма.
		}{
			Туристские ресурсы Польши, Чехии, Словакии, Венгрии.
		}
	\lowGE
\end{minipage}

\vfill



\begin{minipage}[t][\miniH]{\miniL}\centering
	\topGE * {100103.65} {Социально-культурный сервис и туризм}
	\bilGE {11} 
		{
			Порядок создания и регистрации нового предприятия в сфере СКСиТ.
		}{
			Всемирные туристские организации и их деятельность.
		}{
			Туристские ресурсы Греции и Турции.
		}
	\lowGE
\end{minipage}

\vfill



\begin{minipage}[t][\miniH]{\miniL}\centering
	\topGE * {100103.65} {Социально-культурный сервис и туризм}
	\bilGE {12} 
		{
			Стандартизация и сертификация в сфере СКСиТ.
		}{
			Турист как потребитель туристских услуг.
		}{
			Туристские ресурсы Бенилюкса.
		}
	\lowGE
\end{minipage}





\begin{minipage}[t][\miniH]{\miniL}\centering
	\topGE * {100103.65} {Социально-культурный сервис и туризм}
	\bilGE {13} 
		{
			Специфика маркетинга в сфере СКСиТ.
		}{
			Тенденции развития современного международного туризма.
		}{
			Туристские ресурсы Испании.
		}
	\lowGE
\end{minipage}

\vfill



\begin{minipage}[t][\miniH]{\miniL}\centering
	\topGE * {100103.65} {Социально-культурный сервис и туризм}
	\bilGE {14} 
		{
			Специфика менеджмента в сфере СКСиТ.
		}{
			Тенденции развития современного российского туризма.
		}{
			Туристские ресурсы Италии.
		}
	\lowGE
\end{minipage}

\vfill



\begin{minipage}[t][\miniH]{\miniL}\centering
	\topGE * {100103.65} {Социально-культурный сервис и туризм}
	\bilGE {15} 
		{
			Защита прав потребителя в сфере СКСиТ.
		}{
			Проблемы и перспективы развития особых экономических зон туристско-рекреационного типа.
		}{
			Туристские ресурсы Германии.
		}
	\lowGE
\end{minipage}





\begin{minipage}[t][\miniH]{\miniL}\centering
	\topGE * {100103.65} {Социально-культурный сервис и туризм}
	\bilGE {16} 
		{
			Тенденции и проблемы развития малых предприятий в сфере СКСиТ.
		}{
			Гостиница как основной элемент туристской инфраструктуры.
		}{
			Туристские ресурсы Франции.
		}
	\lowGE
\end{minipage}

\vfill



\begin{minipage}[t][\miniH]{\miniL}\centering
	\topGE * {100103.65} {Социально-культурный сервис и туризм}
	\bilGE {17} 
		{
			Типология и структура предпринимательской деятельности в сфере СКСиТ.
		}{
			Транспортное обслуживание в туризме.
		}{
			Туристские ресурсы Великобритании.
		}
	\lowGE
\end{minipage}

\vfill



\begin{minipage}[t][\miniH]{\miniL}\centering
	\topGE * {100103.65} {Социально-культурный сервис и туризм}
	\bilGE {18} 
		{
			Особенности экономической деятельности сфере СКСиТ.
		}{
			Виды и специфика экскурсионного обслуживания туристов.
		}{
			Туристские ресурсы Закавказья.
		}
	\lowGE
\end{minipage}





\begin{minipage}[t][\miniH]{\miniL}\centering
	\topGE * {100103.65} {Социально-культурный сервис и туризм}
	\bilGE {19} 
		{
			Договорные отношения в сфере СКСиТ.
		}{
			Технология разработки маршрутов и формирования туров.
		}{
			Туристские ресурсы Казахстана и стран Средней Азии.
		}
	\lowGE
\end{minipage}

\vfill



\begin{minipage}[t][\miniH]{\miniL}\centering
	\topGE * {100103.65} {Социально-культурный сервис и туризм}
	\bilGE {20} 
		{
			Правовые основы обеспечения социально-культурного сервиса и туризма (на материале РФ).
		}{
			Анимация в туризме.
		}{
			Туристские ресурсы Украины, Беларуси, Молдовы.
		}
	\lowGE
\end{minipage}

\vfill



\begin{minipage}[t][\miniH]{\miniL}\centering
	\topGE * {100103.65} {Социально-культурный сервис и туризм}
	\bilGE {21} 
		{
			Международно-правовое обеспечение СКСиТ.
		}{
			Иностранцы в России: позитивные и негативные факторы в сфере туризма.
		}{
			Туристские ресурсы стран Балтии.
		}
	\lowGE
\end{minipage}





\begin{minipage}[t][\miniH]{\miniL}\centering
	\topGE * {100103.65} {Социально-культурный сервис и туризм}
	\bilGE {22} 
		{
			История сервиса в России (основные этапы).
		}{
			Россияне за рубежом: позитивные и негативные факторы в туризме.
		}{
			Туристские ресурсы Скандинавских стран.
		}
	\lowGE
\end{minipage}

\vfill



\begin{minipage}[t][\miniH]{\miniL}\centering
	\topGE * {100103.65} {Социально-культурный сервис и туризм}
	\bilGE {23} 
		{
			История сервиса за рубежом (основные этапы).
		}{
			Классификация туристских путешествий.
		}{
			Туристские ресурсы Черноморского побережья РФ и Приазовья.
		}
	\lowGE
\end{minipage}

\vfill



\begin{minipage}[t][\miniH]{\miniL}\centering
	\topGE * {100103.65} {Социально-культурный сервис и туризм}
	\bilGE {24} 
		{
			Сервисные технологии.
		}{
			Детский и молодёжный туризм: формы организации, проблемы и перспективы.
		}{
			Туристские ресурсы Северного Кавказа.
		}
	\lowGE
\end{minipage}





\begin{minipage}[t][\miniH]{\miniL}\centering
	\topGE * {100103.65} {Социально-культурный сервис и туризм}
	\bilGE {25} 
		{
			Понятие и виды сервисной деятельности.
		}{
			Социальный туризм: общемировые тенденции, пути возрождения социального туризма в России.
		}{
			Туристские ресурсы Сибири и Дальнего Востока.
		}
	\lowGE
\end{minipage}

\vfill



\begin{minipage}[t][\miniH]{\miniL}\centering
	\topGE * {100103.65} {Социально-культурный сервис и туризм}
	\bilGE {26} 
		{
			Формы, методы и ресурсная база социально-культурной деятельности.
		}{
			Экологический туризм: виды, тенденции, специфика.
		}{
			Туристские ресурсы Поволжья и Урала.
		}
	\lowGE
\end{minipage}

\vfill



\begin{minipage}[t][\miniH]{\miniL}\centering
	\topGE * {100103.65} {Социально-культурный сервис и туризм}
	\bilGE {27} 
		{
			Типы и виды социально-культурных институтов.
		}{
			Круизный туризм: виды, тенденции, специфика.
		}{
			Туристские ресурсы Севера европейской части России.
		}
	\lowGE
\end{minipage}





\begin{minipage}[t][\miniH]{\miniL}\centering
	\topGE * {100103.65} {Социально-культурный сервис и туризм}
	\bilGE {28} 
		{
			Социально-культурная деятельность: проблемы, тенденции развития.
		}{
			Сельский туризм: виды, тенденции, специфика.
		}{
			Туристские ресурсы Санкт-Петербурга и Ленинградской области.
		}
	\lowGE
\end{minipage}

\vfill



\begin{minipage}[t][\miniH]{\miniL}\centering
	\topGE * {100103.65} {Социально-культурный сервис и туризм}
	\bilGE {29} 
		{
			Потребности человека: основные концепции, динамика.
		}{
			Специализированный туризм: виды, тенденции, специфика.
		}{
			Туристские ресурсы Москвы и Подмосковья.
		}
	\lowGE
\end{minipage}

\vfill



\begin{minipage}[t][\miniH]{\miniL}\centering
	\topGE * {100103.65} {Социально-культурный сервис и туризм}
	\bilGE {30} 
		{
			Сервисология как область научного знания.
		}{
			Новые формы в современном туризме.
		}{
			Туристские ресурсы Центральной России.
		}
	\lowGE
\end{minipage}



	

\end{document}