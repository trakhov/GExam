\documentclass[
	11pt,
	a4paper,
	]
	{article}

\usepackage{mysty, xparse, GExam, fancyhdr}

\usepackage[
	top = .7cm,
	right = 1cm,
	bottom = 1.2cm,
	left = 1cm,
	headheight = 14pt
	]{geometry}


\pagestyle{fancy}
\renewcommand{\headrulewidth}{0pt}
\renewcommand{\footrulewidth}{0pt}


\lhead{\begin{tikzpicture}[remember picture, overlay]
	\draw[dashed] (current page.north west) ++ (0, -99mm) -- +(210mm, 0);
	\draw[dashed] (current page.north west) ++ (0, -198mm) -- +(210mm, 0);
\end{tikzpicture}}
\cfoot{}



\begin{document}

\smallGE * {100101.65} {Cервис}

	

\noindent\bilGE {1} 
	{
		Общероссийский классификатор услуг населению (ОКУН).
	}{
		Модельные ряды автомобилей основных производителей.
	}{
		Дооборудование и тюнинг. Классификация. Цели и задачи тюнинга.
	}

\bigskip

\noindent\bilGE {2} 
	{
		Сервисное обеспечение автомобильных дорог различных категорий.
	}{
		Основные технические характеристики автомобиля.
	}{
		Тюнинг двигателя автомобиля.
	}

\bigskip

\noindent\bilGE {3} 
	{
		Современные информационные технологии, применяемые в сервисной деятельности.
	}{
		Факторы, влияющие на приемистость и экономичность автомобиля.
	}{
		Тюнинг трансмиссии автомобиля.
	}

\bigskip

\noindent\bilGE {4} 
	{
		Типологические характеристики различных групп потребителей услуг.
	}{
		Надежность транспортного средства и ее показатели.
	}{
		Тюнинг подвесок автомобиля.
	}

\bigskip

\noindent\bilGE {5} 
	{
		Стратегии завоевания рынка, ассортиментная политика сервисных центров.
	}{
		Основные агрегаты и системы легкового автомобиля.
	}{
		Тюнинг тормозных механизмов автомобиля.
	}

\bigskip

\noindent\bilGE {6} 
	{
		Ценовые и неценовые факторы спроса и предложения на рынке услуг.
	}{
		Классификация и принцип действия двигателей автомобилей.
	}{
		Тюнинг электрооборудования автомобиля.
	}

\bigskip

\noindent\bilGE {7} 
	{
		Основные факторы, влияющие на качество предоставления услуги.
	}{
		Рабочий процесс двигателя внутреннего сгорания.
	}{
		Тюнинг кузова автомобиля. Аэрография.
	}

\bigskip

\noindent\bilGE {8} 
	{
		Стадии и процесс коммуникации сервис менеджера и клиента.
	}{
		Внешняя скоростная характеристика автомобильного двигателя.
	}{
		Тюнинг салона автомобиля.
	}

\bigskip

\noindent\bilGE {9} 
	{
		Порядок разработки и основные элеменнты бизнес-плана сервисного центра .
	}{
		Системы питания двигателей автомобилей и используемые виды топлива.
	}{
		Спортивный и спортивно-развлекательный тюнинг.
	}

\bigskip

\noindent\bilGE {10} 
	{
		Анализ рынка автосервисных услуг. Тенденция развития.
	}{
		Анализ рынка автосервисных услуг. Тенденция развития.
	}{
		Оборудование и технология предоставления услуги по оказанию помощи на дорогах.
	}

\bigskip

\noindent\bilGE {11} 
	{
		Сегментация рынка при проектировании сервисных центров.
	}{
		Системы зажигания двигателей внутреннего сгорания.
	}{
		Оборудование и технология предоставления услуг в дорожных сервисных центрах.
	}

\bigskip

\noindent\bilGE {12} 
	{
		Стратегии привлечения клиентов. Разработка медиа-плана.
	}{
		Системы электронного управления двигателем автомобиля.
	}{
		Технология предоставления услуги по прокату автомобилей и их принадлежностей.
	}

\bigskip

\noindent\bilGE {13} 
	{
		Поведение и действия сервис-менеджера в зависимости от типологии клиентов.
	}{
		Системы смазки двигателя внутреннего сгорания и использование моторного масла.
	}{
		Значение автосервисных центров в сфере услуг.
	}

\bigskip

\noindent\bilGE {14} 
	{
		Лицензирование и сертификация услуг в автосервисе.
	}{
		Система охлаждения двигателя, применяемые эксплуатационные материалы.
	}{
		Конкурентообразующие характеристики сервисных центров.
	}

\bigskip

\noindent\bilGE {15} 
	{
		Назначение и структура сервисной карты автомобиля.
	}{
		Форсирование двигателя внутреннего сгорания наддувом.
	}{
		Основные нормативные документы, регламентирующие работу сервисных центров.
	}

\bigskip

\noindent\bilGE {16} 
	{
		Технология проектирования услуги по тюнингу автомобиля.
	}{
		Основные агрегаты и узлы трансмиссии автомобиля, трансмиссионные масла.
	}{
		Основные разделы и содержание «Положения о техническом обслуживании и ремонте подвижного состава автомобильного транспорта».
	}

\bigskip

\noindent\bilGE {17} 
	{
		Формирование потребительских свойств автомобиля в соответствии с требованиями клиента.
	}{
		Конструкции и характеристики подвесок автомобиля.
	}{
		Основные разделы и содержание «Положения о техническом обслуживании и ремонте автотранспортных средств, принадлежащих гражданам (легковые и грузовые автомобили, автобусы, мини-трактора).
	}

\bigskip

\noindent\bilGE {18} 
	{
		Оборудование и технология предоставления услуги по подготовке и проведению технического осмотра.
	}{
		Конструкции и характеристики колес автомобиля.
	}{
		Основные разделы и содержание «Правила оказания услуг (выполнения работ) по ремонту автомототранспортных средств».
	}

\bigskip

\noindent\bilGE {19} 
	{
		Репутация и имидж автосервисного предприятия.
	}{
		Назначение и типы кузовов легковых автомобилей.
	}{
		Диагностика автотранспортных средств. Назначение и виды.
	}

\bigskip

\noindent\bilGE {20} 
	{
		Значение и формирование Brend-имиджа сервисных центров.
	}{
		Назначение, типы и конструкции рулевого управления автомобиля.
	}{
		Основное технологическое оборудование сервисных центров.
	}

\bigskip

\noindent\bilGE {21} 
	{
		Ценовая политика автосервисных предприятий.
	}{
		Назначение, типы и конструкции тормозных систем автомобиля, используемые эксплуатационные материалы.
	}{
		Контрольно-диагностическое оборудование сервисных центров.
	}

\bigskip

\noindent\bilGE {22} 
	{
		Кадровая политика автосервисных предприятий.
	}{
		Антиблокировочная система автомобиля. Назначение, принцип действия.
	}{
		Прогнозирование остаточного ресурса по результатам диагностирования.
	}

\bigskip

\noindent\bilGE {23} 
	{
		Понятие и характеристики контактной зоны взаимодействия с клиентом.
	}{
		Основные системы и узлы электрооборудования автомобиля.
	}{
		Планово-предупредительная система технического обслуживания и ремонта.
	}

\bigskip

\noindent\bilGE {24} 
	{
		Основные технико-экономические показатели сервисных центров.
	}{
		Техническое обслуживание и ремонт двигателя автомобиля.
	}{
		Способы ремонта и восстановления автомобильных деталей.
	}

\bigskip

\noindent\bilGE {25} 
	{
		Ответственность СЦ перед заказчиком. Нормативные документы.
	}{
		Техническое обслуживание и ремонт узлов трансмиссии автомобиля.
	}{
		Маршрутная технологическая карта ремонта систем и агрегатов автомобиля.
	}

\bigskip

\noindent\bilGE {26} 
	{
		Классификация и основные характеристики автосервисных предприятий.
	}{
		Техническое обслуживание и ремонт подвески автомобиля.
	}{
		Операционная технологическая карта.
	}

\bigskip

\noindent\bilGE {27} 
	{
		Производственная структура предприятий автосервиса.
	}{
		Техническое обслуживание и ремонт кузова автомобиля.
	}{
		Расчет числа постов ремонтной зоны автосервисных предприятий.
	}

\bigskip

\noindent\bilGE {28} 
	{
		Организационная структура управления сервисным центром.
	}{
		Техническое обслуживание и ремонт рулевого управления.
	}{
		Определение площадей участков и подсобных помещений сервисных центров.
	}

\bigskip

\noindent\bilGE {29} 
	{
		Системы автоматизированного управления предприятиями автосервиса.
	}{
		Техническое обслуживание и ремонт тормозной системы.
	}{
		Трудоемкость при техническом обслуживании и ремонте автотранспортных средств.
	}

\bigskip

\noindent\bilGE {30} 
	{
		Определение свободной доли рынка. Расчет годового объема работ автосервисных предприятий.
	}{
		Техническое обслуживание и ремонт электрооборудования.
	}{
		Потребительские и эксплуатационные свойства автомобилей.
	}

\bigskip

\end{document}