\documentclass[
	11pt,
	a4paper,
	]
	{article}

\usepackage{mysty, xparse, GExam, fancyhdr}

\usepackage[
	top = .7cm,
	right = 1cm,
	bottom = 1.2cm,
	left = 1cm,
	headheight = 14pt
	]{geometry}


\pagestyle{fancy}
\renewcommand{\headrulewidth}{0pt}
\renewcommand{\footrulewidth}{0pt}


\lhead{\begin{tikzpicture}[remember picture, overlay]
	\draw[dashed] (current page.north west) ++ (0, -99mm) -- +(210mm, 0);
	\draw[dashed] (current page.north west) ++ (0, -198mm) -- +(210mm, 0);
\end{tikzpicture}}
\cfoot{}



\begin{document}

	

\begin{minipage}[t][\miniH]{\miniL}\centering
	\topGE * {080507.65} {Менеджмент организации}[Менеджмент в таможенном деле]
	\bilGE {1} 
		{
			Роль менеджмента в повышении эффективности хозяйствования в России. Проблемы развития менеджмента в современной России.
		}{
			Спрос и предложения, факторы их определяющие. Закон спроса и предложения. Эластичность спроса и предложения.
		}{
			Общие положения о таможенной стоимости. Декларирование таможенной стоимости товаров.
		}
	\lowGE
\end{minipage}

\vfill



\begin{minipage}[t][\miniH]{\miniL}\centering
	\topGE * {080507.65} {Менеджмент организации}[Менеджмент в таможенном деле]
	\bilGE {2} 
		{
			Организация как основа современного менеджмента. Жизненный цикл организации
		}{
			Валовой внутренний продукт (Валовой национальный продукт): понятие, методы расчета.
		}{
			Налог на прибыль: база начисления, действующая ставка, порядок расчета.
		}
	\lowGE
\end{minipage}

\vfill



\begin{minipage}[t][\miniH]{\miniL}\centering
	\topGE * {080507.65} {Менеджмент организации}[Менеджмент в таможенном деле]
	\bilGE {3} 
		{
			Менеджмент как наука: сущность, структура и содержание.
		}{
			Номинальный и реальный ВВП. Дефлятор ВВП (ВНП)
		}{
			Контроль таможенной стоимости товаров. Решения в отношении таможенной стоимости товаров. Корректировка таможенной стоимости товаров.
		}
	\lowGE
\end{minipage}





\begin{minipage}[t][\miniH]{\miniL}\centering
	\topGE * {080507.65} {Менеджмент организации}[Менеджмент в таможенном деле]
	\bilGE {4} 
		{
			Цели в системе современного менеджмента, основные требования, предъявляемые к ним Основные виды разделения управленческого труда: сущность, содержание
		}{
			Государственное регулирование бизнеса. Задачи государственного регулирования предпринимательской деятельности в сфере околотаможенного сервиса.
		}{
			Ответственность налогоплательщика за нарушение налогового законодательства.
		}
	\lowGE
\end{minipage}

\vfill



\begin{minipage}[t][\miniH]{\miniL}\centering
	\topGE * {080507.65} {Менеджмент организации}[Менеджмент в таможенном деле]
	\bilGE {5} 
		{
			Система функций современного менеджмента. Сущность и основные виды.
		}{
			Организационно-правовые формы хозяйствования и их характеристики.
		}{
			Общие положения о стране происхождения товаров. Подтверждение страны происхождения товаров.
		}
	\lowGE
\end{minipage}

\vfill



\begin{minipage}[t][\miniH]{\miniL}\centering
	\topGE * {080507.65} {Менеджмент организации}[Менеджмент в таможенном деле]
	\bilGE {6} 
		{
			Организация реализации принятых решений как одна из основных функций менеджмента. Критерии принятия решений в условиях неопределенности.
		}{
			Доходы предприятий сферы околотаможенного сервиса: понятие, виды, значение и состав. Пути повышения доходов.
		}{
			Цена, структура цены. Методы ценообразования.
		}
	\lowGE
\end{minipage}





\begin{minipage}[t][\miniH]{\miniL}\centering
	\topGE * {080507.65} {Менеджмент организации}[Менеджмент в таможенном деле]
	\bilGE {7} 
		{
			Сущность и содержание функции контроля хода выполнения принятых решений.
		}{
			Затраты предприятий сферы околотаможенного сервиса: их сущность, классификация и основные направления оптимизации.
		}{
			Декларация о происхождении товаров. Представление документов, подтверждающих страну происхождения товаров.
		}
	\lowGE
\end{minipage}

\vfill



\begin{minipage}[t][\miniH]{\miniL}\centering
	\topGE * {080507.65} {Менеджмент организации}[Менеджмент в таможенном деле]
	\bilGE {8} 
		{
			Современные школы менеджмента: общая характеристика.
		}{
			Формы и методы налогообложения предприятий сферы околотаможенного сервиса.
		}{
			Сроки уплаты таможенных пошлин, налогов. Изменение сроков уплаты таможенных пошлин, налогов. Порядок уплаты таможенных пошлин, налогов.
		}
	\lowGE
\end{minipage}

\vfill



\begin{minipage}[t][\miniH]{\miniL}\centering
	\topGE * {080507.65} {Менеджмент организации}[Менеджмент в таможенном деле]
	\bilGE {9} 
		{
			Особенности японской модели менеджмента: общая характеристика. Характер принятия управленческих решений и ответственности на японских фирмах.
		}{
			Прибыль как экономическая категория, ее сущность и значение.
		}{
			Таможенная процедура – временный ввоз. Назначение и порядок применения.
		}
	\lowGE
\end{minipage}





\begin{minipage}[t][\miniH]{\miniL}\centering
	\topGE * {080507.65} {Менеджмент организации}[Менеджмент в таможенном деле]
	\bilGE {10} 
		{
			Характерные черты американской практики менеджмента: общая характеристика. Сравнительная характеристика американской и японской моделей менеджмента.
		}{
			Рентабельность предприятий сферы околотаможенного сервиса: понятие, значение и методы расчета. Факторы, влияющие на рентабельность предприятий.
		}{
			Таможенная процедура – переработка на таможенной территории.
		}
	\lowGE
\end{minipage}

\vfill



\begin{minipage}[t][\miniH]{\miniL}\centering
	\topGE * {080507.65} {Менеджмент организации}[Менеджмент в таможенном деле]
	\bilGE {11} 
		{
			Основные цели современного предприятия. Роль предприятий сферы околотаможенного сервиса в экономике страны
		}{
			Способы выявления влияния факторов на изменения результативного показателя в экономическом анализе деятельности предприятия в сфере околотаможенного сервиса
		}{
			Место и время совершения таможенных операций, связанных с помещением товаров под таможенную процедуру, выпуск для внутреннего потребления.
		}
	\lowGE
\end{minipage}

\vfill



\begin{minipage}[t][\miniH]{\miniL}\centering
	\topGE * {080507.65} {Менеджмент организации}[Менеджмент в таможенном деле]
	\bilGE {12} 
		{
			Организационная структура, основные классификации. Структуры, ориентированные на нововведения. Основные принципы построения современных организационных структур.
		}{
			Конкуренция: ее сущность, виды и роль в механизме функционирования рынка. Особенности формирования и использования конкурентных преимуществ в сфере околотаможенного сервиса.
		}{
			Таможенная экспертиза при проведении таможенного контроля. Порядок и срок проведения таможенной экспертизы.
		}
	\lowGE
\end{minipage}





\begin{minipage}[t][\miniH]{\miniL}\centering
	\topGE * {080507.65} {Менеджмент организации}[Менеджмент в таможенном деле]
	\bilGE {13} 
		{
			Внутренняя среда организации и ее основные элементы. Характеристика элементов внутренней среды предприятия сферы околотаможенного сервиса.
		}{
			Понятие, сущность инновации, и ее роль в достижении целей производства услуг. Виды инноваций сферы околотаможенного сервиса.
		}{
			Применение в международной торговле классификации торговых терминов (по базисным условиям ИНКОТЕРМС).
		}
	\lowGE
\end{minipage}

\vfill



\begin{minipage}[t][\miniH]{\miniL}\centering
	\topGE * {080507.65} {Менеджмент организации}[Менеджмент в таможенном деле]
	\bilGE {14} 
		{
			Корпоративная культура и ее влияние на работу предприятия. Использование корпоративной культуры на предприятиях сферы околотаможенного сервиса.
		}{
			Инновационный потенциал современного менеджмента. Задачи менеджмента в области инноваций.
		}{
			Владелец таможенного склада. Обязанность, ответственность владельца склада.
		}
	\lowGE
\end{minipage}

\vfill



\begin{minipage}[t][\miniH]{\miniL}\centering
	\topGE * {080507.65} {Менеджмент организации}[Менеджмент в таможенном деле]
	\bilGE {15} 
		{
			Эффективность изменений в организации: виды и причины сопротивления персонала организационным изменениям.
		}{
			Классификация инноваций и организационно-правовые формы инновационного предпринимательства.
		}{
			Владелец склада временного хранения. Обязанности и ответственность владельца склада временного хранения.
		}
	\lowGE
\end{minipage}





\begin{minipage}[t][\miniH]{\miniL}\centering
	\topGE * {080507.65} {Менеджмент организации}[Менеджмент в таможенном деле]
	\bilGE {16} 
		{
			Внешняя среда фирмы: сущность, основные элементы. Характеристика элементов внешней среды предприятия сферы околотаможенного сервиса.
		}{
			Понятие и сущность риск-менеджмента. Риски и их виды. Показатели и методы количественной оценки рисков.
		}{
			Государственная поддержка инновационной деятельности предприятий: направление, методы. Основные меры по реализации государственной политики в области развития национальной инновационной системы.
		}
	\lowGE
\end{minipage}

\vfill



\begin{minipage}[t][\miniH]{\miniL}\centering
	\topGE * {080507.65} {Менеджмент организации}[Менеджмент в таможенном деле]
	\bilGE {17} 
		{
			Стратегическое планирование деятельности фирмы: сущность, решаемые проблемы, основные требования. Основные компоненты стратегического плана фирмы, их содержание.
		}{
			Виды и характеристика стратегии бизнеса. Использование SWOT-анализа в стратегическом менеджменте
		}{
			Представление документов при таможенном декларировании. Дополнительные документы, представляемые при таможенном декларировании в соответствии с условиями таможенных процедур.
		}
	\lowGE
\end{minipage}

\vfill



\begin{minipage}[t][\miniH]{\miniL}\centering
	\topGE * {080507.65} {Менеджмент организации}[Менеджмент в таможенном деле]
	\bilGE {18} 
		{
			Понятие, сущность и значение текущего планирования. Сравнительная характеристика стратегического и текущего планирования.
		}{
			Таможенное регулирование в Таможенном союзе. Единая таможенная территория Таможенного союза и таможенная граница.
		}{
			Особенности управления рисками, связанными с внешнеэкономической деятельностью.
		}
	\lowGE
\end{minipage}





\begin{minipage}[t][\miniH]{\miniL}\centering
	\topGE * {080507.65} {Менеджмент организации}[Менеджмент в таможенном деле]
	\bilGE {19} 
		{
			Бизнес-план предприятия сферы околотаможенного сервиса: понятие, основные разделы и этапы разработки.
		}{
			Уполномоченный экономический оператор. Условия присвоения статуса уполномоченного экономического оператора.
		}{
			Виды таможенных процедур. Выбор и изменение таможенной процедуры. Помещение под таможенную процедуру.
		}
	\lowGE
\end{minipage}

\vfill



\begin{minipage}[t][\miniH]{\miniL}\centering
	\topGE * {080507.65} {Менеджмент организации}[Менеджмент в таможенном деле]
	\bilGE {20} 
		{
			Специфика труда менеджера. Основные требования к руководителю предприятия сферы околотаможенного сервиса.
		}{
			Перемещение товаров через таможенную границу. Места перемещения товаров через таможенную границу.
		}{
			Тактические и стратегические мероприятия по выходу предприятия из кризиса.
		}
	\lowGE
\end{minipage}

\vfill



\begin{minipage}[t][\miniH]{\miniL}\centering
	\topGE * {080507.65} {Менеджмент организации}[Менеджмент в таможенном деле]
	\bilGE {21} 
		{
			Мотивация трудовой деятельности. Теории мотивации. Специфика мотивации сотрудников предприятий сферы околотаможенного сервиса.
		}{
			Основные критерии составления внешнеэкономического контракта.
		}{
			Выпуск товаров при выявлении административного правонарушения или преступления. Условно выпущенные товары. Отказ в выпуске товаров.
		}
	\lowGE
\end{minipage}





\begin{minipage}[t][\miniH]{\miniL}\centering
	\topGE * {080507.65} {Менеджмент организации}[Менеджмент в таможенном деле]
	\bilGE {22} 
		{
			Методы отбора, приема персонала. Состав и структура персонала.
		}{
			Место и время прибытия товаров на таможенную территорию Таможенного союза.
		}{
			Основные звенья логистических систем.
		}
	\lowGE
\end{minipage}

\vfill



\begin{minipage}[t][\miniH]{\miniL}\centering
	\topGE * {080507.65} {Менеджмент организации}[Менеджмент в таможенном деле]
	\bilGE {23} 
		{
			Основные требования к созданию оптимальных условий труда на предприятиях сферы околотаможенного сервиса.
		}{
			Документы и сведения, представляемые в таможенный орган в зависимости от вида транспорта, на котором осуществляется перевозка товаров.
		}{
			Выпуск товаров при необходимости исследования документов, проб и образцов товаров либо получения заключения эксперта.
		}
	\lowGE
\end{minipage}

\vfill



\begin{minipage}[t][\miniH]{\miniL}\centering
	\topGE * {080507.65} {Менеджмент организации}[Менеджмент в таможенном деле]
	\bilGE {24} 
		{
			Полномочия, их распределение и делегирование на предприятии: этапы, основные требования.
		}{
			Методы и принципы таможенного регулирования ВЭД. Правовые основы таможенного регулирования.
		}{
			Государственное регулирование качества услуг (стандартизация, лицензирование).
		}
	\lowGE
\end{minipage}





\begin{minipage}[t][\miniH]{\miniL}\centering
	\topGE * {080507.65} {Менеджмент организации}[Менеджмент в таможенном деле]
	\bilGE {25} 
		{
			Виды и формы оплаты труда. Принципы стимулирования оплаты труда. Модель стимулирующей оплаты труда.
		}{
			Таможенный перевозчик. Понятие, назначение, условия включения в Реестр таможенных перевозчиков. Отличительные признаки и приоритет перед остальными перевозчиками.
		}{
			Формирование и использование денежных фондов предприятий. Собственный и заемный капитал.
		}
	\lowGE
\end{minipage}

\vfill



\begin{minipage}[t][\miniH]{\miniL}\centering
	\topGE * {080507.65} {Менеджмент организации}[Менеджмент в таможенном деле]
	\bilGE {26} 
		{
			Конфликты в организациях: сущность, природа. Современная типология конфликтов в организациях и причины их возникновения. Методы разрешения конфликтных ситуаций в коллективе.
		}{
			Типы базовых конкурентных стратегий: ценовое лидерство, дифференциация, фокусирование. Характерные черты базовых стратегий. Основные достоинства и опасности базовых стратегий.
		}{
			Основания для выпуска товаров и порядок выпуска товаров. Срок выпуска товаров. Выпуск товаров до подачи таможенной декларации.
		}
	\lowGE
\end{minipage}

\vfill



\begin{minipage}[t][\miniH]{\miniL}\centering
	\topGE * {080507.65} {Менеджмент организации}[Менеджмент в таможенном деле]
	\bilGE {27} 
		{
			Организация как открытая или закрытая система. Сравнительная характеристика систем.
		}{
			Понятие конкуренции и конкурентоспособности предприятия. Факторы, влияющие на конкурентоспособность предприятия сферы околотаможенного сервиса.
		}{
			Лизинг как форма деловой сделки: сущность , виды, достоинства.
		}
	\lowGE
\end{minipage}





\begin{minipage}[t][\miniH]{\miniL}\centering
	\topGE * {080507.65} {Менеджмент организации}[Менеджмент в таможенном деле]
	\bilGE {28} 
		{
			Основные законы организации. Сущность проявления и действия законов организации на современных предприятиях
		}{
			Понятие временного хранения товаров. Места временного хранения. Сроки временного хранения товаров.
		}{
			Нетарифные меры регулирования – основные понятия и определения.
		}
	\lowGE
\end{minipage}

\vfill



\begin{minipage}[t][\miniH]{\miniL}\centering
	\topGE * {080507.65} {Менеджмент организации}[Менеджмент в таможенном деле]
	\bilGE {29} 
		{
			Миссия и цели организации. Принципы формулирования миссии предприятия сферы околотаможенного сервиса. Формирование целей.
		}{
			Понятия и сущность Товарной номенклатуры внешнеэкономической деятельности Таможенного союза (ТН ВЭД ТС) назначение и принципы построения.
		}{
			Качество и конкурентно способность услуги: оценка и пути повышения.
		}
	\lowGE
\end{minipage}

\vfill



\begin{minipage}[t][\miniH]{\miniL}\centering
	\topGE * {080507.65} {Менеджмент организации}[Менеджмент в таможенном деле]
	\bilGE {30} 
		{
			Рынок, понятие и характеристики. Разновидности рынков. Характеристика услуг логистического и околотаможенного сектора
		}{
			Таможенные операции, связанные с помещением товаров на временное хранение. Операции с товарами, находящимися на временном хранении. Возникновение и прекращение обязанности по уплате ввозных таможенных пошлин, налогов и сроки их уплаты при временном хранении.
		}{
			Аутсорсинг, аутстаффинг: понятие, примеры использования.
		}
	\lowGE
\end{minipage}





\begin{minipage}[t][\miniH]{\miniL}\centering
	\topGE * {080507.65} {Менеджмент организации}[Менеджмент в таможенном деле]
	\bilGE {31} 
		{
			Понятие блага, продукта, услуги. Ограниченность ресурсов, безграничность потребностей.
		}{
			Таможенные платежи и виды ставок таможенных пошлин. Применение ставок таможенных пошлин, налогов. Плательщики таможенных пошлин, налогов. Возникновение и прекращение обязанности по уплате таможенных пошлин, налогов. Случаи не уплаты пошлин, налогов.
		}{
			Основные формы годовой отчетности предприятия. Состав и содержание.
		}
	\lowGE
\end{minipage}

\vfill



\begin{minipage}[t][\miniH]{\miniL}\centering
	\topGE * {080507.65} {Менеджмент организации}[Менеджмент в таможенном деле]
	\bilGE {32} 
		{
			Экономическое развитие и экономический рост. Факторы экономического роста
		}{
			Общие положения о таможенном декларировании товаров. Таможенная декларация. Декларация на товары. Транзитная декларация.
		}{
			Средства системы ФОССТИС (формирования спроса и стимулирования сбыта) реклама, паблисити, личная продажа, стимулирование сбыта.
		}
	\lowGE
\end{minipage}

\vfill



\begin{minipage}[t][\miniH]{\miniL}\centering
	\topGE * {080507.65} {Менеджмент организации}[Менеджмент в таможенном деле]
	\bilGE {33} 
		{
			Макроэкономическое равновесие и цикличность рыночной экономики. Запреты и ограничения неэкономического характера.
		}{
			Налог на добавленную стоимость ( НДС). Экономическое содержание и порядок расчета.
		}{
			Формы таможенного контроля. Краткая характеристика форм таможенного контроля. Принципы и назначение.
		}
	\lowGE
\end{minipage}





\begin{minipage}[t][\miniH]{\miniL}\centering
	\topGE * {080507.65} {Менеджмент организации}[Менеджмент в таможенном деле]
	\bilGE {34} 
		{
			Миссия и цели организации. Принципы формулирования миссии предприятия сферы околотаможенного сервиса. Формирование целей.
		}{
			Таможенное регулирование в Таможенном союзе. Единая таможенная территория Таможенного союза и таможенная граница.
		}{
			Маркетинговая деятельность предприятия: цели, задачи, функции. Комплекс маркетинга.
		}
	\lowGE
\end{minipage}

\vfill

	

\end{document}