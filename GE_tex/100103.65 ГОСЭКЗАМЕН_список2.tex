\documentclass[
	11pt,
	a4paper,
	]
	{article}

\usepackage{mysty, xparse, GExam, fancyhdr}

\usepackage[
	top = .7cm,
	right = 1cm,
	bottom = 1.2cm,
	left = 1cm,
	headheight = 14pt
	]{geometry}


\pagestyle{empty}

\begin{document}

\newlength{\pblength}\settowidth{\pblength}{Директор филиала СПбГЭУ в г. Пскове}

\hfill\parbox{\pblength}{
	\textbf{\textsc{Утверждаю}}\medskip

	Директор филиала СПбГЭУ в г. Пскове\medskip

	\makebox[3cm]{\hrulefill} А. М. Алексеева\medskip

	\makebox[1.5cm]{<<\hrulefill>>} \makebox[3cm]{\hrulefill}\ 2014 год
}


\smallGE * {100103.65} {Социально-культурный сервис и туризм}

	

\noindent\bilGE {1} 
	{
		Культура края в сфере СКСиТ.
	}{
		Этапы развития международного туризма.
	}{
		Туристские ресурсы Австралии и Океании.
	}

\bigskip

\noindent\bilGE {2} 
	{
		Актуальные проблемы современного сервиса.
	}{
		Этапы развития туризма в России.
	}{
		Туристские ресурсы Латинской Америки.
	}

\bigskip

\noindent\bilGE {3} 
	{
		Речевая коммуникация в сфере СКСиТ.
	}{
		Социокультурная мотивация путешествий: история и современность.
	}{
		Туристские ресурсы Северной Америки.
	}

\bigskip

\noindent\bilGE {4} 
	{
		Типология конфликтов в сфере СКСиТ и способы их регулирования.
	}{
		Великие географические открытия и великие путешественники.
	}{
		Туристские ресурсы Китая.
	}

\bigskip

\noindent\bilGE {5} 
	{
		Профессиональная этика сферы СКСиТ.
	}{
		Теория туризма: основные понятия.
	}{
		Туристские ресурсы Японии.
	}

\bigskip

\noindent\bilGE {6} 
	{
		Профессиональные требования к организатору в сфере СКСиТ.
	}{
		Туризм как социально-культурное явление.
	}{
		Туристские ресурсы Юго-Восточной Азии.
	}

\bigskip

\noindent\bilGE {7} 
	{
		Документационное обеспечение деятельности предприятий сферы СКСиТ.
	}{
		Нормы и правовое регулирование в сфере туризма.
	}{
		Туристские ресурсы Южной Азии.
	}

\bigskip

\noindent\bilGE {8} 
	{
		Связи с общественностью в сфере СКСиТ.
	}{
		Государственная политика в области туризма.
	}{
		Туристские ресурсы Африки.
	}

\bigskip

\noindent\bilGE {9} 
	{
		Реклама в СКСиТ.
	}{
		Региональные целевые программы развития туризма и отдыха в РФ.
	}{
		Туристские ресурсы Израиля.
	}

\bigskip

\noindent\bilGE {10} 
	{
		Инновации в СКСиТ: источники, проблемы внедрения.
	}{
		Договорные отношения в сфере туризма.
	}{
		Туристские ресурсы Польши, Чехии, Словакии, Венгрии.
	}

\bigskip

\noindent\bilGE {11} 
	{
		Порядок создания и регистрации нового предприятия в сфере СКСиТ.
	}{
		Всемирные туристские организации и их деятельность.
	}{
		Туристские ресурсы Греции и Турции.
	}

\bigskip

\noindent\bilGE {12} 
	{
		Стандартизация и сертификация в сфере СКСиТ.
	}{
		Турист как потребитель туристских услуг.
	}{
		Туристские ресурсы Бенилюкса.
	}

\bigskip

\noindent\bilGE {13} 
	{
		Специфика маркетинга в сфере СКСиТ.
	}{
		Тенденции развития современного международного туризма.
	}{
		Туристские ресурсы Испании.
	}

\bigskip

\noindent\bilGE {14} 
	{
		Специфика менеджмента в сфере СКСиТ.
	}{
		Тенденции развития современного российского туризма.
	}{
		Туристские ресурсы Италии.
	}

\bigskip

\noindent\bilGE {15} 
	{
		Защита прав потребителя в сфере СКСиТ.
	}{
		Проблемы и перспективы развития особых экономических зон туристско-рекреационного типа.
	}{
		Туристские ресурсы Германии.
	}

\bigskip

\noindent\bilGE {16} 
	{
		Тенденции и проблемы развития малых предприятий в сфере СКСиТ.
	}{
		Гостиница как основной элемент туристской инфраструктуры.
	}{
		Туристские ресурсы Франции.
	}

\bigskip

\noindent\bilGE {17} 
	{
		Типология и структура предпринимательской деятельности в сфере СКСиТ.
	}{
		Транспортное обслуживание в туризме.
	}{
		Туристские ресурсы Великобритании.
	}

\bigskip

\noindent\bilGE {18} 
	{
		Особенности экономической деятельности сфере СКСиТ.
	}{
		Виды и специфика экскурсионного обслуживания туристов.
	}{
		Туристские ресурсы Закавказья.
	}

\bigskip

\noindent\bilGE {19} 
	{
		Договорные отношения в сфере СКСиТ.
	}{
		Технология разработки маршрутов и формирования туров.
	}{
		Туристские ресурсы Казахстана и стран Средней Азии.
	}

\bigskip

\noindent\bilGE {20} 
	{
		Правовые основы обеспечения социально-культурного сервиса и туризма (на материале РФ).
	}{
		Анимация в туризме.
	}{
		Туристские ресурсы Украины, Беларуси, Молдовы.
	}

\bigskip

\noindent\bilGE {21} 
	{
		Международно-правовое обеспечение СКСиТ.
	}{
		Иностранцы в России: позитивные и негативные факторы в сфере туризма.
	}{
		Туристские ресурсы стран Балтии.
	}

\bigskip

\noindent\bilGE {22} 
	{
		История сервиса в России (основные этапы).
	}{
		Россияне за рубежом: позитивные и негативные факторы в туризме.
	}{
		Туристские ресурсы Скандинавских стран.
	}

\bigskip

\noindent\bilGE {23} 
	{
		История сервиса за рубежом (основные этапы).
	}{
		Классификация туристских путешествий.
	}{
		Туристские ресурсы Черноморского побережья РФ и Приазовья.
	}

\bigskip

\noindent\bilGE {24} 
	{
		Сервисные технологии.
	}{
		Детский и молодёжный туризм: формы организации, проблемы и перспективы.
	}{
		Туристские ресурсы Северного Кавказа.
	}

\bigskip

\noindent\bilGE {25} 
	{
		Понятие и виды сервисной деятельности.
	}{
		Социальный туризм: общемировые тенденции, пути возрождения социального туризма в России.
	}{
		Туристские ресурсы Сибири и Дальнего Востока.
	}

\bigskip

\noindent\bilGE {26} 
	{
		Формы, методы и ресурсная база социально-культурной деятельности.
	}{
		Экологический туризм: виды, тенденции, специфика.
	}{
		Туристские ресурсы Поволжья и Урала.
	}

\bigskip

\noindent\bilGE {27} 
	{
		Типы и виды социально-культурных институтов.
	}{
		Круизный туризм: виды, тенденции, специфика.
	}{
		Туристские ресурсы Севера европейской части России.
	}

\bigskip

\noindent\bilGE {28} 
	{
		Социально-культурная деятельность: проблемы, тенденции развития.
	}{
		Сельский туризм: виды, тенденции, специфика.
	}{
		Туристские ресурсы Санкт-Петербурга и Ленинградской области.
	}

\bigskip

\noindent\bilGE {29} 
	{
		Потребности человека: основные концепции, динамика.
	}{
		Специализированный туризм: виды, тенденции, специфика.
	}{
		Туристские ресурсы Москвы и Подмосковья.
	}

\bigskip

\noindent\bilGE {30} 
	{
		Сервисология как область научного знания.
	}{
		Новые формы в современном туризме.
	}{
		Туристские ресурсы Центральной России.
	}

\bigskip

\end{document}