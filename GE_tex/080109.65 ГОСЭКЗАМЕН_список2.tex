\documentclass[
	11pt,
	a4paper,
	]
	{article}

\usepackage{mysty, xparse, GExam, fancyhdr}

\usepackage[
	top = .7cm,
	right = 1cm,
	bottom = 1.2cm,
	left = 1cm,
	headheight = 14pt
	]{geometry}


\pagestyle{empty}

\begin{document}

\newlength{\pblength}\settowidth{\pblength}{Директор филиала СПбГЭУ в г. Пскове}

\hfill\parbox{\pblength}{
	\textbf{\textsc{Утверждаю}}\medskip

	Директор филиала СПбГЭУ в г. Пскове\medskip

	\makebox[3cm]{\hrulefill} А. М. Алексеева\medskip

	\makebox[1.5cm]{<<\hrulefill>>} \makebox[3cm]{\hrulefill}\ 2014 год
}


\smallGE * {080109.65} {Бухгалтерский учет, анализ и аудит}

	

\noindent\bilGE {1} 
	{
		Регулирование и организация бухгалтерского учета в Российской Федерации.
	}{
		Содержание, виды, и задачи экономического анализа.
	}{
		Задача №14.
	}

\bigskip

\noindent\bilGE {2} 
	{
		Федеральный закон «О бухгалтерском учете» и его значение.
	}{
		Понятие, виды и задачи факторного анализа.
	}{
		Задача №15.
	}

\bigskip

\noindent\bilGE {3} 
	{
		Характеристика бухгалтерских документов и документооборота.
	}{
		Экономическая сущность и виды резервов.
	}{
		Задача №16.
	}

\bigskip

\noindent\bilGE {4} 
	{
		Бухгалтерский баланс: понятие, его виды, структура.
	}{
		Анализ выпуска и реализации продукции.
	}{
		Задача №17.
	}

\bigskip

\noindent\bilGE {5} 
	{
		Методы исправления ошибок в учете и отчетности .
	}{
		Анализ структуры, движения и состояние основных средств.
	}{
		Задача №18.
	}

\bigskip

\noindent\bilGE {6} 
	{
		Учётная политика организации и её раскрытие, влияние учётной политики на отражение в отчётности активов, обязательств, доходов, расходов и капитала организации.
	}{
		Анализ эффективности использования основных средств.
	}{
		Задача №1.
	}

\bigskip

\noindent\bilGE {7} 
	{
		Международные стандарты учета и финансовой отчетности.
	}{
		Анализ обеспеченности трудовыми ресурсами.
	}{
		Задача №19.
	}

\bigskip

\noindent\bilGE {8} 
	{
		Понятие, классификация и оценка основных средств.
	}{
		Анализ эффективности использования трудовых ресурсов.
	}{
		Задача №2.
	}

\bigskip

\noindent\bilGE {9} 
	{
		Документальное оформление и учет амортизации основных средств.
	}{
		Анализ обеспеченности материальными ресурсами.
	}{
		Задача №3.
	}

\bigskip

\noindent\bilGE {10} 
	{
		Документальное оформление и учет движения нематериальных активов.
	}{
		Анализ эффективности использования материальных ресурсов.
	}{
		Задача №4.
	}

\bigskip

\noindent\bilGE {11} 
	{
		Понятие, классификация и оценка материально-производственных запасов.
	}{
		Анализ себестоимости продукции.
	}{
		Задача №5.
	}

\bigskip

\noindent\bilGE {12} 
	{
		Учет поступления материалов в организацию.
	}{
		Анализ финансовых результатов.
	}{
		Задача №6.
	}

\bigskip

\noindent\bilGE {13} 
	{
		Учет расхода материалов на производство продукции, работ и услуг.
	}{
		Анализ рентабельности.
	}{
		Задача №9.
	}

\bigskip

\noindent\bilGE {14} 
	{
		Порядок ведения расчетов с персоналом организации по оплате труда и отражение их в бухгалтерском учете.
	}{
		Анализ финансовой устойчивости.
	}{
		Задача №8.
	}

\bigskip

\noindent\bilGE {15} 
	{
		Учет удержаний из заработной платы персонала.
	}{
		Анализ ликвидности.
	}{
		Задача №7.
	}

\bigskip

\noindent\bilGE {16} 
	{
		Учет налога на доходы физических лиц.
	}{
		Федеральный закон об аудиторской деятельности и его значение.
	}{
		Задача №10.
	}

\bigskip

\noindent\bilGE {17} 
	{
		Учет кассовых операций .
	}{
		Аудиторское заключение. Назначение и виды.
	}{
		Задача №11.
	}

\bigskip

\noindent\bilGE {18} 
	{
		Учет операций по расчетным счетам.
	}{
		Виды аудита.
	}{
		Задача №12.
	}

\bigskip

\noindent\bilGE {19} 
	{
		Учет расчетов с подотчетными лицами.
	}{
		Характеристика прочих и сопутствующих услуг.
	}{
		Задача №13.
	}

\bigskip

\noindent\bilGE {20} 
	{
		Учет затрат основного производства.
	}{
		Права и обязанности аудиторских организаций (индивидуальных аудиторов).
	}{
		Задача №20.
	}

\bigskip

\noindent\bilGE {21} 
	{
		Учет общехозяйственных расходов и методы их распределения.
	}{
		Права и обязанности аудируемых лиц.
	}{
		Задача №21.
	}

\bigskip

\noindent\bilGE {22} 
	{
		Учет реализации готовой продукции.
	}{
		Регулирование аудиторской деятельности.
	}{
		Задача №22.
	}

\bigskip

\noindent\bilGE {23} 
	{
		Учет расчетов с покупателями и заказчиками.
	}{
		Аттестация аудитора.
	}{
		Задача №23.
	}

\bigskip

\noindent\bilGE {24} 
	{
		Учет расчетов с поставщиками и подрядчиками.
	}{
		Договор на оказание аудиторских услуг, его структура.
	}{
		Задача №24.
	}

\bigskip

\noindent\bilGE {25} 
	{
		Учет расчетов с государственными внебюджетными фондами по страховым взносам.
	}{
		Планирование аудиторской поверки.
	}{
		Задача №25.
	}

\bigskip

\noindent\bilGE {26} 
	{
		Учет финансовых результатов организации.
	}{
		Аудиторский риск.
	}{
		Задача №26.
	}

\bigskip

\noindent\bilGE {27} 
	{
		Учет затрат по экономическим элементам и статьям калькуляции .
	}{
		Аудиторская выборка.
	}{
		Задача №27.
	}

\bigskip

\noindent\bilGE {28} 
	{
		Позаказный метод учета затрат и калькулирования себестоимости продукции.
	}{
		Существенность в аудите, порядок расчета.
	}{
		Задача №28.
	}

\bigskip

\noindent\bilGE {29} 
	{
		Система «стандарт-костинг» и ее сущность .
	}{
		Способы получения аудиторских доказательств.
	}{
		Задача №29.
	}

\bigskip

\noindent\bilGE {30} 
	{
		Система «директ-костинг», ее сущность и назначение.
	}{
		Аудит кассовых операций.
	}{
		Задача №30.
	}

\bigskip

\end{document}