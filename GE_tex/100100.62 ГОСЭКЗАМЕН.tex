\documentclass[
	11pt,
	a4paper,
	]
	{article}

\usepackage{mysty, xparse, GExam, fancyhdr}

\usepackage[
	top = .7cm,
	right = 1cm,
	bottom = 1.2cm,
	left = 1cm,
	headheight = 14pt
	]{geometry}


\pagestyle{fancy}
\renewcommand{\headrulewidth}{0pt}
\renewcommand{\footrulewidth}{0pt}


\lhead{\begin{tikzpicture}[remember picture, overlay]
	\draw[dashed] (current page.north west) ++ (0, -99mm) -- +(210mm, 0);
	\draw[dashed] (current page.north west) ++ (0, -198mm) -- +(210mm, 0);
\end{tikzpicture}}
\cfoot{}



\begin{document}

	

\begin{minipage}[t][\miniH]{\miniL}\centering
	\topGE  {100100.62} {Сервис}
	\bilGE {1} 
		{
			Классификация и основные характеристики автосервисных предприятий.
		}{
			Основные агрегаты и системы легкового автомобиля.
		}{
			Цели и процедура проведения инструментального контроля.
		}
	\lowGE
\end{minipage}

\vfill



\begin{minipage}[t][\miniH]{\miniL}\centering
	\topGE  {100100.62} {Сервис}
	\bilGE {2} 
		{
			Порядок разработки и основные элементы бизнес-плана сервисного центра.
		}{
			Основные технические характеристики автомобиля.
		}{
			Планово-предупредительная система технического обслуживания и ремонта.
		}
	\lowGE
\end{minipage}

\vfill



\begin{minipage}[t][\miniH]{\miniL}\centering
	\topGE  {100100.62} {Сервис}
	\bilGE {3} 
		{
			Организационная структура управления сервисным центром.
		}{
			Классификация и принцип действия двигателей легковых автомобилей.
		}{
			Способы ремонта и восстановления автомобильных деталей.
		}
	\lowGE
\end{minipage}





\begin{minipage}[t][\miniH]{\miniL}\centering
	\topGE  {100100.62} {Сервис}
	\bilGE {4} 
		{
			Системы автоматизированного управления предприятиями автосервиса. Система «Автодилер».
		}{
			Системы питания двигателей легковых автомобилей и используемые виды топлива.
		}{
			Оценка технического состояния автомобиля. Виды и методы.
		}
	\lowGE
\end{minipage}

\vfill



\begin{minipage}[t][\miniH]{\miniL}\centering
	\topGE  {100100.62} {Сервис}
	\bilGE {5} 
		{
			Определение свободной доли рынка. Расчет годового объема работ автосервисных предприятий.
		}{
			Системы смазки двигателя внутреннего сгорания и используемые моторные масла.
		}{
			Техническое обслуживание и ремонт двигателя автомобиля.
		}
	\lowGE
\end{minipage}

\vfill



\begin{minipage}[t][\miniH]{\miniL}\centering
	\topGE  {100100.62} {Сервис}
	\bilGE {6} 
		{
			Анализ рынка автосервисных услуг. Тенденции развития.
		}{
			Система охлаждения двигателя, применяемые эксплуатационные материалы.
		}{
			Техническое обслуживание и ремонт узлов трансмиссии.
		}
	\lowGE
\end{minipage}





\begin{minipage}[t][\miniH]{\miniL}\centering
	\topGE  {100100.62} {Сервис}
	\bilGE {7} 
		{
			Значение и формирование Brend-имиджа сервисных центров.
		}{
			Система зажигания двигателя. Принцип действия.
		}{
			Техническое обслуживание и ремонт подвески автомобиля.
		}
	\lowGE
\end{minipage}

\vfill



\begin{minipage}[t][\miniH]{\miniL}\centering
	\topGE  {100100.62} {Сервис}
	\bilGE {8} 
		{
			Прогнозирование спроса потребителей на автосервисные услуги.
		}{
			Система пуска двигателя. Принцип действия.
		}{
			Техническое обслуживание и ремонт кузова автомобиля.
		}
	\lowGE
\end{minipage}

\vfill



\begin{minipage}[t][\miniH]{\miniL}\centering
	\topGE  {100100.62} {Сервис}
	\bilGE {9} 
		{
			Стратегии привлечения клиентов. Разработка медиа-плана.
		}{
			Гидромуфта, гидротрансформатор, вариатор. Назначение, принцип действия.
		}{
			Техническое обслуживание и ремонт рулевого управления.
		}
	\lowGE
\end{minipage}





\begin{minipage}[t][\miniH]{\miniL}\centering
	\topGE  {100100.62} {Сервис}
	\bilGE {10} 
		{
			Ценовая политика автосервисных предприятий.
		}{
			Дифференциал. Назначение, принцип действия.
		}{
			Техническое обслуживание и ремонт тормозной системы.
		}
	\lowGE
\end{minipage}

\vfill



\begin{minipage}[t][\miniH]{\miniL}\centering
	\topGE  {100100.62} {Сервис}
	\bilGE {11} 
		{
			Кадровая политика автосервисных предприятий.
		}{
			Конструкции и характеристики подвесок автомобиля.
		}{
			Техническое обслуживание и ремонт генератора.
		}
	\lowGE
\end{minipage}

\vfill



\begin{minipage}[t][\miniH]{\miniL}\centering
	\topGE  {100100.62} {Сервис}
	\bilGE {12} 
		{
			Понятия «сервис», «услуга», «сервисная деятельность».
		}{
			Конструкции и характеристики колес автомобиля.
		}{
			Техническое обслуживание и ремонт стартера.
		}
	\lowGE
\end{minipage}





\begin{minipage}[t][\miniH]{\miniL}\centering
	\topGE  {100100.62} {Сервис}
	\bilGE {13} 
		{
			Понятие и характеристики контактной зоны взаимодействия с клиентом.
		}{
			Назначение и типы кузов легковых автомобилей.
		}{
			Техническое обслуживание и ремонт АКБ.
		}
	\lowGE
\end{minipage}

\vfill



\begin{minipage}[t][\miniH]{\miniL}\centering
	\topGE  {100100.62} {Сервис}
	\bilGE {14} 
		{
			Ответственность сервисного центра перед заказчиком. Нормативные документы.
		}{
			Узлы рулевого управления легкового автомобиля.
		}{
			Техническое обслуживание и ремонт системы питания.
		}
	\lowGE
\end{minipage}

\vfill



\begin{minipage}[t][\miniH]{\miniL}\centering
	\topGE  {100100.62} {Сервис}
	\bilGE {15} 
		{
			Обязанности сервис-менеджера предприятия автосервиса.
		}{
			Назначение, типы и конструкции тормозных систем автомобиля, используемые эксплуатационные материалы.
		}{
			Техническое обслуживание и ремонт смазочной системы.
		}
	\lowGE
\end{minipage}





\begin{minipage}[t][\miniH]{\miniL}\centering
	\topGE  {100100.62} {Сервис}
	\bilGE {16} 
		{
			Трудоемкость при техническом обслуживании и ремонте автотранспортных средств.
		}{
			Основные системы и узлы электрооборудования автомобиля.
		}{
			Техническое обслуживание и ремонт системы охлаждения.
		}
	\lowGE
\end{minipage}

\vfill



\begin{minipage}[t][\miniH]{\miniL}\centering
	\topGE  {100100.62} {Сервис}
	\bilGE {17} 
		{
			Сегментация рынка при проектировании сервисных центров.
		}{
			Потребительские и эксплуатационные свойства автомобилей.
		}{
			Техническое обслуживание и ремонт системы впуска воздуха.
		}
	\lowGE
\end{minipage}

\vfill



\begin{minipage}[t][\miniH]{\miniL}\centering
	\topGE  {100100.62} {Сервис}
	\bilGE {18} 
		{
			Поведение и действия сервис-менеджера в зависимости от типологии клиентов.
		}{
			Модельные ряды автомобилей основных производителей.
		}{
			Техническое обслуживание и ремонт системы выпуска выхлопных газов.
		}
	\lowGE
\end{minipage}





\begin{minipage}[t][\miniH]{\miniL}\centering
	\topGE  {100100.62} {Сервис}
	\bilGE {19} 
		{
			Сервисное обеспечение автомобильных дорог.
		}{
			Автоэлектроника. Состояние и перспективы развития.
		}{
			Диагностика автотранспортных средств. Назначение и виды.
		}
	\lowGE
\end{minipage}

\vfill



\begin{minipage}[t][\miniH]{\miniL}\centering
	\topGE  {100100.62} {Сервис}
	\bilGE {20} 
		{
			Современные информационные технологии, применяемые в сервисной деятельности.
		}{
			Тюнинг. Цели и задачи, этапы развития тюнинга. Виды тюнинга.
		}{
			Назначение и структура сервисной карты обслуживания потребителя.
		}
	\lowGE
\end{minipage}

\vfill



\begin{minipage}[t][\miniH]{\miniL}\centering
	\topGE  {100100.62} {Сервис}
	\bilGE {21} 
		{
			Типологические характеристики различных групп потребителей услуг.
		}{
			Функциональный тюнинг и дооборудование автомобилей и мотоциклов.
		}{
			Маршрутная технологическая карта ремонта систем и агрегатов автомобиля.
		}
	\lowGE
\end{minipage}





\begin{minipage}[t][\miniH]{\miniL}\centering
	\topGE  {100100.62} {Сервис}
	\bilGE {22} 
		{
			Общероссийский классификатор услуг населению. (ОКУН)
		}{
			Конструктивный тюнинг автомобилей и мотоциклов.
		}{
			Операционная технологическая карта.
		}
	\lowGE
\end{minipage}

\vfill



\begin{minipage}[t][\miniH]{\miniL}\centering
	\topGE  {100100.62} {Сервис}
	\bilGE {23} 
		{
			Стратегия завоевания рынка, ассортиментная политика сервисных центров.
		}{
			Маршрутная технологическая карта услуги по тюнингу автомобилей и мотоциклов.
		}{
			Обоснование мощности СТО. Расчет годового объёма работ.
		}
	\lowGE
\end{minipage}

\vfill



\begin{minipage}[t][\miniH]{\miniL}\centering
	\topGE  {100100.62} {Сервис}
	\bilGE {24} 
		{
			Ценовые и неценовые факторы спроса и предложения на рынке услуг.
		}{
			Тюнинг двигателя.
		}{
			Методы технического обслуживания и ремонта. Расчет числа постов.
		}
	\lowGE
\end{minipage}





\begin{minipage}[t][\miniH]{\miniL}\centering
	\topGE  {100100.62} {Сервис}
	\bilGE {25} 
		{
			Основные факторы, влияющие на качество предоставляемой услуги.
		}{
			Тюнинг трансмиссии.
		}{
			Определение потребности в технологическом оборудовании.
		}
	\lowGE
\end{minipage}

\vfill



\begin{minipage}[t][\miniH]{\miniL}\centering
	\topGE  {100100.62} {Сервис}
	\bilGE {26} 
		{
			Стадии и процесс коммуникации сервис-менеджера и клиента.
		}{
			Тюнинг тормозной системы автомобилей и мотоциклов.
		}{
			Определение площадей помещений СТО.
		}
	\lowGE
\end{minipage}

\vfill



\begin{minipage}[t][\miniH]{\miniL}\centering
	\topGE  {100100.62} {Сервис}
	\bilGE {27} 
		{
			Классификация и основные характеристики автосервисных предприятий.
		}{
			Тюнинг кузова и силового каркаса автомобилей и мотоциклов.
		}{
			Виды складов. Их оборудование.
		}
	\lowGE
\end{minipage}





\begin{minipage}[t][\miniH]{\miniL}\centering
	\topGE  {100100.62} {Сервис}
	\bilGE {28} 
		{
			Порядок разработки и основные элементы бизнес-плана сервисного центра.
		}{
			Эстетический тюнинг автомобилей и мотоциклов.
		}{
			Организация снабжения, хранения и учета запасных частей и материалов.
		}
	\lowGE
\end{minipage}

\vfill



\begin{minipage}[t][\miniH]{\miniL}\centering
	\topGE  {100100.62} {Сервис}
	\bilGE {29} 
		{
			Организационная структура управления сервисным центром.
		}{
			Технология проектирования услуги по тюнингу автомобилей и мотоциклов.
		}{
			Экологические требования к размещению и проектированию предприятий автомобильного транспорта.
		}
	\lowGE
\end{minipage}

\vfill



\begin{minipage}[t][\miniH]{\miniL}\centering
	\topGE  {100100.62} {Сервис}
	\bilGE {30} 
		{
			Определение свободной доли рынка. Расчет годового объема работ автосервисных предприятий.
		}{
			Системы эффективного управления тормозами: антиблокировочная система (АБС); противобуксовочная система (ПБС); система динамической стабилизации (СДС).
		}{
			Экологические требования при эксплуатации предприятий автомобильного транспорта.
		}
	\lowGE
\end{minipage}



	

\end{document}