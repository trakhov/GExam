\documentclass[
	11pt,
	a4paper,
	]
	{article}

\usepackage{mysty, xparse, GExam, fancyhdr}

\usepackage[
	top = .7cm,
	right = 1cm,
	bottom = 1.2cm,
	left = 1cm,
	headheight = 14pt
	]{geometry}


\pagestyle{fancy}
\renewcommand{\headrulewidth}{0pt}
\renewcommand{\footrulewidth}{0pt}


\lhead{\begin{tikzpicture}[remember picture, overlay]
	\draw[dashed] (current page.north west) ++ (0, -99mm) -- +(210mm, 0);
	\draw[dashed] (current page.north west) ++ (0, -198mm) -- +(210mm, 0);
\end{tikzpicture}}
\cfoot{}



\begin{document}

\smallGE * {080507.65} {Менеджмент организации}[Менеджмент торговых предприятий]

	

\noindent\bilGE {1} 
	{
		Задачи внедрения принципов и методов современного менеджмента в современную хозяйственную практику в РФ.
	}{
		Инновационная деятельность и риск. Инновационные проекты фирмы: сущность, основные этапы формирования и реализации.
	}{
		Коммерческая работа по оптовой продаже товаров.
	}

\bigskip

\noindent\bilGE {2} 
	{
		Значение совершенствования управления для успеха в коммерческой деятельности современных организаций.
	}{
		Процесс реализации инновационного потенциала менеджмента, способы преодоления сопротивления персонала фирмы инновациям.
	}{
		Организация продажи товаров. Правила продажи товаров
	}

\bigskip

\noindent\bilGE {3} 
	{
		Роль менеджмента в повышении эффективности хозяйствования в России. Проблемы развития менеджмента в современной России.
	}{
		Конфликты в организациях: сущность, природа. Современная типология конфликтов в организациях и причины их возникновения. Методы разрешения конфликтных ситуаций в коллективе.
	}{
		Основные цели и задачи современного торгового предприятия в сфере маркетинга.
	}

\bigskip

\noindent\bilGE {4} 
	{
		Организация как основа современного менеджмента. Жизненный цикл организации
	}{
		Понятие блага, продукта, услуги. Ограниченность ресурсов, безграничность потребностей.
	}{
		Поведение потребителя. Процесс принятия решения о покупке в потребительской сфере.
	}

\bigskip

\noindent\bilGE {5} 
	{
		Менеджмент как наука: сущность, структура и содержание.
	}{
		Экономическое развитие и экономический рост. Факторы экономического роста
	}{
		Методы позиционирования торгового предприятия. Оценка конкурентной позиции.
	}

\bigskip

\noindent\bilGE {6} 
	{
		Цели в системе современного менеджмента, основные требования, предъявляемые к ним.
	}{
		Макроэкономическое равновесие и цикличность рыночной экономики. Запреты и ограничения неэкономического характера.
	}{
		Реклама: понятие, цели, задачи, виды и их характеристики. Планирование и организация рекламной деятельности на торговом предприятии.
	}

\bigskip

\noindent\bilGE {7} 
	{
		Основные виды разделения управленческого труда: сущность, содержание
	}{
		Спрос и предложения, факторы их определяющие. Закон спроса и предложения. Эластичность спроса и предложения.
	}{
		Методика выбора средств и носителей рекламы на торговом предприятии. Оценка коммуникационной и экономической эффективности рекламы.
	}

\bigskip

\noindent\bilGE {8} 
	{
		Система функций современного менеджмента. Сущность и основные виды. Организация реализации принятых решений как одна из основных функций менеджмента.
	}{
		Монополия. Максимизация прибыли монополистами. Естественная монополия.
	}{
		Комплексное исследование товарного рынка: цели, методы, этапы.
	}

\bigskip

\noindent\bilGE {9} 
	{
		Функция мотивации персонала в выполнении принятых решений. Сущность процессуальных теорий мотивации.
	}{
		Олигополия. Производство, ценообразование и эффективность в олигополии.
	}{
		Принципы сегментирования потребителей торгового предприятия. Цели сегментирования, этапы.
	}

\bigskip

\noindent\bilGE {10} 
	{
		Сущность и содержание функции контроля хода выполнения принятых решений.
	}{
		Монополистическая конкуренция. Определение цены и объема производства монополии.
	}{
		Финансовая работа и финансовое планирование в системе управления торговым предприятием
	}

\bigskip

\noindent\bilGE {11} 
	{
		Сущность и задачи процесса принятия решений. Критерии принятия решений в условиях неопределенности.
	}{
		Миссия и цели организации. Принципы формулирования миссии торгового предприятия
	}{
		Персональное развитие в организации. Методы развития персонала в сфере торговли
	}

\bigskip

\noindent\bilGE {12} 
	{
		Современные школы менеджмента: общая характеристика.
	}{
		Формирование целей организации. Функционирования организации. Организационная и информационная структуры.
	}{
		Организация и средства информационных технологий обеспечения управленческой деятельности в сфере торговли
	}

\bigskip

\noindent\bilGE {13} 
	{
		Особенности японской модели менеджмента: общая характеристика. Характер принятия управленческих решений и ответственности на японских фирмах.
	}{
		Государственное регулирование бизнеса. Задачи государственного регулирования предпринимательской деятельности в сфере торговли.
	}{
		Антикризисное управление: основные стратегии. Меры государственного антикризисного регулирования.
	}

\bigskip

\noindent\bilGE {14} 
	{
		Характерные черты американской практики менеджмента: общая характеристика. Сравнительная характеристика американской и японской моделей менеджмента.
	}{
		Типы базовых конкурентных стратегий: ценовое лидерство, дифференциация, фокусирование. Характерные черты базовых стратегий. Основные достоинства и опасности базовых стратегий.
	}{
		Формы продажи: магазинная и внемагазинная. Типы продаж: личная и неличная продажа.
	}

\bigskip

\noindent\bilGE {15} 
	{
		Основные цели современного предприятия сферы торговли. Роль торговых предприятий в экономике страны
	}{
		Понятие конкуренции и конкурентоспособности товара. Факторы, определяющие конкурентоспособность товара
	}{
		Методы (традиционный, через прилавок, самообслуживание и др.) и стили продажи (воспринимающая и агрессивная продажа).
	}

\bigskip

\noindent\bilGE {16} 
	{
		Организация как открытая или закрытая система. Сравнительная характеристика систем.
	}{
		Понятие конкуренции и конкурентоспособности предприятия. Факторы, влияющие на конкурентоспособность торгового предприятия.
	}{
		Организационное построение служб продаж: подходы и принципы (товарный, функциональный, территориальный, по типу клиентуры, и др.), их преимущества, недостатки и критерии выбора.
	}

\bigskip

\noindent\bilGE {17} 
	{
		Основные законы организации. Сущность проявления и действия законов организации на современных предприятиях
	}{
		Содержание и этапы разработки ценовой стратегии. Методы работы с клиентом в форматах ценового стимулирования
	}{
		Общая характеристика основных этапов процесса продажи товаров: перечень этапов, их цели, задачи и характеристика.
	}

\bigskip

\noindent\bilGE {18} 
	{
		Организационная структура, основные классификации. Структуры, ориентированные на нововведения. Основные принципы построения современных организационных структур.
	}{
		Деловые стратегии: портфельные стратегии, стратегии роста, стратегии вертикальной интеграции, конкурентные стратегии. Характеристики и условия применения
	}{
		Основные формы продаж (сбыта) – прямой и косвенный: понятие, преимущества и недостатки.
	}

\bigskip

\noindent\bilGE {19} 
	{
		Внутренняя среда организации и ее основные элементы. Характеристика элементов внутренней среды торгового предприятия.
	}{
		Разработка услуги и ее реализация в торговом предприятии.
	}{
		Основные характеристики каналов продаж (сбыта): длина, ширина, открытость, зона продажи, участники каналов.
	}

\bigskip

\noindent\bilGE {20} 
	{
		Внешняя среда фирмы: сущность, основные элементы. Характеристика элементов внешней среды торгового предприятия.
	}{
		Ассортиментная политика торгового предприятия. Факторы, оказывающие влияние на формирование ассортиментной политики. Мероприятия по её оптимизации.
	}{
		Сущность, коммуникационные особенности и виды личных продаж. Планирование и организации личных продаж в компании
	}

\bigskip

\noindent\bilGE {21} 
	{
		Характеристика и функции корпоративной культуры. Методы формирования и поддержания корпоративной культуры.
	}{
		Характеристика торговой услуги. Сервисное обслуживание как часть товарной политики торгового предприятия. Виды сервиса и эффективность их применения.
	}{
		Роль и задачи личной продажи в системе продвижения товаров. Характеристика этапов процесса личной продажи
	}

\bigskip

\noindent\bilGE {22} 
	{
		Стратегическое планирование деятельности фирмы: сущность, решаемые проблемы, основные требования. Основные компоненты стратегического плана фирмы, их содержание.
	}{
		Фирменный стиль торгового предприятия. Этапы разработки фирменного стиля.
	}{
		Понятие и параметры качества обслуживания. Направления повышения качества обслуживания в торговом бизнесе.
	}

\bigskip

\noindent\bilGE {23} 
	{
		Понятие, сущность и значение текущего планирования. Сравнительная характеристика стратегического и текущего планирования.
	}{
		Прибыль торгового предприятия. Условия максимизации прибыли торгового предприятия.
	}{
		Создание фирменного стиля торгового предприятия. Основные элементы фирменного стиля.
	}

\bigskip

\noindent\bilGE {24} 
	{
		Бизнес-план фирмы – сущность, структура, общая характеристика основных разделов.
	}{
		Доходы торговых предприятий: понятие, виды, значение и состав. Пути повышения доходов.
	}{
		Роль и особенности реализации функций мерчендайзинга в различных субъектах коммерческой деятельности (производственные предприятия, предприятия оптовой торговли, предприятия розничной торговли).
	}

\bigskip

\noindent\bilGE {25} 
	{
		Виды и формы оплаты труда. Принципы стимулирования оплаты труда. Модель стимулирующей оплаты труда.
	}{
		Издержки торговых предприятий: их сущность, классификация и основные направления оптимизации издержек.
	}{
		Планирование отдельных этапов процесса продажи товаров: поиск покупателей и установление контактов с ними; выявление нужд и потребностей покупателей, их ожиданий от покупки; представление (презентация) товаров: модели, правила и специфика их применения; ответы на вопросы и возражения; завершение процесса продажи: основные приемы; мероприятия, следующие за продажей.
	}

\bigskip

\noindent\bilGE {26} 
	{
		Роль менеджера в управлении предприятием.
	}{
		Рентабельность торговых предприятий: понятие, значение и методы расчета. Факторы, влияющие на рентабельность торговых предприятий.
	}{
		Управление продажами и продвижение продукции на основе трейд-маркетинга. Инструмента трейд-маркетинга.
	}

\bigskip

\noindent\bilGE {27} 
	{
		Кадровая политика на предприятии. Особенности и формы управления персоналом на торговом предприятии.
	}{
		Себестоимость продаж: понятие, методы определения, факторы, влияющие на себестоимость. Соотношение себестоимости и прибыли, их взаимосвязь.
	}{
		Виды целей стимулирования продаж (стратегические, специфические, разовые). Цели стимулирование продаж по отношению к объектам стимулирования продаж.
	}

\bigskip

\noindent\bilGE {28} 
	{
		Организация контроля за результатами работы фирмы. Методы контроля на торговом предприятии.
	}{
		Способы выявления влияния факторов на изменения результативного показателя в экономическом анализе деятельности предприятия в сфере торгового сервиса
	}{
		Поиск, отбор и обучение торгового персонала: цели, методы (приёмы), проблемы.
	}

\bigskip

\noindent\bilGE {29} 
	{
		Виды, цели и компоненты контроллинга. Специфика контроллинга на торговой фирме.
	}{
		Товарная марка и ее роль в деятельности торгового предприятия.
	}{
		Оценка собственного торгового персонала: количественные и качественные показатели.
	}

\bigskip

\noindent\bilGE {30} 
	{
		Понятие и сущность риск-менеджмента. Риски и их виды. Показатели и методы количественной оценки рисков.
	}{
		Организация расчетов в коммерческой деятельности. Эффективность коммерческой деятельности торговых предприятий.
	}{
		Торговые сети: понятие, функции, классификация (по стационарному признаку и товарно-ассортиментному профилю). Влияния торговой сети на региональную экономику.
	}

\bigskip

\noindent\bilGE {31} 
	{
		Инновационный потенциал современного менеджмента. Задачи менеджмента в области инноваций.
	}{
		Характеристика торгово-технологического процесса на торговом предприятии.
	}{
		Составление и оформление организационно-распорядительных документов на предприятии торговли: штатное расписание, положение о структурном подразделении, должностная инструкции.
	}

\bigskip

\noindent\bilGE {32} 
	{
		Управление инновациями на современном предприятии.
	}{
		Организация коммерческой деятельности малых предприятий в розничной торговле
	}{
		Документирование движения персонала на предприятии торговли (прием, увольнение, перевод, отпуск, командировки)
	}

\bigskip

\noindent\bilGE {33} 
	{
		Инновации в торговле. Критерии отбора нововведений в торговых предприятиях.
	}{
		Факторинговые операции торговых предприятий. Технология торговой деятельности.
	}{
		Требования к оформлению документов на предприятии торговли. Оперативное хранение. Передача дел в архив.
	}

\bigskip

\noindent\bilGE {34} 
	{
		Функция мотивации персонала в выполнении принятых решений. Сущность процессуальных теорий мотивации.
	}{
		Рентабельность торговых предприятий: понятие, значение и методы расчета. Факторы, влияющие на рентабельность торговых предприятий.
	}{
		Коммуникации и переговоры в торговом бизнесе. Функции делового общения.
	}

\bigskip

\end{document}