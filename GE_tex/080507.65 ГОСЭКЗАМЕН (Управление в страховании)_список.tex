\documentclass[
	11pt,
	a4paper,
	]
	{article}

\usepackage{mysty, xparse, GExam, fancyhdr}

\usepackage[
	top = .7cm,
	right = 1cm,
	bottom = 1.2cm,
	left = 1cm,
	headheight = 14pt
	]{geometry}


\pagestyle{fancy}
\renewcommand{\headrulewidth}{0pt}
\renewcommand{\footrulewidth}{0pt}


\lhead{\begin{tikzpicture}[remember picture, overlay]
	\draw[dashed] (current page.north west) ++ (0, -99mm) -- +(210mm, 0);
	\draw[dashed] (current page.north west) ++ (0, -198mm) -- +(210mm, 0);
\end{tikzpicture}}
\cfoot{}



\begin{document}

\smallGE * {080507.65} {Менеджмент организации}[Управление в страховании]

	

\noindent\bilGE {1} 
	{
		Задачи внедрения принципов и методов современного менеджмента в современную хозяйственную практику в РФ.
	}{
		Виды и формы оплаты труда. Принципы стимулирования оплаты труда. Модель стимулирующей оплаты труда.
	}{
		Виды налоговых выплат юридическими и физическими лицами при применении упрощенной системы налогообложения, учета и отчетности.
	}

\bigskip

\noindent\bilGE {2} 
	{
		Значение совершенствования управления для успеха в коммерческой деятельности современных организаций.
	}{
		Конфликты в организациях: сущность, природа. Современная типология конфликтов в организациях и причины их возникновения. Методы разрешения конфликтных ситуаций в коллективе. Характеристика конфликтов в страховом бизнесе.
	}{
		НДС: экономическое содержание и порядок расчета.
	}

\bigskip

\noindent\bilGE {3} 
	{
		Роль менеджмента в повышении эффективности хозяйствования в России. Проблемы развития менеджмента в современной России.
	}{
		Пути повышения эффективности использования персонала. Аттестация работников сферы страхового бизнеса.
	}{
		Налог на прибыль: определение, значение для предприятия и процедура взимания. Порядок расчета налога на
	}

\bigskip

\noindent\bilGE {4} 
	{
		Организация как основа современного менеджмента. Жизненный цикл организации
	}{
		Миссия и цели организации. Принципы формулирования миссии предприятия сферы страхового бизнеса. Формирование целей.
	}{
		Состав платежей и налогов, используемых в регулировании ВЭД, их целевое назначение.
	}

\bigskip

\noindent\bilGE {5} 
	{
		Менеджмент как наука: сущность, структура и содержание.
	}{
		Рынок, понятие и характеристики. Разновидности рынков. Общая характеристика рынка страхования России.
	}{
		Способы выявления влияния факторов на изменения результативного показателя в экономическом анализе деятельности предприятия в сфере страхового сервиса
	}

\bigskip

\noindent\bilGE {6} 
	{
		Цели в системе современного менеджмента, основные требования, предъявляемые к ним Основные виды разделения управленческого труда: сущность, содержание
	}{
		Понятие блага, продукта, услуги. Ограниченность ресурсов, безграничность потребностей.
	}{
		Понятие производительности труда. Основные показатели оценки уровня и динамики производительности труда.
	}

\bigskip

\noindent\bilGE {7} 
	{
		Система функций современного менеджмента. Сущность и основные виды. Организация реализации принятых решений как одна из основных функций менеджмента.
	}{
		Экономическое развитие и экономический рост. Факторы экономического роста
	}{
		Составление и оформление организационно-распорядительных документов в сфере страхования: штатное расписание, положение о структурном подразделении, должностная инструкции.
	}

\bigskip

\noindent\bilGE {8} 
	{
		Сущность и задачи процесса принятия решений. Критерии принятия решений в условиях неопределенности.
	}{
		Макроэкономическое равновесие и цикличность рыночной экономики. Запреты и ограничения неэкономического
	}{
		Документирование движения персонала в сфере страхования (прием, увольнение, перевод, отпуск, командировки)
	}

\bigskip

\noindent\bilGE {9} 
	{
		Управленческий контроль как одна из функций управления. Формы контроля реализации решения.
	}{
		Спрос и предложения, факторы их определяющие. Закон спроса и предложения. Эластичность спроса и
	}{
		Основные требования к кадрам в страховом бизнесе. Организация продвижения по службе в страховом бизнесе
	}

\bigskip

\noindent\bilGE {10} 
	{
		Функция мотивации персонала в выполнении принятых решений. Сущность процессуальных теорий мотивации.
	}{
		Стили управления. Достоинства и недостатки различных стилей управления.
	}{
		Корпоративные нормы в страховом бизнесе. Методы подбора кадров в страховом бизнесе
	}

\bigskip

\noindent\bilGE {11} 
	{
		Сущность и содержание функции контроля хода выполнения принятых решений.
	}{
		Организационно-правовые формы хозяйствования и их характеристика. Методы изменения форм собственности и
	}{
		Анализ структуры страхового портфеля страховой организации
	}

\bigskip

\noindent\bilGE {12} 
	{
		Современные школы менеджмента: общая характеристика.
	}{
		Государственное регулирование бизнеса. Задачи государственного регулирования предпринимательской деятельности в сфере сферы страхового бизнеса.
	}{
		Особенности страхования имущества физических и юридических лиц.
	}

\bigskip

\noindent\bilGE {13} 
	{
		Особенности японской модели менеджмента: общая характеристика. Характер принятия управленческих решений и ответственности на японских фирмах.
	}{
		Доходы предприятий сферы страхового бизнеса: понятие, виды, значение и состав. Пути повышения доходов.
	}{
		Методы определения ущерба и страхового возмещения.
	}

\bigskip

\noindent\bilGE {14} 
	{
		Характерные черты американской практики менеджмента: общая характеристика. Сравнительная характеристика американской и японской моделей менеджмента.
	}{
		Затраты предприятий сферы страхового бизнеса: их сущность, классификация и основные направления
	}{
		Страхование грузов на территории Российской Федерации.
	}

\bigskip

\noindent\bilGE {15} 
	{
		Основные цели современного предприятия сферы страхового бизнеса. Роль в развитии экономики страны
	}{
		Формы и методы налогообложения предприятий сферы страхового бизнеса.
	}{
		Особенности управления рисками, связанными с внешнеэкономической деятельностью.
	}

\bigskip

\noindent\bilGE {16} 
	{
		Организация как открытая или закрытая система. Сравнительная характеристика систем.
	}{
		Прибыль как экономическая категория, ее сущность и значение. Значение прибыли для развития предприятия страхового бизнеса.
	}{
		Особенности страхования валютных рисков во внешнеэкономической деятельности.
	}

\bigskip

\noindent\bilGE {17} 
	{
		Основные законы организации. Сущность проявления и действия законов организации на современных
	}{
		Рентабельность предприятий сферы страхового бизнеса: понятие, значение и методы расчета. Факторы, влияющие на рентабельность предприятий сферы страхового бизнеса.
	}{
		Особенности страхования валютных рисков во внешнеэкономической деятельности.
	}

\bigskip

\noindent\bilGE {18} 
	{
		Организационная структура, основные классификации. Структуры, ориентированные на нововведения. Основные принципы построения современных организационных структур.
	}{
		Сущность услуг и их место в экономической системе. Факторы, определяющие новую роль услуг. Характеристика услуги страхования.
	}{
		Имущественные виды страхования во внешнеэкономических отношениях.
	}

\bigskip

\noindent\bilGE {19} 
	{
		Внутренняя среда организации и ее основные элементы. Характеристика элементов внутренней среды предприятия сферы страхового бизнеса.
	}{
		Конкуренция: ее сущность, виды и роль в механизме функционирования рынка. Особенности формирования и использования конкурентных преимуществ в сфере услуг страхования.
	}{
		Страхование как необходимый элемент туристической деятельности.
	}

\bigskip

\noindent\bilGE {20} 
	{
		Корпоративная культура и ее влияние на работу предприятия. Использование корпоративной культуры на современном предприятии сферы страхового бизнеса
	}{
		Элементы инфраструктуры бизнеса. Функции и задачи инфраструктуры малых предприятий.
	}{
		Виды страхования в туризме. Субъекты оказания страховых услуг туристам.
	}

\bigskip

\noindent\bilGE {21} 
	{
		Внешняя среда фирмы: сущность, основные элементы. Характеристика элементов внешней среды современного предприятия сферы страхового бизнеса
	}{
		Малое предпринимательство в России. Экономические, социальные и правовые условия предпринимательства. Современные формы организации малого бизнеса.
	}{
		Особенности страхования граждан, выезжающих за рубеж. Формы организации страхования граждан, выезжающих за рубеж.
	}

\bigskip

\noindent\bilGE {22} 
	{
		Цели и задачи стратегического менеджмента. Тактические и стратегические мероприятия по выходу предприятия
	}{
		Государственное регулирование сферы страхового бизнеса: цели, основные инструменты.
	}{
		Договор страхования граждан, выезжающих за рубеж. Имущественное страхование туристов.
	}

\bigskip

\noindent\bilGE {23} 
	{
		Типы базовых конкурентных стратегий: ценовое лидерство, дифференциация, фокусирование. Характерные черты базовых стратегий. Основные достоинства и опасности базовых стратегий.
	}{
		Понятие, сущность инновации, и ее роль в достижении целей производства услуг. Виды инноваций.
	}{
		Источники финансирования добровольного медицинского страхования. Варианты оплаты страхового случая. Страхование неотложной помощи.
	}

\bigskip

\noindent\bilGE {24} 
	{
		Особенности стратегического и тактического управления в страховом бизнесе
	}{
		Сущность инновационного предпринимательства. Венчурный капитал в инновационном предпринимательстве.
	}{
		Особенности страхования международной торговли.
	}

\bigskip

\noindent\bilGE {25} 
	{
		Стратегическое планирование деятельности фирмы: сущность, решаемые проблемы, основные требования. Основные компоненты стратегического плана фирмы, их содержание.
	}{
		Оценка эффективных инновационных проектов. Инновационная политика.
	}{
		Страховая премия, форма и порядок ее уплаты. Франшиза.
	}

\bigskip

\noindent\bilGE {26} 
	{
		Бизнес-план предприятия сферы страхового бизнеса: понятие, основные разделы.
	}{
		Основные компоненты инновационного процесса. Факторы, способствующие созданию инноваций в сфере услуг.
	}{
		Организация имущественного страхования.
	}

\bigskip

\noindent\bilGE {27} 
	{
		Структура бизнес-плана. Показатели эффективности бизнес-плана: чистый приведенный доход, индекс прибыльности, внутренняя норма доходности, период окупаемости.
	}{
		Понятие и сущность риск-менеджмента. Риски и их виды. Показатели и методы количественной оценки рисков.
	}{
		Страхование профессиональной ответственности.
	}

\bigskip

\noindent\bilGE {28} 
	{
		Характеристика рисков в бизнес-планировании деятельности фирмы в сфере сервиса.
	}{
		Маркетинговые стратегии для организации сферы услуг. Жизненный цикл услуги страхования. Особенности ценообразования на различных стадиях.
	}{
		Страхование гражданской ответственности владельцев автотранспортных средств
	}

\bigskip

\noindent\bilGE {29} 
	{
		Специфические черты страховой организации как объекта управления.
	}{
		Методы ценообразования. Структура цены на услуги страхования.
	}{
		Страхование гражданской ответственности авиаперевозчиков перед третьими лицами.
	}

\bigskip

\noindent\bilGE {30} 
	{
		Специфика труда менеджера. Основные требования к руководителю предприятия сферы страхового бизнеса.
	}{
		Виды и источники инвестиций в сфере страхования.
	}{
		Права и обязанности граждан в системе обязательного медицинского страхования
	}

\bigskip

\noindent\bilGE {31} 
	{
		Специфика мотивации сотрудников предприятий сферы страхового бизнеса.
	}{
		Основные и оборотные фонды страховых организаций: состав, структура.
	}{
		Страхование финансовых рисков внешнеэкономической деятельности
	}

\bigskip

\noindent\bilGE {32} 
	{
		Основные требования к созданию оптимальных условий труда на предприятиях сферы страхового бизнеса.
	}{
		Отчетность в страховом бизнесе. Порядок ее составления и предоставления.
	}{
		Особенности страхования кредитов. Страхование международных кредитных отношений. Формы страхования кредитных рисков
	}

\bigskip

\noindent\bilGE {33} 
	{
		Полномочия, их распределение и делегирование на предприятии страхового бизнеса: этапы, основные требования.
	}{
		Ответственность налогоплательщика за нарушение налогового законодательства.
	}{
		Страхование рисков внешнеэкономической деятельности отечественными страховыми компаниями.
	}

\bigskip

\noindent\bilGE {34} 
	{
		Система функций современного менеджмента. Сущность и основные виды. Организация реализации принятых решений как одна из основных функций менеджмента.
	}{
		Методы ценообразования. Структура цены на услуги страхования.
	}{
		Коммуникации и переговоры в страховом бизнесе. Функции делового общения.
	}

\bigskip

\end{document}