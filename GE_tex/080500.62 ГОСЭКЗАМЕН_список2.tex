\documentclass[
	11pt,
	a4paper,
	]
	{article}

\usepackage{mysty, xparse, GExam, fancyhdr}

\usepackage[
	top = .7cm,
	right = 1cm,
	bottom = 1.2cm,
	left = 1cm,
	headheight = 14pt
	]{geometry}


\pagestyle{empty}

\begin{document}

\newlength{\pblength}\settowidth{\pblength}{Директор филиала СПбГЭУ в г. Пскове}

\hfill\parbox{\pblength}{
	\textbf{\textsc{Утверждаю}}\medskip

	Директор филиала СПбГЭУ в г. Пскове\medskip

	\makebox[3cm]{\hrulefill} А. М. Алексеева\medskip

	\makebox[1.5cm]{<<\hrulefill>>} \makebox[3cm]{\hrulefill}\ 2014 год
}


\smallGE  {080500.62} {Менеджмент}

	

\noindent\bilGE {1} 
	{
		Сущность рынка и его виды. Конъюнктура рынка.
	}{
		Порядок расчёта себестоимости единицы продукции, работ и услуг на предприятии. Пути снижения себестоимости продукции и услуг.
	}{
		Анализ себестоимости (задачи, объекты, этапы, источники информации, основные показатели, используемые методики).
	}

\bigskip

\noindent\bilGE {2} 
	{
		Рыночный механизм как взаимосвязь и взаимодействие элементов рынка: спроса, предложения и цены.
	}{
		Общие и частные показатели эффективности хозяйственной деятельности предприятия, пути ее повышения.
	}{
		Регионализация, понятие о регионе, виды регионов.
	}

\bigskip

\noindent\bilGE {3} 
	{
		Функции спроса и предложения. Точка равновесия. Эластичность спроса и предложения.
	}{
		Структура цены. Методы ценообразования. Методы государственного регулирования цен на продукцию и услуги.
	}{
		Государственная региональная политика и ее аспекты.
	}

\bigskip

\noindent\bilGE {4} 
	{
		Методы государственного регулирования экономики.
	}{
		Жизненный цикл товара. Особенности ценообразования на различных стадиях.
	}{
		Особенности развития Псковской области: стратегии и проекты.
	}

\bigskip

\noindent\bilGE {5} 
	{
		Менеджмент: сущность, содержание, цель и задачи. Функции управления.
	}{
		Сущность и роль планирования на предприятии, виды планов.
	}{
		Классификация отраслей хозяйства по назначению продукции и по характеру предметов труда.
	}

\bigskip

\noindent\bilGE {6} 
	{
		Вклад различных направлений научных школ в науку и практику управления.
	}{
		Роль и структура бизнес-плана.
	}{
		Особенности отраслевой и территориальной структуры народного хозяйства Северо-Западного федерального округа.
	}

\bigskip

\noindent\bilGE {7} 
	{
		Организационно-правовые формы хозяйствования в РФ и их характеристика. Методы изменения форм собственности и сущность.
	}{
		Инвестиции, их классификация и источники формирования.
	}{
		Понятие о праве и его определение. Специфические признаки права.
	}

\bigskip

\noindent\bilGE {8} 
	{
		Распределение прав и полномочий в менеджменте.
	}{
		Методы оценки эффективности инвестиций.
	}{
		Понятие трудовых правоотношений. Трудовой договор. Трудовой распорядок.
	}

\bigskip

\noindent\bilGE {9} 
	{
		Роль руководителя в организации (навыки, стили руководителя и т.д.).
	}{
		Принципы обеспечения качества и методы контроля качества. Частные и общие факторы качества.
	}{
		Нормативные акты в области защиты информации и государственной тайны.
	}

\bigskip

\noindent\bilGE {10} 
	{
		Классическая теория организации и основные принципы организации управления.
	}{
		Качество и конкурентоспособность продукции и услуг: оценка и пути повышения.
	}{
		Роль и место управления персоналом в областях менеджмента, маркетинга, психологии, социологии, экономики и экономики труда.
	}

\bigskip

\noindent\bilGE {11} 
	{
		Классификация структур управления предприятием.
	}{
		Формы организации и оплаты труда на предприятии.
	}{
		Методы управления персоналом: сущность и практическое применение.
	}

\bigskip

\noindent\bilGE {12} 
	{
		Характеристика различных типов организационных структур.
	}{
		Формы и системы заработной платы. Структура доходов работника предприятия.
	}{
		Формы и цель планирование деловой карьеры сотрудников.
	}

\bigskip

\noindent\bilGE {13} 
	{
		Типология организационных культур и динамика развития организации.
	}{
		НДС: экономическое содержание, объект и субъект налогообложения, порядок расчета
	}{
		Конфликты в коллективе: опасности и возможности. Методы диагностики и оценки конфликтов.
	}

\bigskip

\noindent\bilGE {14} 
	{
		Сущность и задачи процесса принятия решений.
	}{
		Налог на прибыль: экономическое содержание, объект и субъект налогообложения, порядок расчета
	}{
		Условия труда персонала: принципы и тенденции.
	}

\bigskip

\noindent\bilGE {15} 
	{
		Методы принятия решений в менеджменте.
	}{
		ЕСН: экономическое содержание, объект и субъект налогообложения, порядок расчета
	}{
		Функции управления персоналом, методы оценки их реализации.
	}

\bigskip

\noindent\bilGE {16} 
	{
		Управленческий контроль как одна из функций управления. Формы контроля реализации решения.
	}{
		Единый налог по упрощенной системе налогообложения: экономическое содержание, объект и субъект налогообложения, порядок расчета
	}{
		Государственное регулирование и государственная поддержка местного самоуправления.
	}

\bigskip

\noindent\bilGE {17} 
	{
		Основные фонды предприятия, их состав, виды стоимостной оценки, источники формирования и пополнения. Оценка их состояния и использования.
	}{
		Единый налог на вмененный доход: экономическое содержание, объект и субъект налогообложения, порядок расчета
	}{
		Полномочия органов государственной власти в области муниципального управления.
	}

\bigskip

\noindent\bilGE {18} 
	{
		Оборотные средства предприятия, их состав, источники формирования и пополнения. Оценка их состояния и использования.
	}{
		История логистики. Понятие и содержание логистики.
	}{
		Проблемы территориального аспекта организации местного самоуправления в Российской Федерации.
	}

\bigskip

\noindent\bilGE {19} 
	{
		Показатели, характеризующие кадровый потенциал предприятия сервиса. Оценка состояния и использования трудовых ресурсов предприятия.
	}{
		Логистика как фактор повышения конкурентоспособности.
	}{
		Взаимодействие местной власти с гражданами и общественными объединениями граждан.
	}

\bigskip

\noindent\bilGE {20} 
	{
		Понятие производительности труда. Основные показатели оценки уровня и динамики производительности труда, пути повышения производительности труда.
	}{
		Задачи и функции производственной логистики.
	}{
		Производственная структура предприятия, факторы ее определяющие.
	}

\bigskip

\noindent\bilGE {21} 
	{
		Порядок формирования чистой прибыли предприятия. Содержание отчета о прибылях и убытках.
	}{
		Способы снижения уровня логистических затрат.
	}{
		Цели, задачи, принципы построения системы управления качеством на предприятии. Рекомендации международных ИСО 9000 по обеспечению качества.
	}

\bigskip

\noindent\bilGE {22} 
	{
		Доходы и расходы предприятия. Их классификация.
	}{
		Виды анализа, его роль в управлении производством.
	}{
		Цели и задачи стратегического менеджмента.
	}

\bigskip

\noindent\bilGE {23} 
	{
		Постоянные и переменные затраты. Эффект масштаба.
	}{
		Анализ использования трудовых ресурсов (задачи, объекты, этапы, источники информации, основные показатели, используемые методики).
	}{
		Формирование миссии и стратегических целей организации. Классификация и типы стратегий.
	}

\bigskip

\noindent\bilGE {24} 
	{
		Точка безубыточности: экономическая сущность и способ расчета.
	}{
		Анализ использования основных фондов, (задачи, объекты, этапы, источники информации, основные показатели, используемые методики).
	}{
		Использование SWOT - анализа в стратегическом менеджменте.
	}

\bigskip

\noindent\bilGE {25} 
	{
		Роль руководителя в организации (навыки, стили руководителя и т.д.).
	}{
		Анализ финансовых результатов (задачи, объекты, этапы, источники информации, основные показатели, используемые методики).
	}{
		Понятие системы и системного подхода.
	}

\bigskip

\noindent\bilGE {26} 
	{
		Организационно-правовые формы хозяйствования в РФ и их характеристика. Методы изменения форм собственности и сущность.
	}{
		Жизненный цикл товара. Особенности ценообразования на различных стадиях.
	}{
		Конфликты в коллективе: опасности и возможности. Методы диагностики и оценки конфликтов.
	}

\bigskip

\noindent\bilGE {27} 
	{
		Менеджмент: сущность, содержание, цель и задачи. Функции управления.
	}{
		Виды анализа, его роль в управлении производством.
	}{
		Нормативные акты в области защиты информации и государственной тайны.
	}

\bigskip

\noindent\bilGE {28} 
	{
		Распределение прав и полномочий в менеджменте.
	}{
		Формы и системы заработной платы. Структура доходов работника предприятия.
	}{
		Функции управления персоналом, методы оценки их реализации.
	}

\bigskip

\noindent\bilGE {29} 
	{
		Доходы и расходы предприятия. Их классификация.
	}{
		Порядок расчёта себестоимости единицы продукции, работ и услуг на предприятии. Пути снижения себестоимости продукции и услуг.
	}{
		Государственное регулирование и государственная поддержка местного самоуправления.
	}

\bigskip

\noindent\bilGE {30} 
	{
		Классификация структур управления предприятием.
	}{
		Задачи и функции производственной логистики.
	}{
		Особенности отраслевой и территориальной структуры народного хозяйства Северо-Западного федерального округа.
	}

\bigskip

\end{document}