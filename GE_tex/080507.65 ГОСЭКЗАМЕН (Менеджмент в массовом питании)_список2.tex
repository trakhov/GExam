\documentclass[
	11pt,
	a4paper,
	]
	{article}

\usepackage{mysty, xparse, GExam, fancyhdr}

\usepackage[
	top = .7cm,
	right = 1cm,
	bottom = 1.2cm,
	left = 1cm,
	headheight = 14pt
	]{geometry}


\pagestyle{empty}

\begin{document}

\newlength{\pblength}\settowidth{\pblength}{Директор филиала СПбГЭУ в г. Пскове}

\hfill\parbox{\pblength}{
	\textbf{\textsc{Утверждаю}}\medskip

	Директор филиала СПбГЭУ в г. Пскове\medskip

	\makebox[3cm]{\hrulefill} А. М. Алексеева\medskip

	\makebox[1.5cm]{<<\hrulefill>>} \makebox[3cm]{\hrulefill}\ 2014 год
}


\smallGE * {080507.65} {Менеджмент организации}[Менеджмент в массовом питании]

	

\noindent\bilGE {1} 
	{
		Задачи внедрения принципов и методов современного менеджмента в современную хозяйственную практику в РФ.
	}{
		Понятие блага, продукта, услуги. Ограниченность ресурсов, безграничность потребностей.
	}{
		Виды и стратегии маркетинговой коммуникации на рынке товаров и услуг массового питания. Реклама и стимулирование сбыта.
	}

\bigskip

\noindent\bilGE {2} 
	{
		Значение совершенствования управления для успеха в коммерческой деятельности современных организаций.
	}{
		Экономическое развитие и экономический рост. Факторы экономического роста
	}{
		Задачи, функции и организация службы маркетинга на предприятии питания.
	}

\bigskip

\noindent\bilGE {3} 
	{
		Роль менеджмента в повышении эффективности хозяйствования в России. Проблемы развития менеджмента в современной России.
	}{
		Макроэкономическое равновесие и цикличность рыночной экономики. Запреты и ограничения неэкономического характера.
	}{
		Коммуникации в управлении. Виды коммуникаций. Этапы коммуникационного процесса.
	}

\bigskip

\noindent\bilGE {4} 
	{
		Организация как основа современного менеджмента. Жизненный цикл организации
	}{
		Спрос и предложения, факторы их определяющие. Закон спроса и предложения. Эластичность спроса и предложения.
	}{
		Ценообразование в организациях сферы услуг (специфика цены, виды, состав и структура цены). Калькуляции как способ расчета цены на единицу произведенной продукции в сфере питания
	}

\bigskip

\noindent\bilGE {5} 
	{
		Менеджмент как наука: сущность, структура и содержание.
	}{
		Классификация предприятий общественного питания и их характеристика. Услуги питания.
	}{
		Виды и характеристика стратегии бизнеса. Использование SWOT-анализа в стратегическом менеджменте
	}

\bigskip

\noindent\bilGE {6} 
	{
		Цели в системе современного менеджмента, основные требования, предъявляемые к ним Основные виды разделения управленческого труда: сущность, содержание
	}{
		Общая характеристика рынка питания в России: тенденции и перспективы развития.
	}{
		Типы базовых конкурентных стратегий: ценовое лидерство, дифференциация, фокусирование. Характерные черты базовых стратегий. Основные достоинства и опасности базовых стратегий.
	}

\bigskip

\noindent\bilGE {7} 
	{
		Система функций современного менеджмента. Сущность и основные виды. Организация реализации принятых решений как одна из основных функций менеджмента.
	}{
		Государственное регулирование бизнеса. Задачи государственного регулирования предпринимательской деятельности в сфере питания.
	}{
		Организация обслуживания банкета-коктейль. Банкет чай
	}

\bigskip

\noindent\bilGE {8} 
	{
		Функция мотивации персонала в выполнении принятых решений. Сущность процессуальных теорий мотивации.
	}{
		Организационно-правовые формы хозяйствования и их характеристики.
	}{
		Работа сомелье. Рекомендации по выбору и подачи вин. Правила подачи табачных изделий.
	}

\bigskip

\noindent\bilGE {9} 
	{
		Сущность и содержание функции контроля хода выполнения принятых решений.
	}{
		Доходы предприятий питания: понятие, виды, значение и состав. Пути повышения доходов.
	}{
		Организация обслуживания учащихся в общеобразовательных школах и студентов высших и средних учебных заведений.
	}

\bigskip

\noindent\bilGE {10} 
	{
		Сущность и задачи процесса принятия решений. Критерии принятия решений в условиях неопределенности.
	}{
		Затраты предприятий питания: их сущность, классификация и основные направления оптимизации.
	}{
		Организация обслуживания туристических групп в ресторанах и кафе.
	}

\bigskip

\noindent\bilGE {11} 
	{
		Современные школы менеджмента: общая характеристика.
	}{
		Формы и методы налогообложения предприятий питания.
	}{
		Основные и оборотные фонды предприятий питания: состав, структура.
	}

\bigskip

\noindent\bilGE {12} 
	{
		Особенности японской модели менеджмента: общая характеристика. Характер принятия управленческих решений и ответственности на японских фирмах.
	}{
		Прибыль как экономическая категория, ее сущность и значение.
	}{
		Источники снабжения на предприятия общественного питания. Формы и способы доставки продуктов.
	}

\bigskip

\noindent\bilGE {13} 
	{
		Характерные черты американской практики менеджмента: общая характеристика. Сравнительная характеристика американской и японской моделей менеджмента.
	}{
		Рентабельность предприятий питания: понятие, значение и методы расчета. Факторы, влияющие на рентабельность предприятий питания.
	}{
		Функции предприятия общественного питания. Общие требования к организации рабочих мест в цехах.
	}

\bigskip

\noindent\bilGE {14} 
	{
		Основные цели современного предприятия сферы питания. Роль предприятий массового питания в экономике страны
	}{
		Способы выявления влияния факторов на изменения результативного показателя в экономическом анализе деятельности предприятия в сфере питания
	}{
		Прогрессивные технологии в обслуживании на предприятия массового питания и их характеристика.
	}

\bigskip

\noindent\bilGE {15} 
	{
		Организационная структура, основные классификации. Структуры, ориентированные на нововведения. Основные принципы построения современных организационных структур.
	}{
		Конкуренция: ее сущность, виды и роль в механизме функционирования рынка. Особенности формирования и использования конкурентных преимуществ в сфере услуг питания.
	}{
		Характеристика методов подачи блюд на предприятиях общественного питания.
	}

\bigskip

\noindent\bilGE {16} 
	{
		Внутренняя среда организации и ее основные элементы. Характеристика элементов внутренней среды предприятия питания.
	}{
		Конкурентоспособность товара (услуги), технико-экономические и социально-организационные факторы. Конкурентоспособность предприятий массового питания.
	}{
		Назначение и принципы составления меню, карты вин, коктейлей и их оформление. Виды меню.
	}

\bigskip

\noindent\bilGE {17} 
	{
		Корпоративная культура и ее влияние на работу предприятия. Использование корпоративной культуры на предприятиях массового питания.
	}{
		Элементы инфраструктуры бизнеса. Функции и задачи инфраструктуры малых предприятий.
	}{
		Основные элементы обслуживания потребителей в ресторане. Организация процесса обслуживания в зале.
	}

\bigskip

\noindent\bilGE {18} 
	{
		Внешняя среда фирмы: сущность, основные элементы. Характеристика элементов внешней среды предприятия питания.
	}{
		Малое предпринимательство в России. Экономические, социальные и правовые условия предпринимательства. Современные формы организации малого бизнеса.
	}{
		Правила подачи продукции сервис-бара.
	}

\bigskip

\noindent\bilGE {19} 
	{
		Стратегическое планирование деятельности фирмы: сущность, решаемые проблемы, основные требования. Основные компоненты стратегического плана фирмы, их содержание.
	}{
		Развитие малых предприятий в массовом питании. Мировой опыт и российская специфика.
	}{
		Организация обслуживания на производственных предприятиях. Методы и формы обслуживания.
	}

\bigskip

\noindent\bilGE {20} 
	{
		Понятие, сущность и значение текущего планирования. Сравнительная характеристика стратегического и текущего планирования.
	}{
		Государственное регулирование сферы массового питания: цели, основные инструменты.
	}{
		Организация обслуживания банкетов за столом с полным и частичным обслуживанием официантами.
	}

\bigskip

\noindent\bilGE {21} 
	{
		Бизнес-план предприятия массового питания: понятие, основные разделы и этапы разработки.
	}{
		Понятие, сущность инновации, и ее роль в достижении целей производства услуг. Виды инноваций.
	}{
		Содержание и функции ресторанного бизнеса. Понятие ресторанного рынка.
	}

\bigskip

\noindent\bilGE {22} 
	{
		Специфика труда менеджера. Основные требования к руководителю предприятия массового питания.
	}{
		Инновационный потенциал современного менеджмента. Задачи менеджмента в области инноваций.
	}{
		Организация обслуживания питанием на транспорте
	}

\bigskip

\noindent\bilGE {23} 
	{
		Мотивация трудовой деятельности. Теории мотивации. Специфика мотивации сотрудников предприятий массового питания.
	}{
		Сущность инновационного предпринимательства. Венчурный капитал в инновационном предпринимательстве.
	}{
		Роль менеджера в организации банкетной службы.
	}

\bigskip

\noindent\bilGE {24} 
	{
		Методы отбора, приема персонала. Состав и структура персонала.
	}{
		Классификация инноваций и организационно-правовые формы инновационного предпринимательства.
	}{
		Правила подбора напитков к блюдам и закускам.
	}

\bigskip

\noindent\bilGE {25} 
	{
		Основные требования к созданию оптимальных условий труда на предприятиях общественного питания.
	}{
		Оценка эффективных инновационных проектов. Инновационная политика.
	}{
		Современные методы подачи блюд.
	}

\bigskip

\noindent\bilGE {26} 
	{
		Полномочия, их распределение и делегирование на предприятии: этапы, основные требования.
	}{
		Понятие и сущность риск-менеджмента. Риски и их виды. Показатели и методы количественной оценки рисков.
	}{
		Этикет еды – общие правила потребления блюд.
	}

\bigskip

\noindent\bilGE {27} 
	{
		Виды и формы оплаты труда. Принципы стимулирования оплаты труда. Модель стимулирующей оплаты труда.
	}{
		Понятие качества услуги и его место в менеджменте организации. Организация контроля производства и качества блюд, напитков и изделий, реализуемых предприятием питания.
	}{
		Дипломатические приемы и банкеты.
	}

\bigskip

\noindent\bilGE {28} 
	{
		Конфликты в организациях: сущность, природа. Современная типология конфликтов в организациях и причины их возникновения. Методы разрешения конфликтных ситуаций в коллективе.
	}{
		Франчайзинг как форма взаимодействия предприятий крупного и малого бизнеса. Лизинг и его разновидности.
	}{
		Обслуживающий персонал в предприятиях общественного питания, требования к нему.
	}

\bigskip

\noindent\bilGE {29} 
	{
		Пути повышения эффективности использования персонала. Аттестация работников ресторана.
	}{
		Маркетинговые альянсы: понятие и критерии целесообразности создания. Возможные схемы формирования маркетинговых альянсов.
	}{
		Организация обслуживания питанием в номерах гостиницы.
	}

\bigskip

\noindent\bilGE {30} 
	{
		Организация как открытая или закрытая система. Сравнительная характеристика систем.
	}{
		Методы маркетинговых исследований, анализа и представления результатов маркетинговых исследований. Бенчмаркинг.
	}{
		Кейтеринг и мерчендайзинг в общественном питании.
	}

\bigskip

\noindent\bilGE {31} 
	{
		Основные законы организации. Сущность проявления и действия законов организации на современных предприятиях.
	}{
		Процесс и методы сегментации рынка. Особенности сегментации рынка предприятий массового питания.
	}{
		Организация обслуживания питанием в местах массового отдыха.
	}

\bigskip

\noindent\bilGE {32} 
	{
		Миссия и цели организации. Принципы формулирования миссии предприятия питания. Формирование целей.
	}{
		Этапы процесса разработки нового товара (услуги). Жизненный цикл товара (услуги). Услуга питания: маркетинговая политика на разных этапах развития
	}{
		Составление и оформление организационно-распорядительных документов на предприятии питания: штатное расписание, положение о структурном подразделении, должностная инструкции.
	}

\bigskip

\noindent\bilGE {33} 
	{
		Рынок, понятие и характеристики. Разновидности рынков. Рынок услуг предприятий массового питания как объект управления.
	}{
		Определение целевого сегмента и позиционирование услуг в системе разработки услуги.
	}{
		Документирование движения персонала на предприятии питания (прием, увольнение, перевод, отпуск, командировки)
	}

\bigskip

\noindent\bilGE {34} 
	{
		Организация как основа современного менеджмента. Жизненный цикл организации
	}{
		Этапы процесса разработки нового товара (услуги). Жизненный цикл товара (услуги). Услуга питания: маркетинговая политика на разных этапах развития
	}{
		Требования к оформлению документов на предприятии питания. Оперативное хранение. Передача дел в архив.
	}

\bigskip

\end{document}