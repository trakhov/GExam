\documentclass[
	11pt,
	a4paper,
	]
	{article}

\usepackage{mysty, xparse, GExam, fancyhdr}

\usepackage[
	top = .7cm,
	right = 1cm,
	bottom = 1.2cm,
	left = 1cm,
	headheight = 14pt
	]{geometry}


\pagestyle{fancy}
\renewcommand{\headrulewidth}{0pt}
\renewcommand{\footrulewidth}{0pt}


\lhead{\begin{tikzpicture}[remember picture, overlay]
	\draw[dashed] (current page.north west) ++ (0, -99mm) -- +(210mm, 0);
	\draw[dashed] (current page.north west) ++ (0, -198mm) -- +(210mm, 0);
\end{tikzpicture}}
\cfoot{}



\begin{document}

	

\begin{minipage}[t][\miniH]{\miniL}\centering
	\topGE  {080100.62} {Экономика}
	\bilGE {1} 
		{
			Сущность рынка и его виды. Конъюнктура рынка.
		}{
			Порядок расчёта себестоимости единицы продукции, работ и услуг на предприятии. Пути снижения себестоимости продукции и услуг.
		}{
			Анализ себестоимости (задачи, объекты, этапы, источники информации, основные показатели, используемые методики).
		}
	\lowGE
\end{minipage}

\vfill



\begin{minipage}[t][\miniH]{\miniL}\centering
	\topGE  {080100.62} {Экономика}
	\bilGE {2} 
		{
			Рыночный механизм как взаимосвязь и взаимодействие элементов рынка: спроса, предложения и цены.
		}{
			Общие и частные показатели эффективности хозяйственной деятельности предприятия, пути ее повышения.
		}{
			Регионализация, понятие о регионе, виды регионов.
		}
	\lowGE
\end{minipage}

\vfill



\begin{minipage}[t][\miniH]{\miniL}\centering
	\topGE  {080100.62} {Экономика}
	\bilGE {3} 
		{
			Функции спроса и предложения. Точка равновесия. Эластичность спроса и предложения.
		}{
			Структура цены. Методы ценообразования. Методы государственного регулирования цен на продукцию и услуги.
		}{
			Государственная региональная политика и ее аспекты.
		}
	\lowGE
\end{minipage}





\begin{minipage}[t][\miniH]{\miniL}\centering
	\topGE  {080100.62} {Экономика}
	\bilGE {4} 
		{
			Методы государственного регулирования экономики.
		}{
			Жизненный цикл товара. Особенности ценообразования на различных стадиях.
		}{
			Особенности развития Псковской области: стратегии и проекты.
		}
	\lowGE
\end{minipage}

\vfill



\begin{minipage}[t][\miniH]{\miniL}\centering
	\topGE  {080100.62} {Экономика}
	\bilGE {5} 
		{
			Менеджмент: сущность, содержание, цель и задачи. Функции управления.
		}{
			Сущность и роль планирования на предприятии, виды планов.
		}{
			Классификация отраслей хозяйства по назначению продукции и по характеру предметов труда.
		}
	\lowGE
\end{minipage}

\vfill



\begin{minipage}[t][\miniH]{\miniL}\centering
	\topGE  {080100.62} {Экономика}
	\bilGE {6} 
		{
			Вклад различных направлений научных школ в науку и практику управления.
		}{
			Роль и структура бизнес-плана.
		}{
			Особенности отраслевой и территориальной структуры народного хозяйства Северо-Западного федерального округа.
		}
	\lowGE
\end{minipage}





\begin{minipage}[t][\miniH]{\miniL}\centering
	\topGE  {080100.62} {Экономика}
	\bilGE {7} 
		{
			Организационно-правовые формы хозяйствования в РФ и их характеристика. Методы изменения форм собственности и сущность.
		}{
			Инвестиции, их классификация и источники формирования.
		}{
			Понятие о праве и его определение. Специфические признаки права.
		}
	\lowGE
\end{minipage}

\vfill



\begin{minipage}[t][\miniH]{\miniL}\centering
	\topGE  {080100.62} {Экономика}
	\bilGE {8} 
		{
			Распределение прав и полномочий в менеджменте.
		}{
			Методы оценки эффективности инвестиций.
		}{
			Понятие трудовых правоотношений. Трудовой договор. Трудовой распорядок.
		}
	\lowGE
\end{minipage}

\vfill



\begin{minipage}[t][\miniH]{\miniL}\centering
	\topGE  {080100.62} {Экономика}
	\bilGE {9} 
		{
			Роль руководителя в организации (навыки, стили руководителя и т.д.).
		}{
			Принципы обеспечения качества и методы контроля качества. Частные и общие факторы качества.
		}{
			Нормативные акты в области защиты информации и государственной тайны.
		}
	\lowGE
\end{minipage}





\begin{minipage}[t][\miniH]{\miniL}\centering
	\topGE  {080100.62} {Экономика}
	\bilGE {10} 
		{
			Классическая теория организации и основные принципы организации управления.
		}{
			Качество и конкурентоспособность продукции и услуг: оценка и пути повышения.
		}{
			Роль и место управления персоналом в областях менеджмента, маркетинга, психологии, социологии, экономики и экономики труда.
		}
	\lowGE
\end{minipage}

\vfill



\begin{minipage}[t][\miniH]{\miniL}\centering
	\topGE  {080100.62} {Экономика}
	\bilGE {11} 
		{
			Классификация структур управления предприятием.
		}{
			Формы организации и оплаты труда на предприятии.
		}{
			Методы управления персоналом: сущность и практическое применение.
		}
	\lowGE
\end{minipage}

\vfill



\begin{minipage}[t][\miniH]{\miniL}\centering
	\topGE  {080100.62} {Экономика}
	\bilGE {12} 
		{
			Характеристика различных типов организационных структур.
		}{
			Формы и системы заработной платы. Структура доходов работника предприятия.
		}{
			Формы и цель планирование деловой карьеры сотрудников.
		}
	\lowGE
\end{minipage}





\begin{minipage}[t][\miniH]{\miniL}\centering
	\topGE  {080100.62} {Экономика}
	\bilGE {13} 
		{
			Типология организационных культур и динамика развития организации.
		}{
			НДС: экономическое содержание, объект и субъект налогообложения, порядок расчета
		}{
			Конфликты в коллективе: опасности и возможности. Методы диагностики и оценки конфликтов.
		}
	\lowGE
\end{minipage}

\vfill



\begin{minipage}[t][\miniH]{\miniL}\centering
	\topGE  {080100.62} {Экономика}
	\bilGE {14} 
		{
			Сущность и задачи процесса принятия решений.
		}{
			Налог на прибыль: экономическое содержание, объект и субъект налогообложения, порядок расчета
		}{
			Условия труда персонала: принципы и тенденции.
		}
	\lowGE
\end{minipage}

\vfill



\begin{minipage}[t][\miniH]{\miniL}\centering
	\topGE  {080100.62} {Экономика}
	\bilGE {15} 
		{
			Методы принятия решений в менеджменте.
		}{
			ЕСН: экономическое содержание, объект и субъект налогообложения, порядок расчета
		}{
			Функции управления персоналом, методы оценки их реализации.
		}
	\lowGE
\end{minipage}





\begin{minipage}[t][\miniH]{\miniL}\centering
	\topGE  {080100.62} {Экономика}
	\bilGE {16} 
		{
			Управленческий контроль как одна из функций управления. Формы контроля реализации решения.
		}{
			Единый налог по упрощенной системе налогообложения: экономическое содержание, объект и субъект налогообложения, порядок расчета
		}{
			Понятия «регулирование экономики», управление экономикой и «государственное регулирование экономики». Необходимость государственного регулирования экономики.
		}
	\lowGE
\end{minipage}

\vfill



\begin{minipage}[t][\miniH]{\miniL}\centering
	\topGE  {080100.62} {Экономика}
	\bilGE {17} 
		{
			Основные фонды предприятия, их состав, виды стоимостной оценки, источники формирования и пополнения. Оценка их состояния и использования.
		}{
			Единый налог на вмененный доход: экономическое содержание, объект и субъект налогообложения, порядок расчета
		}{
			Государственный сектор экономики и государственная собственность как инструменты регулирования.
		}
	\lowGE
\end{minipage}

\vfill



\begin{minipage}[t][\miniH]{\miniL}\centering
	\topGE  {080100.62} {Экономика}
	\bilGE {18} 
		{
			Оборотные средства предприятия, их состав, источники формирования и пополнения. Оценка их состояния и использования.
		}{
			История логистики. Понятие и содержание логистики.
		}{
			Сущность и цели денежно-кредитной политики. Понятие кредитной и денежной систем государства. Инструменты денежно-кредитной политики.
		}
	\lowGE
\end{minipage}





\begin{minipage}[t][\miniH]{\miniL}\centering
	\topGE  {080100.62} {Экономика}
	\bilGE {19} 
		{
			Показатели, характеризующие кадровый потенциал предприятия сервиса. Оценка состояния и использования трудовых ресурсов предприятия.
		}{
			Логистика как фактор повышения конкурентоспособности.
		}{
			Социальная политика современного государства: цели, основные направления.
		}
	\lowGE
\end{minipage}

\vfill



\begin{minipage}[t][\miniH]{\miniL}\centering
	\topGE  {080100.62} {Экономика}
	\bilGE {20} 
		{
			Понятие производительности труда. Основные показатели оценки уровня и динамики производительности труда, пути повышения производительности труда.
		}{
			Задачи и функции производственной логистики.
		}{
			Общественный сектор экономики: причины усиления экономической роли государства.
		}
	\lowGE
\end{minipage}

\vfill



\begin{minipage}[t][\miniH]{\miniL}\centering
	\topGE  {080100.62} {Экономика}
	\bilGE {21} 
		{
			Порядок формирования чистой прибыли предприятия. Содержание отчета о прибылях и убытках.
		}{
			Способы снижения уровня логистических затрат.
		}{
			Несостоятельность государства и провалы рынка. Виды провалов рынка.
		}
	\lowGE
\end{minipage}





\begin{minipage}[t][\miniH]{\miniL}\centering
	\topGE  {080100.62} {Экономика}
	\bilGE {22} 
		{
			Доходы и расходы предприятия. Их классификация.
		}{
			Виды анализа, его роль в управлении производством.
		}{
			Место и роль общественного сектора в рыночной экономике.
		}
	\lowGE
\end{minipage}

\vfill



\begin{minipage}[t][\miniH]{\miniL}\centering
	\topGE  {080100.62} {Экономика}
	\bilGE {23} 
		{
			Постоянные и переменные затраты. Эффект масштаба.
		}{
			Анализ использования трудовых ресурсов (задачи, объекты, этапы, источники информации, основные показатели, используемые методики).
		}{
			Место и роль общественного сектора в смешанной экономике.
		}
	\lowGE
\end{minipage}

\vfill



\begin{minipage}[t][\miniH]{\miniL}\centering
	\topGE  {080100.62} {Экономика}
	\bilGE {24} 
		{
			Точка безубыточности: экономическая сущность и способ расчета.
		}{
			Анализ использования основных фондов, (задачи, объекты, этапы, источники информации, основные показатели, используемые методики).
		}{
			Понятие общественных расходов. Общественные расходы в Российской Федерации.
		}
	\lowGE
\end{minipage}





\begin{minipage}[t][\miniH]{\miniL}\centering
	\topGE  {080100.62} {Экономика}
	\bilGE {25} 
		{
			Распределение прав и полномочий в менеджменте.
		}{
			Анализ финансовых результатов (задачи, объекты, этапы, источники информации, основные показатели, используемые методики).
		}{
			Перспективы развития общественного сектора экономики Российской Федерации.
		}
	\lowGE
\end{minipage}

\vfill



\begin{minipage}[t][\miniH]{\miniL}\centering
	\topGE  {080100.62} {Экономика}
	\bilGE {26} 
		{
			Порядок формирования чистой прибыли предприятия. Содержание отчета о прибылях и убытках.
		}{
			Анализ использования трудовых ресурсов (задачи, объекты, этапы, источники информации, основные показатели, используемые методики).
		}{
			Место и роль общественного сектора в смешанной экономике.
		}
	\lowGE
\end{minipage}

\vfill



\begin{minipage}[t][\miniH]{\miniL}\centering
	\topGE  {080100.62} {Экономика}
	\bilGE {27} 
		{
			Точка безубыточности: экономическая сущность и способ расчета.
		}{
			Единый налог по упрощенной системе налогообложения: экономическое содержание, объект и субъект налогообложения, порядок расчета
		}{
			Формы и цель планирование деловой карьеры сотрудников.
		}
	\lowGE
\end{minipage}





\begin{minipage}[t][\miniH]{\miniL}\centering
	\topGE  {080100.62} {Экономика}
	\bilGE {28} 
		{
			Типология организационных культур и динамика развития организации.
		}{
			Жизненный цикл товара. Особенности ценообразования на различных стадиях.
		}{
			Условия труда персонала: принципы и тенденции.
		}
	\lowGE
\end{minipage}

\vfill



\begin{minipage}[t][\miniH]{\miniL}\centering
	\topGE  {080100.62} {Экономика}
	\bilGE {29} 
		{
			Вклад различных направлений научных школ в науку и практику управления.
		}{
			Способы снижения уровня логистических затрат.
		}{
			Сущность и цели денежно-кредитной политики. Понятие кредитной и денежной систем государства. Инструменты денежно-кредитной политики.
		}
	\lowGE
\end{minipage}

\vfill



\begin{minipage}[t][\miniH]{\miniL}\centering
	\topGE  {080100.62} {Экономика}
	\bilGE {30} 
		{
			Рыночный механизм как взаимосвязь и взаимодействие элементов рынка: спроса, предложения и цены.
		}{
			Качество и конкурентоспособность продукции и услуг: оценка и пути повышения.
		}{
			Социальная политика современного государства: цели, основные направления.
		}
	\lowGE
\end{minipage}



	

\end{document}