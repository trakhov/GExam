\documentclass[
	11pt,
	a4paper,
	]
	{article}

\usepackage{mysty, xparse, GExam, fancyhdr}

\usepackage[
	top = .7cm,
	right = 1cm,
	bottom = 1.2cm,
	left = 1cm,
	headheight = 14pt
	]{geometry}


\pagestyle{fancy}
\renewcommand{\headrulewidth}{0pt}
\renewcommand{\footrulewidth}{0pt}


\lhead{\begin{tikzpicture}[remember picture, overlay]
	\draw[dashed] (current page.north west) ++ (0, -99mm) -- +(210mm, 0);
	\draw[dashed] (current page.north west) ++ (0, -198mm) -- +(210mm, 0);
\end{tikzpicture}}
\cfoot{}



\begin{document}

	

\begin{minipage}[t][\miniH]{\miniL}\centering
	\topGE * {080507.65} {Менеджмент организации}[Гостиничный и туристический бизнес]
	\bilGE {1} 
		{
			Задачи внедрения принципов и методов современного менеджмента в современную хозяйственную практику в РФ.
		}{
			Формирование целей организации. Функционирования организации. Организационная и информационная структуры.
		}{
			Основные требования к созданию оптимальных условий труда на предприятиях сферы туризма.
		}
	\lowGE
\end{minipage}

\vfill



\begin{minipage}[t][\miniH]{\miniL}\centering
	\topGE * {080507.65} {Менеджмент организации}[Гостиничный и туристический бизнес]
	\bilGE {2} 
		{
			Значение совершенствования управления для успеха в коммерческой деятельности современных организаций.
		}{
			Государственное регулирование бизнеса. Задачи государственного регулирования предпринимательской деятельности в сфере туризма.
		}{
			Полномочия, их распределение и делегирование на предприятии: этапы, основные требования.
		}
	\lowGE
\end{minipage}

\vfill



\begin{minipage}[t][\miniH]{\miniL}\centering
	\topGE * {080507.65} {Менеджмент организации}[Гостиничный и туристический бизнес]
	\bilGE {3} 
		{
			Роль менеджмента в повышении эффективности хозяйствования в России. Проблемы развития менеджмента в современной России.
		}{
			Типы базовых конкурентных стратегий: ценовое лидерство, дифференциация, фокусирование. Характерные черты базовых стратегий. Основные достоинства и опасности базовых стратегий.
		}{
			Виды и формы оплаты труда. Принципы стимулирования оплаты труда. Модель стимулирующей оплаты труда.
		}
	\lowGE
\end{minipage}





\begin{minipage}[t][\miniH]{\miniL}\centering
	\topGE * {080507.65} {Менеджмент организации}[Гостиничный и туристический бизнес]
	\bilGE {4} 
		{
			Организация как основа современного менеджмента. Жизненный цикл организации
		}{
			Понятие конкуренции и конкурентоспособности товара. Факторы, определяющие конкурентоспособность товара
		}{
			Коммуникации и переговоры в туризме. Функции делового общения.
		}
	\lowGE
\end{minipage}

\vfill



\begin{minipage}[t][\miniH]{\miniL}\centering
	\topGE * {080507.65} {Менеджмент организации}[Гостиничный и туристический бизнес]
	\bilGE {5} 
		{
			Менеджмент как наука: сущность, структура и содержание.
		}{
			Понятие конкуренции и конкурентоспособности предприятия. Факторы, влияющие на конкурентоспособность предприятия сферы туризма.
		}{
			Пути повышения эффективности использования персонала. Аттестация работников сферы туризма.
		}
	\lowGE
\end{minipage}

\vfill



\begin{minipage}[t][\miniH]{\miniL}\centering
	\topGE * {080507.65} {Менеджмент организации}[Гостиничный и туристический бизнес]
	\bilGE {6} 
		{
			Цели в системе современного менеджмента, основные требования, предъявляемые к ним.
		}{
			Содержание и этапы разработки ценовой стратегии. Методы работы с клиентом-туристом в форматах ценового стимулирования
		}{
			Методы подготовки и повышения квалификации работников гостиничных предприятий.
		}
	\lowGE
\end{minipage}





\begin{minipage}[t][\miniH]{\miniL}\centering
	\topGE * {080507.65} {Менеджмент организации}[Гостиничный и туристический бизнес]
	\bilGE {7} 
		{
			Основные виды разделения управленческого труда: сущность, содержание
		}{
			Деловые стратегии: портфельные стратегии, стратегии роста, стратегии вертикальной интеграции, конкурентные стратегии. Характеристики и условия применения
		}{
			Классификация затрат туристской фирмы (турагент, туроператор).
		}
	\lowGE
\end{minipage}

\vfill



\begin{minipage}[t][\miniH]{\miniL}\centering
	\topGE * {080507.65} {Менеджмент организации}[Гостиничный и туристический бизнес]
	\bilGE {8} 
		{
			Система функций современного менеджмента. Сущность и основные виды. Организация реализации принятых решений как одна из основных функций менеджмента.
		}{
			Эластичность спроса и предложения, коэффициент эластичности спроса по цене.
		}{
			Состав затрат гостиничного предприятия.
		}
	\lowGE
\end{minipage}

\vfill



\begin{minipage}[t][\miniH]{\miniL}\centering
	\topGE * {080507.65} {Менеджмент организации}[Гостиничный и туристический бизнес]
	\bilGE {9} 
		{
			Функция мотивации персонала в выполнении принятых решений. Сущность содержательных и процессуальных теорий мотивации.
		}{
			Понятие блага, продукта, услуги. Ограниченность ресурсов, безграничность потребностей.
		}{
			Особенности менеджмента в туризме и гостиничном сервисе.
		}
	\lowGE
\end{minipage}





\begin{minipage}[t][\miniH]{\miniL}\centering
	\topGE * {080507.65} {Менеджмент организации}[Гостиничный и туристический бизнес]
	\bilGE {10} 
		{
			Сущность и содержание функции контроля хода выполнения принятых решений.
		}{
			Экономическое развитие и экономический рост. Факторы экономического роста
		}{
			Классификация видов туризма. Инфраструктура туризма: состав организаций и их взаимодействие
		}
	\lowGE
\end{minipage}

\vfill



\begin{minipage}[t][\miniH]{\miniL}\centering
	\topGE * {080507.65} {Менеджмент организации}[Гостиничный и туристический бизнес]
	\bilGE {11} 
		{
			Сущность и задачи процесса принятия решений. Критерии принятия решений в условиях неопределенности.
		}{
			Макроэкономическое равновесие и цикличность рыночной экономики. Запреты и ограничения неэкономического характера.
		}{
			Сезонность спроса на туристские гостиничные услуги. Методы оценки. Способы компенсации сезонных спадов.
		}
	\lowGE
\end{minipage}

\vfill



\begin{minipage}[t][\miniH]{\miniL}\centering
	\topGE * {080507.65} {Менеджмент организации}[Гостиничный и туристический бизнес]
	\bilGE {12} 
		{
			Современные школы менеджмента: общая характеристика.
		}{
			Спрос и предложения, факторы их определяющие. Закон спроса и предложения. Эластичность спроса и предложения.
		}{
			Сущность программно-целевого подхода к управлению в туризме. Федеральные и региональные программы развития туризма.
		}
	\lowGE
\end{minipage}





\begin{minipage}[t][\miniH]{\miniL}\centering
	\topGE * {080507.65} {Менеджмент организации}[Гостиничный и туристический бизнес]
	\bilGE {13} 
		{
			Особенности японской модели менеджмента: общая характеристика. Характер принятия управленческих решений и ответственности на японских фирмах.
		}{
			Конкуренция. Типы конкурентных структур рынка. Факторы конкурентоспособности.
		}{
			Организация управления в туризме и гостиничном бизнесе на федеральном и региональном уровнях
		}
	\lowGE
\end{minipage}

\vfill



\begin{minipage}[t][\miniH]{\miniL}\centering
	\topGE * {080507.65} {Менеджмент организации}[Гостиничный и туристический бизнес]
	\bilGE {14} 
		{
			Характерные черты американской практики менеджмента: общая характеристика. Сравнительная характеристика американской и японской моделей менеджмента.
		}{
			Маркетинговые стратегии для организации сферы туристских услуг.
		}{
			Сетевые формы организации гостиничного и туристского бизнеса
		}
	\lowGE
\end{minipage}

\vfill



\begin{minipage}[t][\miniH]{\miniL}\centering
	\topGE * {080507.65} {Менеджмент организации}[Гостиничный и туристический бизнес]
	\bilGE {15} 
		{
			Основные цели современного предприятия сферы туризма. Роль туристических предприятий в развитии экономики региона, страны
		}{
			Методы продвижения услуг. Стимулирование спроса на туруслуги
		}{
			Международная классификация гостиниц. Классификация гостиниц в Российской Федерации.
		}
	\lowGE
\end{minipage}





\begin{minipage}[t][\miniH]{\miniL}\centering
	\topGE * {080507.65} {Менеджмент организации}[Гостиничный и туристический бизнес]
	\bilGE {16} 
		{
			Организационная структура, основные классификации. Основные принципы построения современных организационных структур. Структуры, ориентированные на нововведения.
		}{
			Методы ценообразования. Структура цены. Калькуляция туруслуги
		}{
			Организационная структура управления гостиничным предприятием. Службы гостиницы и их характеристика.
		}
	\lowGE
\end{minipage}

\vfill



\begin{minipage}[t][\miniH]{\miniL}\centering
	\topGE * {080507.65} {Менеджмент организации}[Гостиничный и туристический бизнес]
	\bilGE {17} 
		{
			Внутренняя среда организации и ее основные элементы. Характеристика элементов внутренней среды предприятия сферы туризма.
		}{
			Прибыль предприятия сферы туризма. Условия максимизации прибыли предприятия сферы туризма.
		}{
			Сегментирование клиентов в туризме. Учёт психологических аспектов в обслуживание клиентов.
		}
	\lowGE
\end{minipage}

\vfill



\begin{minipage}[t][\miniH]{\miniL}\centering
	\topGE * {080507.65} {Менеджмент организации}[Гостиничный и туристический бизнес]
	\bilGE {18} 
		{
			Внешняя среда фирмы: сущность, основные элементы. Характеристика элементов внешней среды предприятия сферы туризма.
		}{
			Доходы предприятия сферы туризма: понятие, виды, значение и состав. Пути повышения доходов.
		}{
			Особенности регистрации и размещения гостей. Виды и правила расчетов за проживание.
		}
	\lowGE
\end{minipage}





\begin{minipage}[t][\miniH]{\miniL}\centering
	\topGE * {080507.65} {Менеджмент организации}[Гостиничный и туристический бизнес]
	\bilGE {19} 
		{
			Характеристика и функции корпоративной культуры. Методы формирования и поддержания корпоративной культуры.
		}{
			Издержки предприятия сферы туризма: их сущность, классификация и основные направления оптимизации издержек.
		}{
			Системы бронирования. Продажи в GDS/ADS.
		}
	\lowGE
\end{minipage}

\vfill



\begin{minipage}[t][\miniH]{\miniL}\centering
	\topGE * {080507.65} {Менеджмент организации}[Гостиничный и туристический бизнес]
	\bilGE {20} 
		{
			Бизнес-план фирмы – сущность, структура, общая характеристика основных разделов.
		}{
			Рентабельность предприятия сферы туризма: понятие, значение и методы расчета. Факторы, влияющие на рентабельность.
		}{
			Использование информационных технологий в деятельности туристских организаций
		}
	\lowGE
\end{minipage}

\vfill



\begin{minipage}[t][\miniH]{\miniL}\centering
	\topGE * {080507.65} {Менеджмент организации}[Гостиничный и туристический бизнес]
	\bilGE {21} 
		{
			Виды и формы оплаты труда. Принципы стимулирования оплаты труда. Модель стимулирующей оплаты труда.
		}{
			Себестоимость продаж: понятие, методы определения, факторы, влияющие на себестоимость. Соотношение себестоимости и прибыли, их взаимосвязь.
		}{
			Организация питания в гостиницах
		}
	\lowGE
\end{minipage}





\begin{minipage}[t][\miniH]{\miniL}\centering
	\topGE * {080507.65} {Менеджмент организации}[Гостиничный и туристический бизнес]
	\bilGE {22} 
		{
			Роль менеджера в управлении предприятием.
		}{
			Характеристика ресурсов предприятия сферы туризма, пути их мобилизации.
		}{
			Разработка маршрутов и формирование туров. Классификация, основные требования.
		}
	\lowGE
\end{minipage}

\vfill



\begin{minipage}[t][\miniH]{\miniL}\centering
	\topGE * {080507.65} {Менеджмент организации}[Гостиничный и туристический бизнес]
	\bilGE {23} 
		{
			Кадровая политика на предприятии. Особенности и формы управления персоналом на предприятии сферы туризма.
		}{
			Способы выявления влияния факторов на изменения результативного показателя в экономическом анализе деятельности предприятия сферы туризма
		}{
			Минимальные требования к оборудованию гостиниц категорий различных категорий (одна – пять звезд).
		}
	\lowGE
\end{minipage}

\vfill



\begin{minipage}[t][\miniH]{\miniL}\centering
	\topGE * {080507.65} {Менеджмент организации}[Гостиничный и туристический бизнес]
	\bilGE {24} 
		{
			Организация контроля за результатами работы фирмы. Методы контроля на предприятии сферы туризма.
		}{
			Виды и источники инвестиций в сфере туризма. Значение процесса инвестирования для развития дестинации.
		}{
			Особенности рекламы в туризме и гостиничном бизнесе. Основные виды рекламных носителей.
		}
	\lowGE
\end{minipage}





\begin{minipage}[t][\miniH]{\miniL}\centering
	\topGE * {080507.65} {Менеджмент организации}[Гостиничный и туристический бизнес]
	\bilGE {25} 
		{
			Виды, цели и компоненты контроллинга. Специфика контроллинга в сфере туризма.
		}{
			Формы и методы налогообложения предприятий туризма. Ответственность предприятия- налогоплательщика за нарушение налогового законодательства.
		}{
			Формирование рекламного бюджета в туризме и гостиничном бизнесе. Методы оценки эффективности рекламы.
		}
	\lowGE
\end{minipage}

\vfill



\begin{minipage}[t][\miniH]{\miniL}\centering
	\topGE * {080507.65} {Менеджмент организации}[Гостиничный и туристический бизнес]
	\bilGE {26} 
		{
			Понятие и сущность риск-менеджмента. Риски и их виды. Показатели и методы количественной оценки рисков.
		}{
			Организационно-правовые формы хозяйствования и их характеристика. Методы изменения форм собственности и сущность.
		}{
			Понятие «бренд» туристской организации и гостиничного комплекса. Основные характеристики и задачи брендинга в туризме и гостиничном сервисе.
		}
	\lowGE
\end{minipage}

\vfill



\begin{minipage}[t][\miniH]{\miniL}\centering
	\topGE * {080507.65} {Менеджмент организации}[Гостиничный и туристический бизнес]
	\bilGE {27} 
		{
			Инновационный потенциал современного менеджмента. Задачи менеджмента в области инноваций.
		}{
			НДС: экономическое содержание и порядок расчета.
		}{
			Составление и оформление организационно-распорядительных документов: штатное расписание, положение о структурном подразделении, должностные инструкции.
		}
	\lowGE
\end{minipage}





\begin{minipage}[t][\miniH]{\miniL}\centering
	\topGE * {080507.65} {Менеджмент организации}[Гостиничный и туристический бизнес]
	\bilGE {28} 
		{
			Управление инновациями на современном предприятии туризма.
		}{
			Налог на прибыль: определение, значение для предприятия и процедура взимания. Порядок расчета налога на прибыль.
		}{
			Составление номенклатуры дел. Правила формирования дел. Контроль за исполнением документов.
		}
	\lowGE
\end{minipage}

\vfill



\begin{minipage}[t][\miniH]{\miniL}\centering
	\topGE * {080507.65} {Менеджмент организации}[Гостиничный и туристический бизнес]
	\bilGE {29} 
		{
			Инновации в туризме. Критерии отбора нововведений в сфере туризма.
		}{
			Виды налоговых выплат юридическими и физическими лицами при применении упрощенной системы налогообложения, учета и отчетности.
		}{
			Требования к оформлению документов. Оперативное хранение. Передача дел в архив.
		}
	\lowGE
\end{minipage}

\vfill



\begin{minipage}[t][\miniH]{\miniL}\centering
	\topGE * {080507.65} {Менеджмент организации}[Гостиничный и туристический бизнес]
	\bilGE {30} 
		{
			Инновационная деятельность и риск. Инновационные проекты фирмы: сущность, основные этапы формирования и реализации.
		}{
			Понятие производительности труда. Основные показатели оценки уровня и динамики производительности труда.
		}{
			Документирование движения персонала (прием, увольнение, перевод, отпуск, командировки).
		}
	\lowGE
\end{minipage}





\begin{minipage}[t][\miniH]{\miniL}\centering
	\topGE * {080507.65} {Менеджмент организации}[Гостиничный и туристический бизнес]
	\bilGE {31} 
		{
			Процесс реализации инновационного потенциала менеджмента, способы преодоления сопротивления персонала фирмы инновациям.
		}{
			Специфика труда менеджера. Основные требования к руководителю и персоналу гостиничных и туристских предприятий
		}{
			Разработка новых туристических продуктов: этапы и приемы.
		}
	\lowGE
\end{minipage}

\vfill



\begin{minipage}[t][\miniH]{\miniL}\centering
	\topGE * {080507.65} {Менеджмент организации}[Гостиничный и туристический бизнес]
	\bilGE {32} 
		{
			Конфликты в организациях: сущность, природа. Современная типология конфликтов в организациях и причины их возникновения. Методы разрешения конфликтных ситуаций в коллективе.
		}{
			Специфика мотивации сотрудников предприятия сферы туризма.
		}{
			Продвижение инновационных турпродуктов на региональном, национальном, международном рынках
		}
	\lowGE
\end{minipage}

\vfill



\begin{minipage}[t][\miniH]{\miniL}\centering
	\topGE * {080507.65} {Менеджмент организации}[Гостиничный и туристический бизнес]
	\bilGE {33} 
		{
			Миссия и цели организации. Принципы формулирования миссии предприятия сферы туризма.
		}{
			Методы отбора, приема персонала. Состав и структура персонала.
		}{
			Характеристика основных видов туризма в Псковской области.
		}
	\lowGE
\end{minipage}





\begin{minipage}[t][\miniH]{\miniL}\centering
	\topGE * {080507.65} {Менеджмент организации}[Гостиничный и туристический бизнес]
	\bilGE {34} 
		{
			Сущность и содержание функции контроля хода выполнения принятых решений.
		}{
			Характеристика ресурсов предприятия сферы туризма, пути их мобилизации.
		}{
			Туристские кластеры: проблемы и перспективы развития на региональном рынке
		}
	\lowGE
\end{minipage}

\vfill

	

\end{document}