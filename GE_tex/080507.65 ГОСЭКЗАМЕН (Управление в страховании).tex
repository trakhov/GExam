\documentclass[
	11pt,
	a4paper,
	]
	{article}

\usepackage{mysty, xparse, GExam, fancyhdr}

\usepackage[
	top = .7cm,
	right = 1cm,
	bottom = 1.2cm,
	left = 1cm,
	headheight = 14pt
	]{geometry}


\pagestyle{fancy}
\renewcommand{\headrulewidth}{0pt}
\renewcommand{\footrulewidth}{0pt}


\lhead{\begin{tikzpicture}[remember picture, overlay]
	\draw[dashed] (current page.north west) ++ (0, -99mm) -- +(210mm, 0);
	\draw[dashed] (current page.north west) ++ (0, -198mm) -- +(210mm, 0);
\end{tikzpicture}}
\cfoot{}



\begin{document}

	

\begin{minipage}[t][\miniH]{\miniL}\centering
	\topGE * {080507.65} {Менеджмент организации}[Управление в страховании]
	\bilGE {1} 
		{
			Задачи внедрения принципов и методов современного менеджмента в современную хозяйственную практику в РФ.
		}{
			Виды и формы оплаты труда. Принципы стимулирования оплаты труда. Модель стимулирующей оплаты труда.
		}{
			Виды налоговых выплат юридическими и физическими лицами при применении упрощенной системы налогообложения, учета и отчетности.
		}
	\lowGE
\end{minipage}

\vfill



\begin{minipage}[t][\miniH]{\miniL}\centering
	\topGE * {080507.65} {Менеджмент организации}[Управление в страховании]
	\bilGE {2} 
		{
			Значение совершенствования управления для успеха в коммерческой деятельности современных организаций.
		}{
			Конфликты в организациях: сущность, природа. Современная типология конфликтов в организациях и причины их возникновения. Методы разрешения конфликтных ситуаций в коллективе. Характеристика конфликтов в страховом бизнесе.
		}{
			НДС: экономическое содержание и порядок расчета.
		}
	\lowGE
\end{minipage}

\vfill



\begin{minipage}[t][\miniH]{\miniL}\centering
	\topGE * {080507.65} {Менеджмент организации}[Управление в страховании]
	\bilGE {3} 
		{
			Роль менеджмента в повышении эффективности хозяйствования в России. Проблемы развития менеджмента в современной России.
		}{
			Пути повышения эффективности использования персонала. Аттестация работников сферы страхового бизнеса.
		}{
			Налог на прибыль: определение, значение для предприятия и процедура взимания. Порядок расчета налога на
		}
	\lowGE
\end{minipage}





\begin{minipage}[t][\miniH]{\miniL}\centering
	\topGE * {080507.65} {Менеджмент организации}[Управление в страховании]
	\bilGE {4} 
		{
			Организация как основа современного менеджмента. Жизненный цикл организации
		}{
			Миссия и цели организации. Принципы формулирования миссии предприятия сферы страхового бизнеса. Формирование целей.
		}{
			Состав платежей и налогов, используемых в регулировании ВЭД, их целевое назначение.
		}
	\lowGE
\end{minipage}

\vfill



\begin{minipage}[t][\miniH]{\miniL}\centering
	\topGE * {080507.65} {Менеджмент организации}[Управление в страховании]
	\bilGE {5} 
		{
			Менеджмент как наука: сущность, структура и содержание.
		}{
			Рынок, понятие и характеристики. Разновидности рынков. Общая характеристика рынка страхования России.
		}{
			Способы выявления влияния факторов на изменения результативного показателя в экономическом анализе деятельности предприятия в сфере страхового сервиса
		}
	\lowGE
\end{minipage}

\vfill



\begin{minipage}[t][\miniH]{\miniL}\centering
	\topGE * {080507.65} {Менеджмент организации}[Управление в страховании]
	\bilGE {6} 
		{
			Цели в системе современного менеджмента, основные требования, предъявляемые к ним Основные виды разделения управленческого труда: сущность, содержание
		}{
			Понятие блага, продукта, услуги. Ограниченность ресурсов, безграничность потребностей.
		}{
			Понятие производительности труда. Основные показатели оценки уровня и динамики производительности труда.
		}
	\lowGE
\end{minipage}





\begin{minipage}[t][\miniH]{\miniL}\centering
	\topGE * {080507.65} {Менеджмент организации}[Управление в страховании]
	\bilGE {7} 
		{
			Система функций современного менеджмента. Сущность и основные виды. Организация реализации принятых решений как одна из основных функций менеджмента.
		}{
			Экономическое развитие и экономический рост. Факторы экономического роста
		}{
			Составление и оформление организационно-распорядительных документов в сфере страхования: штатное расписание, положение о структурном подразделении, должностная инструкции.
		}
	\lowGE
\end{minipage}

\vfill



\begin{minipage}[t][\miniH]{\miniL}\centering
	\topGE * {080507.65} {Менеджмент организации}[Управление в страховании]
	\bilGE {8} 
		{
			Сущность и задачи процесса принятия решений. Критерии принятия решений в условиях неопределенности.
		}{
			Макроэкономическое равновесие и цикличность рыночной экономики. Запреты и ограничения неэкономического
		}{
			Документирование движения персонала в сфере страхования (прием, увольнение, перевод, отпуск, командировки)
		}
	\lowGE
\end{minipage}

\vfill



\begin{minipage}[t][\miniH]{\miniL}\centering
	\topGE * {080507.65} {Менеджмент организации}[Управление в страховании]
	\bilGE {9} 
		{
			Управленческий контроль как одна из функций управления. Формы контроля реализации решения.
		}{
			Спрос и предложения, факторы их определяющие. Закон спроса и предложения. Эластичность спроса и
		}{
			Основные требования к кадрам в страховом бизнесе. Организация продвижения по службе в страховом бизнесе
		}
	\lowGE
\end{minipage}





\begin{minipage}[t][\miniH]{\miniL}\centering
	\topGE * {080507.65} {Менеджмент организации}[Управление в страховании]
	\bilGE {10} 
		{
			Функция мотивации персонала в выполнении принятых решений. Сущность процессуальных теорий мотивации.
		}{
			Стили управления. Достоинства и недостатки различных стилей управления.
		}{
			Корпоративные нормы в страховом бизнесе. Методы подбора кадров в страховом бизнесе
		}
	\lowGE
\end{minipage}

\vfill



\begin{minipage}[t][\miniH]{\miniL}\centering
	\topGE * {080507.65} {Менеджмент организации}[Управление в страховании]
	\bilGE {11} 
		{
			Сущность и содержание функции контроля хода выполнения принятых решений.
		}{
			Организационно-правовые формы хозяйствования и их характеристика. Методы изменения форм собственности и
		}{
			Анализ структуры страхового портфеля страховой организации
		}
	\lowGE
\end{minipage}

\vfill



\begin{minipage}[t][\miniH]{\miniL}\centering
	\topGE * {080507.65} {Менеджмент организации}[Управление в страховании]
	\bilGE {12} 
		{
			Современные школы менеджмента: общая характеристика.
		}{
			Государственное регулирование бизнеса. Задачи государственного регулирования предпринимательской деятельности в сфере сферы страхового бизнеса.
		}{
			Особенности страхования имущества физических и юридических лиц.
		}
	\lowGE
\end{minipage}





\begin{minipage}[t][\miniH]{\miniL}\centering
	\topGE * {080507.65} {Менеджмент организации}[Управление в страховании]
	\bilGE {13} 
		{
			Особенности японской модели менеджмента: общая характеристика. Характер принятия управленческих решений и ответственности на японских фирмах.
		}{
			Доходы предприятий сферы страхового бизнеса: понятие, виды, значение и состав. Пути повышения доходов.
		}{
			Методы определения ущерба и страхового возмещения.
		}
	\lowGE
\end{minipage}

\vfill



\begin{minipage}[t][\miniH]{\miniL}\centering
	\topGE * {080507.65} {Менеджмент организации}[Управление в страховании]
	\bilGE {14} 
		{
			Характерные черты американской практики менеджмента: общая характеристика. Сравнительная характеристика американской и японской моделей менеджмента.
		}{
			Затраты предприятий сферы страхового бизнеса: их сущность, классификация и основные направления
		}{
			Страхование грузов на территории Российской Федерации.
		}
	\lowGE
\end{minipage}

\vfill



\begin{minipage}[t][\miniH]{\miniL}\centering
	\topGE * {080507.65} {Менеджмент организации}[Управление в страховании]
	\bilGE {15} 
		{
			Основные цели современного предприятия сферы страхового бизнеса. Роль в развитии экономики страны
		}{
			Формы и методы налогообложения предприятий сферы страхового бизнеса.
		}{
			Особенности управления рисками, связанными с внешнеэкономической деятельностью.
		}
	\lowGE
\end{minipage}





\begin{minipage}[t][\miniH]{\miniL}\centering
	\topGE * {080507.65} {Менеджмент организации}[Управление в страховании]
	\bilGE {16} 
		{
			Организация как открытая или закрытая система. Сравнительная характеристика систем.
		}{
			Прибыль как экономическая категория, ее сущность и значение. Значение прибыли для развития предприятия страхового бизнеса.
		}{
			Особенности страхования валютных рисков во внешнеэкономической деятельности.
		}
	\lowGE
\end{minipage}

\vfill



\begin{minipage}[t][\miniH]{\miniL}\centering
	\topGE * {080507.65} {Менеджмент организации}[Управление в страховании]
	\bilGE {17} 
		{
			Основные законы организации. Сущность проявления и действия законов организации на современных
		}{
			Рентабельность предприятий сферы страхового бизнеса: понятие, значение и методы расчета. Факторы, влияющие на рентабельность предприятий сферы страхового бизнеса.
		}{
			Особенности страхования валютных рисков во внешнеэкономической деятельности.
		}
	\lowGE
\end{minipage}

\vfill



\begin{minipage}[t][\miniH]{\miniL}\centering
	\topGE * {080507.65} {Менеджмент организации}[Управление в страховании]
	\bilGE {18} 
		{
			Организационная структура, основные классификации. Структуры, ориентированные на нововведения. Основные принципы построения современных организационных структур.
		}{
			Сущность услуг и их место в экономической системе. Факторы, определяющие новую роль услуг. Характеристика услуги страхования.
		}{
			Имущественные виды страхования во внешнеэкономических отношениях.
		}
	\lowGE
\end{minipage}





\begin{minipage}[t][\miniH]{\miniL}\centering
	\topGE * {080507.65} {Менеджмент организации}[Управление в страховании]
	\bilGE {19} 
		{
			Внутренняя среда организации и ее основные элементы. Характеристика элементов внутренней среды предприятия сферы страхового бизнеса.
		}{
			Конкуренция: ее сущность, виды и роль в механизме функционирования рынка. Особенности формирования и использования конкурентных преимуществ в сфере услуг страхования.
		}{
			Страхование как необходимый элемент туристической деятельности.
		}
	\lowGE
\end{minipage}

\vfill



\begin{minipage}[t][\miniH]{\miniL}\centering
	\topGE * {080507.65} {Менеджмент организации}[Управление в страховании]
	\bilGE {20} 
		{
			Корпоративная культура и ее влияние на работу предприятия. Использование корпоративной культуры на современном предприятии сферы страхового бизнеса
		}{
			Элементы инфраструктуры бизнеса. Функции и задачи инфраструктуры малых предприятий.
		}{
			Виды страхования в туризме. Субъекты оказания страховых услуг туристам.
		}
	\lowGE
\end{minipage}

\vfill



\begin{minipage}[t][\miniH]{\miniL}\centering
	\topGE * {080507.65} {Менеджмент организации}[Управление в страховании]
	\bilGE {21} 
		{
			Внешняя среда фирмы: сущность, основные элементы. Характеристика элементов внешней среды современного предприятия сферы страхового бизнеса
		}{
			Малое предпринимательство в России. Экономические, социальные и правовые условия предпринимательства. Современные формы организации малого бизнеса.
		}{
			Особенности страхования граждан, выезжающих за рубеж. Формы организации страхования граждан, выезжающих за рубеж.
		}
	\lowGE
\end{minipage}





\begin{minipage}[t][\miniH]{\miniL}\centering
	\topGE * {080507.65} {Менеджмент организации}[Управление в страховании]
	\bilGE {22} 
		{
			Цели и задачи стратегического менеджмента. Тактические и стратегические мероприятия по выходу предприятия
		}{
			Государственное регулирование сферы страхового бизнеса: цели, основные инструменты.
		}{
			Договор страхования граждан, выезжающих за рубеж. Имущественное страхование туристов.
		}
	\lowGE
\end{minipage}

\vfill



\begin{minipage}[t][\miniH]{\miniL}\centering
	\topGE * {080507.65} {Менеджмент организации}[Управление в страховании]
	\bilGE {23} 
		{
			Типы базовых конкурентных стратегий: ценовое лидерство, дифференциация, фокусирование. Характерные черты базовых стратегий. Основные достоинства и опасности базовых стратегий.
		}{
			Понятие, сущность инновации, и ее роль в достижении целей производства услуг. Виды инноваций.
		}{
			Источники финансирования добровольного медицинского страхования. Варианты оплаты страхового случая. Страхование неотложной помощи.
		}
	\lowGE
\end{minipage}

\vfill



\begin{minipage}[t][\miniH]{\miniL}\centering
	\topGE * {080507.65} {Менеджмент организации}[Управление в страховании]
	\bilGE {24} 
		{
			Особенности стратегического и тактического управления в страховом бизнесе
		}{
			Сущность инновационного предпринимательства. Венчурный капитал в инновационном предпринимательстве.
		}{
			Особенности страхования международной торговли.
		}
	\lowGE
\end{minipage}





\begin{minipage}[t][\miniH]{\miniL}\centering
	\topGE * {080507.65} {Менеджмент организации}[Управление в страховании]
	\bilGE {25} 
		{
			Стратегическое планирование деятельности фирмы: сущность, решаемые проблемы, основные требования. Основные компоненты стратегического плана фирмы, их содержание.
		}{
			Оценка эффективных инновационных проектов. Инновационная политика.
		}{
			Страховая премия, форма и порядок ее уплаты. Франшиза.
		}
	\lowGE
\end{minipage}

\vfill



\begin{minipage}[t][\miniH]{\miniL}\centering
	\topGE * {080507.65} {Менеджмент организации}[Управление в страховании]
	\bilGE {26} 
		{
			Бизнес-план предприятия сферы страхового бизнеса: понятие, основные разделы.
		}{
			Основные компоненты инновационного процесса. Факторы, способствующие созданию инноваций в сфере услуг.
		}{
			Организация имущественного страхования.
		}
	\lowGE
\end{minipage}

\vfill



\begin{minipage}[t][\miniH]{\miniL}\centering
	\topGE * {080507.65} {Менеджмент организации}[Управление в страховании]
	\bilGE {27} 
		{
			Структура бизнес-плана. Показатели эффективности бизнес-плана: чистый приведенный доход, индекс прибыльности, внутренняя норма доходности, период окупаемости.
		}{
			Понятие и сущность риск-менеджмента. Риски и их виды. Показатели и методы количественной оценки рисков.
		}{
			Страхование профессиональной ответственности.
		}
	\lowGE
\end{minipage}





\begin{minipage}[t][\miniH]{\miniL}\centering
	\topGE * {080507.65} {Менеджмент организации}[Управление в страховании]
	\bilGE {28} 
		{
			Характеристика рисков в бизнес-планировании деятельности фирмы в сфере сервиса.
		}{
			Маркетинговые стратегии для организации сферы услуг. Жизненный цикл услуги страхования. Особенности ценообразования на различных стадиях.
		}{
			Страхование гражданской ответственности владельцев автотранспортных средств
		}
	\lowGE
\end{minipage}

\vfill



\begin{minipage}[t][\miniH]{\miniL}\centering
	\topGE * {080507.65} {Менеджмент организации}[Управление в страховании]
	\bilGE {29} 
		{
			Специфические черты страховой организации как объекта управления.
		}{
			Методы ценообразования. Структура цены на услуги страхования.
		}{
			Страхование гражданской ответственности авиаперевозчиков перед третьими лицами.
		}
	\lowGE
\end{minipage}

\vfill



\begin{minipage}[t][\miniH]{\miniL}\centering
	\topGE * {080507.65} {Менеджмент организации}[Управление в страховании]
	\bilGE {30} 
		{
			Специфика труда менеджера. Основные требования к руководителю предприятия сферы страхового бизнеса.
		}{
			Виды и источники инвестиций в сфере страхования.
		}{
			Права и обязанности граждан в системе обязательного медицинского страхования
		}
	\lowGE
\end{minipage}





\begin{minipage}[t][\miniH]{\miniL}\centering
	\topGE * {080507.65} {Менеджмент организации}[Управление в страховании]
	\bilGE {31} 
		{
			Специфика мотивации сотрудников предприятий сферы страхового бизнеса.
		}{
			Основные и оборотные фонды страховых организаций: состав, структура.
		}{
			Страхование финансовых рисков внешнеэкономической деятельности
		}
	\lowGE
\end{minipage}

\vfill



\begin{minipage}[t][\miniH]{\miniL}\centering
	\topGE * {080507.65} {Менеджмент организации}[Управление в страховании]
	\bilGE {32} 
		{
			Основные требования к созданию оптимальных условий труда на предприятиях сферы страхового бизнеса.
		}{
			Отчетность в страховом бизнесе. Порядок ее составления и предоставления.
		}{
			Особенности страхования кредитов. Страхование международных кредитных отношений. Формы страхования кредитных рисков
		}
	\lowGE
\end{minipage}

\vfill



\begin{minipage}[t][\miniH]{\miniL}\centering
	\topGE * {080507.65} {Менеджмент организации}[Управление в страховании]
	\bilGE {33} 
		{
			Полномочия, их распределение и делегирование на предприятии страхового бизнеса: этапы, основные требования.
		}{
			Ответственность налогоплательщика за нарушение налогового законодательства.
		}{
			Страхование рисков внешнеэкономической деятельности отечественными страховыми компаниями.
		}
	\lowGE
\end{minipage}





\begin{minipage}[t][\miniH]{\miniL}\centering
	\topGE * {080507.65} {Менеджмент организации}[Управление в страховании]
	\bilGE {34} 
		{
			Организационная структура, основные классификации. Структуры, ориентированные на нововведения. Основные принципы построения современных организационных структур.
		}{
			Конфликты в организациях: сущность, природа. Современная типология конфликтов в организациях и причины их возникновения. Методы разрешения конфликтных ситуаций в коллективе. Характеристика конфликтов в страховом бизнесе.
		}{
			Коммуникации и переговоры в страховом бизнесе. Функции делового общения.
		}
	\lowGE
\end{minipage}

\vfill

	

\end{document}