\documentclass[
	11pt,
	a4paper,
	]
	{article}

\usepackage{mysty, xparse, GExam, fancyhdr}

\usepackage[
	top = .7cm,
	right = 1cm,
	bottom = 1.2cm,
	left = 1cm,
	headheight = 14pt
	]{geometry}


\pagestyle{fancy}
\renewcommand{\headrulewidth}{0pt}
\renewcommand{\footrulewidth}{0pt}


\lhead{\begin{tikzpicture}[remember picture, overlay]
	\draw[dashed] (current page.north west) ++ (0, -99mm) -- +(210mm, 0);
	\draw[dashed] (current page.north west) ++ (0, -198mm) -- +(210mm, 0);
\end{tikzpicture}}
\cfoot{}



\begin{document}

\smallGE  {100100.62} {Сервис}

	

\noindent\bilGE {1} 
	{
		Классификация и основные характеристики автосервисных предприятий.
	}{
		Основные агрегаты и системы легкового автомобиля.
	}{
		Цели и процедура проведения инструментального контроля.
	}

\bigskip

\noindent\bilGE {2} 
	{
		Порядок разработки и основные элементы бизнес-плана сервисного центра.
	}{
		Основные технические характеристики автомобиля.
	}{
		Планово-предупредительная система технического обслуживания и ремонта.
	}

\bigskip

\noindent\bilGE {3} 
	{
		Организационная структура управления сервисным центром.
	}{
		Классификация и принцип действия двигателей легковых автомобилей.
	}{
		Способы ремонта и восстановления автомобильных деталей.
	}

\bigskip

\noindent\bilGE {4} 
	{
		Системы автоматизированного управления предприятиями автосервиса. Система «Автодилер».
	}{
		Системы питания двигателей легковых автомобилей и используемые виды топлива.
	}{
		Оценка технического состояния автомобиля. Виды и методы.
	}

\bigskip

\noindent\bilGE {5} 
	{
		Определение свободной доли рынка. Расчет годового объема работ автосервисных предприятий.
	}{
		Системы смазки двигателя внутреннего сгорания и используемые моторные масла.
	}{
		Техническое обслуживание и ремонт двигателя автомобиля.
	}

\bigskip

\noindent\bilGE {6} 
	{
		Анализ рынка автосервисных услуг. Тенденции развития.
	}{
		Система охлаждения двигателя, применяемые эксплуатационные материалы.
	}{
		Техническое обслуживание и ремонт узлов трансмиссии.
	}

\bigskip

\noindent\bilGE {7} 
	{
		Значение и формирование Brend-имиджа сервисных центров.
	}{
		Система зажигания двигателя. Принцип действия.
	}{
		Техническое обслуживание и ремонт подвески автомобиля.
	}

\bigskip

\noindent\bilGE {8} 
	{
		Прогнозирование спроса потребителей на автосервисные услуги.
	}{
		Система пуска двигателя. Принцип действия.
	}{
		Техническое обслуживание и ремонт кузова автомобиля.
	}

\bigskip

\noindent\bilGE {9} 
	{
		Стратегии привлечения клиентов. Разработка медиа-плана.
	}{
		Гидромуфта, гидротрансформатор, вариатор. Назначение, принцип действия.
	}{
		Техническое обслуживание и ремонт рулевого управления.
	}

\bigskip

\noindent\bilGE {10} 
	{
		Ценовая политика автосервисных предприятий.
	}{
		Дифференциал. Назначение, принцип действия.
	}{
		Техническое обслуживание и ремонт тормозной системы.
	}

\bigskip

\noindent\bilGE {11} 
	{
		Кадровая политика автосервисных предприятий.
	}{
		Конструкции и характеристики подвесок автомобиля.
	}{
		Техническое обслуживание и ремонт генератора.
	}

\bigskip

\noindent\bilGE {12} 
	{
		Понятия «сервис», «услуга», «сервисная деятельность».
	}{
		Конструкции и характеристики колес автомобиля.
	}{
		Техническое обслуживание и ремонт стартера.
	}

\bigskip

\noindent\bilGE {13} 
	{
		Понятие и характеристики контактной зоны взаимодействия с клиентом.
	}{
		Назначение и типы кузов легковых автомобилей.
	}{
		Техническое обслуживание и ремонт АКБ.
	}

\bigskip

\noindent\bilGE {14} 
	{
		Ответственность сервисного центра перед заказчиком. Нормативные документы.
	}{
		Узлы рулевого управления легкового автомобиля.
	}{
		Техническое обслуживание и ремонт системы питания.
	}

\bigskip

\noindent\bilGE {15} 
	{
		Обязанности сервис-менеджера предприятия автосервиса.
	}{
		Назначение, типы и конструкции тормозных систем автомобиля, используемые эксплуатационные материалы.
	}{
		Техническое обслуживание и ремонт смазочной системы.
	}

\bigskip

\noindent\bilGE {16} 
	{
		Трудоемкость при техническом обслуживании и ремонте автотранспортных средств.
	}{
		Основные системы и узлы электрооборудования автомобиля.
	}{
		Техническое обслуживание и ремонт системы охлаждения.
	}

\bigskip

\noindent\bilGE {17} 
	{
		Сегментация рынка при проектировании сервисных центров.
	}{
		Потребительские и эксплуатационные свойства автомобилей.
	}{
		Техническое обслуживание и ремонт системы впуска воздуха.
	}

\bigskip

\noindent\bilGE {18} 
	{
		Поведение и действия сервис-менеджера в зависимости от типологии клиентов.
	}{
		Модельные ряды автомобилей основных производителей.
	}{
		Техническое обслуживание и ремонт системы выпуска выхлопных газов.
	}

\bigskip

\noindent\bilGE {19} 
	{
		Сервисное обеспечение автомобильных дорог.
	}{
		Автоэлектроника. Состояние и перспективы развития.
	}{
		Диагностика автотранспортных средств. Назначение и виды.
	}

\bigskip

\noindent\bilGE {20} 
	{
		Современные информационные технологии, применяемые в сервисной деятельности.
	}{
		Тюнинг. Цели и задачи, этапы развития тюнинга. Виды тюнинга.
	}{
		Назначение и структура сервисной карты обслуживания потребителя.
	}

\bigskip

\noindent\bilGE {21} 
	{
		Типологические характеристики различных групп потребителей услуг.
	}{
		Функциональный тюнинг и дооборудование автомобилей и мотоциклов.
	}{
		Маршрутная технологическая карта ремонта систем и агрегатов автомобиля.
	}

\bigskip

\noindent\bilGE {22} 
	{
		Общероссийский классификатор услуг населению. (ОКУН)
	}{
		Конструктивный тюнинг автомобилей и мотоциклов.
	}{
		Операционная технологическая карта.
	}

\bigskip

\noindent\bilGE {23} 
	{
		Стратегия завоевания рынка, ассортиментная политика сервисных центров.
	}{
		Маршрутная технологическая карта услуги по тюнингу автомобилей и мотоциклов.
	}{
		Обоснование мощности СТО. Расчет годового объёма работ.
	}

\bigskip

\noindent\bilGE {24} 
	{
		Ценовые и неценовые факторы спроса и предложения на рынке услуг.
	}{
		Тюнинг двигателя.
	}{
		Методы технического обслуживания и ремонта. Расчет числа постов.
	}

\bigskip

\noindent\bilGE {25} 
	{
		Основные факторы, влияющие на качество предоставляемой услуги.
	}{
		Тюнинг трансмиссии.
	}{
		Определение потребности в технологическом оборудовании.
	}

\bigskip

\noindent\bilGE {26} 
	{
		Стадии и процесс коммуникации сервис-менеджера и клиента.
	}{
		Тюнинг тормозной системы автомобилей и мотоциклов.
	}{
		Определение площадей помещений СТО.
	}

\bigskip

\noindent\bilGE {27} 
	{
		Классификация и основные характеристики автосервисных предприятий.
	}{
		Тюнинг кузова и силового каркаса автомобилей и мотоциклов.
	}{
		Виды складов. Их оборудование.
	}

\bigskip

\noindent\bilGE {28} 
	{
		Порядок разработки и основные элементы бизнес-плана сервисного центра.
	}{
		Эстетический тюнинг автомобилей и мотоциклов.
	}{
		Организация снабжения, хранения и учета запасных частей и материалов.
	}

\bigskip

\noindent\bilGE {29} 
	{
		Организационная структура управления сервисным центром.
	}{
		Технология проектирования услуги по тюнингу автомобилей и мотоциклов.
	}{
		Экологические требования к размещению и проектированию предприятий автомобильного транспорта.
	}

\bigskip

\noindent\bilGE {30} 
	{
		Определение свободной доли рынка. Расчет годового объема работ автосервисных предприятий.
	}{
		Системы эффективного управления тормозами: антиблокировочная система (АБС); противобуксовочная система (ПБС); система динамической стабилизации (СДС).
	}{
		Экологические требования при эксплуатации предприятий автомобильного транспорта.
	}

\bigskip

\end{document}