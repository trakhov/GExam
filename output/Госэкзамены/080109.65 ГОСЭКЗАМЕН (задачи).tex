\documentclass[
	11pt,
	a4paper,
	]
	{article}

\usepackage{mysty, xparse, GExam, fancyhdr}

\usepackage[
	top = 1cm,
	right = 1cm,
	bottom = 1.2cm,
	left = 1cm,
	headheight = 14pt
	]{geometry}


\pagestyle{empty}
\renewcommand{\headrulewidth}{0pt}
\renewcommand{\footrulewidth}{0pt}



\renewcommand{\miniH}{150mm}
\renewcommand{\miniL}{185mm}



\begin{document}


\topGE * {080109.65} {Бухгалтерский учет, анализ и аудит}

\prGE

	Из кассы организации работнику был выдан аванс на командировочные расходы в сумме $25000$ руб. Срок командировки составляет $15$ суток. Норматив суточных, установленных в организации, составляет $350$ руб. в сутки. По возвращении из командировки работником в установленные сроки был предоставлен авансовый отчет:

	\begin{itemize}
		\item проезд до места командировки и обратно, в соответствии со стоимостью билетов 3200 руб.
		\item суточные 5250 руб.
		\item оплата проживания 16250 руб., в т.ч. НДС.
	\end{itemize}

	Остаток неизрасходованного аванса был возвращен в кассу. Отразить ситуацию на счетах бухгалтерского учета.

\begin{center}\lowGE * \end{center}

\newpage




\topGE * {080109.65} {Бухгалтерский учет, анализ и аудит}
\prGE

	Организация продает материалы (не основной вид деятельности). 

	\begin{itemize}
		\item продажная стоимость --- $77 700$ руб., в том числе НДС; 
		\item фактическая себестоимость материалов --- $ 56 000 $ руб., 
		\item затраты по доставке материалов покупателю, оплаченные поставщиком транспортной организации --- $ 3 600 $ руб., в том числе НДС. 
	\end{itemize}

	Определить финансовый результат от продажи материалов и отразить ситуацию на счетах бухгалтерского учета.

\begin{center}\lowGE * \end{center}
\newpage



% 6.66
\topGE * {080109.65} {Бухгалтерский учет, анализ и аудит}
\prGE

	При проведении инвентаризации материальных ценностей по состоянию на $ 31 $ декабря 2013 года в организации выявлена недостача материалов. Фактическая себестоимость материалов составляет $ 3 900 $ руб. Рыночная стоимость материалов составляет $ 5700 $ руб. Приказом руководителя сумма недостачи подлежит возмещению материально-ответственным лицом.
	\medskip

	Какими бухгалтерскими записями следует отразить данную операцию?

\begin{center}\lowGE * \end{center}
\newpage



% 6.66
\topGE * {080109.65} {Бухгалтерский учет, анализ и аудит}
\prGE

	Предприятие применяет способ уменьшаемого остатка при начислении амортизации. 
	Приобретен объект основных средств стоимостью $325$ тыс. руб. Срок полезного использования в соответствии с Классификатором составляет $5$ лет. 
	\medskip

	Рассчитайте сумму амортизации во второй год эксплуатации и отразите ее на счетах бухгалтерского учета, если объект используется в основном производстве. Коэффициент ускорения, применяемый организацией равен $2$.

\begin{center}\lowGE * \end{center}
\newpage




% 6.66
\topGE * {080109.65} {Бухгалтерский учет, анализ и аудит}
\prGE

	Приобретен объект основных средств стоимостью $120$ тыс. руб. со сроком полезного использования $10$ лет. \medskip

	Рассчитать сумму амортизации за отчетный месяц и отразить ее на счетах бухгалтерского учета, если организация применяет линейный метод начисления амортизации. Объект используется в управленческой деятельности.

\begin{center}\lowGE * \end{center}
\newpage



% 6.66
\topGE * {080109.65} {Бухгалтерский учет, анализ и аудит}
\prGE

	Организацией приобретен автомобиль грузоподъемностью свыше $2$ т. с предполагаемым пробегом до $500$ тыс. км стоимостью $100$ тыс. руб. В отчетном периоде пробег составляет $5$ тыс. км. 
	\medskip

	Рассчитайте сумму амортизационных отчислений при способе списания стоимости пропорционально объему продукции (работ).

\begin{center}\lowGE * \end{center}
\newpage



% 6.66
\topGE * {080109.65} {Бухгалтерский учет, анализ и аудит}
\prGE

	Организация передает безвозмездно основные средства. Первоначальная стоимость --- $85 500$ руб., сумма амортизации на дату передачи --- $54 000$ руб. Заработная плата, начисленная рабочим за работы по демонтажу объекта основных средств, составила $30000$ руб. 
	\medskip

	Отразить ситуацию на счетах бухгалтерского учета и определить финансовый результат от передачи основного средства.

\begin{center}\lowGE * \end{center}
\newpage




% 6.66
\topGE * {080109.65} {Бухгалтерский учет, анализ и аудит}
\prGE

	Организация реализует объект основных средств. Цена продажи --- $100 000$ руб., в том числе НДС. Первоначальная стоимость объекта --- $85 500$ руб., срок полезного использования в соответствии с Классификатором --- $7$ лет. Объект находился в 	эксплуатации $3$ года $5$ месяцев. 
	\medskip

	Определить финансовый результат и отразить ситуацию на счетах бухгалтерского учета. Способ начисления амортизации --- линейный.
	
\begin{center}\lowGE * \end{center}
\newpage





% 6.66
\topGE * {080109.65} {Бухгалтерский учет, анализ и аудит}
\prGE

	Составить расчет оплаты труда и пособия по временной нетрудоспособности главному бухгалтеру организации, занимающейся торговлей. 
	\medskip

	Имеет одного ребенка в возрасте до 18 лет. Стаж работы --- 6 лет. Оклад главного бухгалтера --- 16 000 руб. Премия --- 50\%. Листок временной нетрудоспособности представлен с 9 по 14 января. Оплата труда за 2012 год --- 242 400 руб., за 2013 год --- 285300 руб. 
	\medskip

	Произвести удержания из оплаты труда и определить сумму к выдаче. Отразить операции на счетах бухгалтерского учета.

\begin{center}\lowGE * \end{center}
\newpage



% 6.66
\topGE * {080109.65} {Бухгалтерский учет, анализ и аудит}
\prGE	 

	Основной деятельностью ООО «Пламя» является оказание консалтинговых услуг. В отчетном периоде организация оказала услуги ЗАО «Восток» стоимостью 365 000 руб., в том числе НДС. Фактическая себестоимость услуг составила 200 000 руб. Расчеты между сторонами произведены после подписания Акта об оказании услуг. 
	\medskip

	Отразить ситуацию на счетах бухгалтерского учета.

\begin{center}\lowGE * \end{center}
\newpage





% 6.66
\topGE * {080109.65} {Бухгалтерский учет, анализ и аудит}
\prGE

	По итогам отчетного года ОАО выявлена сумма нераспределенной прибыли в размере 450 000 руб. По решению собрания акционеров прибыль распределена следующим образом:

	\begin{itemize}
		\item 25\% направлено на увеличение Уставного капитала;
		\item 5\% направлено на формирование Резервного фонда;
		\item оставшаяся часть прибыли направлена на выплату дивидендов акционерам (25\% --- работникам организации, 75\% --- сторонним акционерам). 
	\end{itemize}

	Дивиденды выплачены акционерам из кассы полностью. Отразить ситуацию на счетах бухгалтерского учета.

\begin{center}\lowGE * \end{center}
\newpage





% 6.66
\topGE * {080109.65} {Бухгалтерский учет, анализ и аудит}
\prGE

	Составить расчет заработной платы и оплаты за дни отпуска главному бухгалтеру организации (имеет одного ребенка в возрасте до 18 лет).
	\medskip

	Оклад главного бухгалтера --- 15 000 руб. Отпуск предоставлен с 5 июня (понедельник) на 28 календарных дней. Заработная плата за 12 месяцев, предшествующих отпуску составила 197 600 руб. Совокупный доход с начала года --- 108 750 руб. 
	\medskip

	Произвести удержания из суммы начисленной заработной платы и отпускных и определить сумму к выдаче. Отразить операции на счетах бухгалтерского учета (в организации создается резерв на оплату отпускных).

\begin{center}\lowGE * \end{center}
\newpage



% 6.66
\topGE * {080109.65} {Бухгалтерский учет, анализ и аудит}
\prGE

	Организация изготавливает тару для упаковки готовой продукции силами вспомогательного подразделения. Стоимость израсходованных материалов составила 5000 руб., заработная плата рабочих, занятых изготовлением --- 1000 руб., отчисления на страхование от профессиональных заболеваний и трудовых --- 1\%. 
	\medskip

	Отразить ситуацию на счетах бухгалтерского учета.

\begin{center}\lowGE * \end{center}
\newpage




% АУДИТ

% 6.66
\topGE * {080109.65} {Бухгалтерский учет, анализ и аудит}
\prGE

	При проведении аудиторской проверки ОАО завода «Металлист» за 2013 год, 14.03.2014 г. на складе № 1 у заведующей складом Ивановой И.И. была выявлена недостача материалов на сумму 1 161 590 р. Иванова И.И. от возмещения недостачи в полном размере отказалась, т.к. представила акт на порчу материалов в сумме 525 500 р., составленный 10.01.2013 г.
	\medskip

	Директор завода Куприн А.В. своим распоряжением освободил Иванову И.И. от возмещения недостачи на сумму 525 500р. На основании распоряжения директора завода с Ивановой И.И. взыскана сумма 636 090 р. Рыночная стоимость материалов --- 650000р. 
	\medskip

	В журнале операций сделаны записи:

	\begin{itemize}
		\item Дебет --- 50 Кредит --- 10 --- 636 090
		\item Дебет --- 25 Кредит --- 10 --- 525 500
	\end{itemize}

	Определить правильность составления бухгалтерских записей. Указать правильные варианты отражения данной хозяйственной операции.

\begin{center}\lowGE * \end{center}
\newpage


% 6.66
\topGE * {080109.65} {Бухгалтерский учет, анализ и аудит}
\prGE

	ООО «Магнат» образовалось 11 января 2013 г. в г. Пскове. Уставный капитал составляет 34 223 руб. ЗАО «Прогресс» имеет долю в размере 50\% от уставного капитала номинальной стоимостью 17 112 руб. ЗАО «Импульс» имеет долю в размере 40\% от уставного капитала номинальной стоимостью 13 690 руб. Г-н Петров Г.В. имеет долю в размере 10\% от уставного капитала номинальной стоимостью 3421 руб.
	\medskip

	ЗАО «Импульс» и г-н Петров Г.В. вносят свои доли денежными средствами. ЗАО «Прогресс» внесло в качестве вклада в уставный капитал оборудование. Участники решили, что оно стоит 17 112 руб. Независимый оценщик также оценил оборудование в 17 112 руб. 
	\medskip

	Основным видом деятельности предприятия является производство мебели. В бухгалтерском учете ООО «Магнат» по учету уставного капитала и расчетам с учредителями были сделаны следующие записи:

	\begin{itemize}
		\item 11.01.2013г.: Дебет 75 Кредит 80 — 34 223 руб. — отражена задолженность учредителей по вкладам в уставный капитал;
		\item 11.01.2013 г.: Дебет 51 Кредит 75 — 3421 руб. (приходный кассовый ордер № 1 от 11.01. 2013 г.) поступил вклад в УК от г-на Петрова Г. В.;
		\item 11.01.2013 г.: Дебет 01 Кредит 80 — 17 112 руб. (Акт приема-передачи №2 от 11.01.2013 г.) поступил вклад от ЗАО «Прогресс»;
	\end{itemize}

	\textbf{Требуется:}
	\begin{enumerate}
		\item Проверить своевременность расчетов с учредителями по взносам в уставный капитал. 
		\item Проверить правильность ведения учета.
		\item Указать правильные варианты отражения данной хозяйственной операции.
	\end{enumerate}

\begin{center}\lowGE * \end{center}
\newpage





% 6.66
\topGE * {080109.65} {Бухгалтерский учет, анализ и аудит}
\prGE
	Организацией в апреле проверяемого года на фондовой бирже были проданы акции ОАО КБ «Московский банк». В свидетельстве № 321 от 7 апреля указано: 

	\begin{itemize}
		\item продажная стоимость акций --- 70000 рублей;
		\item плата за услуги аукциона --- 16000 рублей;
		\item стоимость акций по данным бухгалтерского учета --- 18000 рублей.
	\end{itemize}

	В учете организации были сделаны следующие записи:
	\begin{itemize}
		\item Дебет счета 50 Кредит счета 58-1 --- 54000 рублей --- принято наличными за акции;
		\item Дебет счета 58-1 Кредит счета 99 --- 36000 рублей --- отражен доход от продажи.
	\end{itemize}

	\textbf{Требуется:}
	\begin{enumerate}
		\item Проверить правильность ведения учета. 
		\item Указать правильные варианты отражения данной хозяйственной операции.
	\end{enumerate}

\begin{center}\lowGE * \end{center}
\newpage







% 6.66
\topGE * {080109.65} {Бухгалтерский учет, анализ и аудит}
\prGE

	В ходе аудиторской проверки установлено, что организацией в проверяемом периоде приобретен грузовой автомобиль. Цена автомобиля 354000 рублей, в том числе НДС 54000 рублей, комиссионное вознаграждение 5\% от стоимости автомобиля, в т.ч. НДС. 
	\medskip

	Счет оплачен, автомобиль оприходован и поставлен на баланс, произведены расчеты с бюджетом по НДС. В бухгалтерском учете сделаны записи: 
	\begin{itemize}
		\item Дебет счета 60 Кредит счета 51 --- 354000 рублей --- оплата счета за автомобиль;
		\item Дебет счета 08 Кредит счета 60 --- 300000 рублей --- оприходован автомобиль;
		\item Дебет счета 19 Кредит счета 60 --- 54000 рублей --- отражена сумма НДС;
		\item Дебет счета 01 Кредит счета 08 --- 300000 рублей --- автомобиль введен в эксплуатацию;
		\item Дебет счета 68 Кредит счета 19 --- 54000 рублей --- предъявлен НДС к вычету;
		\item Дебет счета 20 Кредит счета 60 --- 17700 рублей --- отражена сумма комиссионного вознаграждения.
	\end{itemize}
	\textbf{Требуется:}
	\begin{enumerate}
		\item Проверить правильность ведения учета. 
		\item Сделать выводы аудитора при анализе данной ситуации.
		\item Указать правильные варианты отражения данной хозяйственной операции.
	\end{enumerate}

\begin{center}\lowGE * \end{center}
\newpage




% 6.66
\topGE * {080109.65} {Бухгалтерский учет, анализ и аудит}
\prGE

	Инженеру Сидорову А.А. выплачен аванс на командировочные расходы --- 25 000руб. В командировочном удостоверении указывается, что Сидоров командируется в Москву сроком на 20 дней с 05.01.14г. по 25.01.14г. Утвержден авансовый отчет инженера Сидорова А.А. по командировке в г.Москву в следующем размере:

	\begin{table}[ht!]\centering
	\begin{tabular}{rl}
		Суточные за 20 дней: 		& $250 \times 20 = 5000$ \\
		Квартирные за 20 дней:	& $1000 \times 20 = 20 000$ \\
		Проезд туда и обратно поездом: &  $1250 \times 2 = 2500$ \\\hline
		Итого: & $27 500$ 
	\end{tabular}
	\end{table}

	Отметки на обороте командировочного удостоверения:
	\begin{itemize}
		\item выбыл из Пскова 05.01.14 г. --- прибыл в Москву 06.01.14 г.
		\item выбыл из Москвы 25.01.14 г. --- прибыл в Псков 26.01.14 г.
	\end{itemize}

	При проверке проездных документов выявлено: компостер датирован на билетах Псков --- Москва 05.01.14~г., Москва --- Псков --- 25.01.14~г. Документы для оплаты квартирных отсутствуют.
	\medskip

	Нормы командировочных расходов:
	\begin{itemize}
		\item суточные в соответствии с локальным актом --- 250 руб.
		\item оплата найма жилого помещения:
		\begin{itemize}
			\item при наличии документов --- в соответствии с документами;
			\item без документов --- 12 руб.
		\end{itemize}
	\end{itemize}

	\textbf{Требуется:}
	\begin{enumerate}
		\item Определить ошибки, допущенные в оплате командировочных расходов инженеру Сидорову А.А.;
		\item Сделать выводы аудитора при анализе данной ситуации.
		\item Указать правильные варианты отражения данной хозяйственной операции.
	\end{enumerate}

\begin{center}\lowGE * \end{center}
\newpage




% 6.66
\topGE * {080109.65} {Бухгалтерский учет, анализ и аудит}
\prGE
	
	 При выборочной инвентаризации основных средств аудитором выявлена недостача объекта основных средств первоначальной стоимостью 160000 рублей и суммой начисленной амортизации 60000 рублей. В декабре проверяемого года этот объект основных средств был продан. Согласно договору продажная стоимость составила 240000 рублей, в т.ч. НДС. От работников аудируемого лица получены устные разъяснения. Договор и акт приема передачи представлены аудитору при проведении инвентаризации. Расчеты с покупателем не произведены. Эта хозяйственная операция не отражена на счетах бухгалтерского учета по состоянию на 31 декабря. 
	 \medskip

	\textbf{Требуется:}
	\begin{enumerate}
		\item Проверить правильность ведения учета и обоснованность совершенных операций. 

		\item Сделать выводы аудитора при анализе данной ситуации.

		\item Указать правильные варианты отражения данной хозяйственной операции.
	\end{enumerate}

\begin{center}\lowGE * \end{center}
\newpage



% АНАЛИЗ


% 6.66
\topGE * {080109.65} {Бухгалтерский учет, анализ и аудит}
\prGE
	Проанализировать выполнение плана по ассортименту продукции и при этом рассчитать: 

	\begin{enumerate}
		\item Фактический выпуск продукции в пределах плана, сверх плана и не предусмотренной планом.

		\item Удельный вес в общем фактическом выпуске продукции сверхплановой продукции и фактически выпущенной продукции в пределах плана.

		\item Рассчитать процент выполнения плана по каждому изделию.

		\item Оценить выполнение плана по ассортименту 3 способами.

		\item Сделать выводы по результатам анализа.
	\end{enumerate}

\begin{table}[ht!]\centering
	\begin{tabular}{|c|c|c |c|c|c| c|} 
		\hline
		&&& \multicolumn{3}{c|}{Выпуск продукции} &
		\% выполнения 
		\\ \cline{4-6}
		Изделие & По плану & Фактически & 	в пределах & 	сверх & 	не предусмотрено & плана
		\\
		& & & плана & плана & планом & 
		\\ \hline
		А & 3264 & 3245 & & & & \\ \hline
		Б & 760 & 655 & & & & \\ \hline
		В & 1913 & 2065 & & & & \\ \hline
		Итого & & & & & & \\ \hline
		Уд. вес, \%  & & & & & & 
		\\ \hline
	\end{tabular}
\end{table}

\begin{center}\lowGE * \end{center}
\newpage





% 6.66
\topGE * {080109.65} {Бухгалтерский учет, анализ и аудит}
\prGE

	Проанализировать выполнение плана по качеству продукции и при этом рассчитать: 

	\begin{enumerate}
		\item общий выпуск продукции в натуральном выражении.
		\item общий выпуск продукции и по каждому изделию в стоимостном выражении в тыс. руб. по плану и фактически.
		\item удельный вес каждого сорта в общем объеме продукции по плану и фактически.
		\item процент выполнения плана по каждому сорту.
		\item фактический и плановый коэффициент сортности.
		\item влияние изменения качества продукции на объем производства.
		\item сделать выводы по результатам анализа
	\end{enumerate}

\begin{table}[ht!]\centering
\begin{tabular}{|c |c |c|c |c|c |c|c |c|} 
\hline
Сорт & 
Стоимость & 
\multicolumn{2}{c|}{Выпуск} & 
\multicolumn{2}{c|}{Выпуск} &
\multicolumn{2}{c|}{Уд. вес, \%} & 
\% выполнения 
\\ 
изделия & 
1 изделия, & 
\multicolumn{2}{c|}{продукции} & 
\multicolumn{2}{c|}{продукции} &
\multicolumn{2}{c|}{} & плана 
\\ \cline{3-8}
& руб. & по плану & факт. & по плану & факт. & по плану & факт. & 
\\ \hline
1 & $2{,}0$ & 5010 & 5040 & & & & & 
\\ \hline
2 & $0{,}95$ & 805 & 790 & & & & &
\\ \hline
3 & $0{,}85$ & 122 & 135 & & & & & 
\\ \hline
\textbf{Итого} & X & & & & & & & 
\\ \hline
\end{tabular}
\end{table}

\begin{center}\lowGE * \end{center}
\newpage



% 6.66
\topGE * {080109.65} {Бухгалтерский учет, анализ и аудит}
\prGE

	Дать характеристику ритмичности работы предприятия и при этом рассчитать: 

\begin{enumerate}
	\item Удельный вес каждого месяца в общем объеме выпуска.
	\item Коэффициент ритмичности, коэффициент вариации, коэффициент аритмичности.
	\item Посчитать упущенные возможности по выпуску продукции3 способами.
	\item Сделать выводы по результатам анализа
\end{enumerate}

\begin{table}[ht!]\centering
\begin{tabular}{|c |c|c |c|c |c|}
\hline
\makebox[3cm]{Месяц} &
\multicolumn{2}{c|}{Выпуск продукции,} &
\multicolumn{2}{c|}{Удельный вес, \%} &
Фактический
\\
&
\multicolumn{2}{c|}{тыс. штук} &
\multicolumn{2}{c|}{} &
выпуск в
\\ \cline{2-5}
&
План &
Факт &
План &
Факт &
пределах плана
\\ \hline
Январь & $1959{,}21$ & $1543{,}62$ & & & 
\\ \hline
Февраль & $1959{,}21$ & $1840{,}47$ & & & 
\\ \hline
Март & $2018{,}58$ & $2580{,}91$ & & & 
\\ \hline
Квартал & & & & & 
\\ \hline
\end{tabular}
\end{table}

\begin{center}\lowGE * \end{center}
\newpage





% 6.66
\topGE * {080109.65} {Бухгалтерский учет, анализ и аудит}
\prGE

	Провести анализ эластичности спроса по цене и при этом рассчитать: 

\begin{enumerate}
	\item Процент изменения цен.
	\item Процент изменения спроса в результате изменения цен.
	\item Коэффициент эластичности спроса.
	\item Прогнозную выручку.
	\item Прогнозную прибыль.
	\item Сделать выводы по результатам анализа.
\end{enumerate}

\begin{table}[ht!]\centering
\begin{tabular}{|c |c |cc |c |c|c|c|}
\hline
Цена, &
Спрос, &
\multicolumn{2}{c|}{Прирост, \%} &
Коэффициент &
\multicolumn{3}{c|}{Результаты продаж, тыс. руб.}
\\ \cline{3-4} \cline{6-8}
тыс. руб. &
тыс. руб. &
Цены &
Спроса &
эластичности спроса &
Выручка &
Затраты &
Прибыль
\\ \hline
10 & 3000 & & & & & 23000 & \\ \hline
11 & 2900 & & & & & 24780 & \\ \hline
12 & 2800 & & & & & 26420 & \\ \hline
13 & 2700 & & & & & 27010 & \\ \hline
14 & 2600 & & & & & 28200 & \\ \hline
15 & 2550 & & & & & 30020 & \\ \hline
16 & 2500 & & & & & 31750 & \\ \hline
17 & 2350 & & & & & 31740 & \\ \hline
18 & 2300 & & & & & 33280 & \\ \hline
19 & 2200 & & & & & 33770 & \\ \hline
\end{tabular}
\end{table}

\begin{center}\lowGE * \end{center}
\newpage



% 6.66
\topGE * {080109.65} {Бухгалтерский учет, анализ и аудит}
\prGE

	Провести анализ технического состояния и эффективности использования основных фондов и при этом рассчитать: 

\begin{enumerate}
	\item Коэффициент износа
	\item Коэффициент годности
	\item Фондоотдачу
	\item Фондоемкость
	\item Фондоворуженность труда
	\item Рентабельность ОФ
	\item Коэффициент использования мощностей
	\item Сделать выводы по результатам анализа
\end{enumerate}

\begin{table}[ht!]\centering
\begin{tabular}{|l|c|c|}
\hline
\hfil Показатели & Отчетный год & Базисный год \\ \hline
Среднегодовая первоначальная стоимость основных фондов & 4627 & 4618 \\ \hline
% основных фондов & & \\ \hline
Среднегодовой износ ОФ & $956{,}8$ &  $902{,}1$ \\ \hline
Остаточная стоимость ОФ & & \\ \hline
Объем выпуска продукции & $2470{,}6$ & $2108{,}4$ \\ \hline
Численность рабочих, чел. & 1200 & 1100 \\ \hline
Прибыль & $368{,}02$ &  $230{,}01$ \\ \hline
Среднегодовая производственная мощность & 7000 & 7000 \\ \hline
\end{tabular}
\end{table}

\begin{center}\lowGE * \end{center}
\newpage




% 6.66
\topGE * {080109.65} {Бухгалтерский учет, анализ и аудит}
\prGE

	Рассчитать влияние на фондоотдачу различных факторов способом абсолютных разниц. Сделать выводы по результатам анализа.

\begin{table}[ht!]\centering
\begin{tabular}{|l|c|c|}
\hline
& Прошлый год & Отчетный год \\ \hline
Коэффициент материальных затрат & $3{,}05$ & $3{,}09$ \\ \hline
Производительность оборудования, руб. & 24410 & 27101 \\ \hline
Коэффициент сменности & $1{,}1$ & $1{,}2$ \\ \hline
Стоимость единицы оборудования, руб. & 3110 & 3058 \\ \hline
Удельный вес машин и оборудования & & \\
в общей сумме основных фондов, \% & 38 & $38{,}4$
\\ \hline
\end{tabular}
\end{table}

\begin{center}\lowGE * \end{center}
\newpage





% 6.66
\topGE * {080109.65} {Бухгалтерский учет, анализ и аудит}
\prGE

	Проанализировать использование фонда заработной платы и при этом определить:

\begin{enumerate}
	\item \% плана к прошлому году, \% выполнения плана, темпы роста.
	\item расход заработной платы на 1 рубль объема производства
	\item средняя заработная плата 1 рабочего ППП за год
	\item экономию средств на заработную плату по сравнению с прошлым годом
	\item абсолютное отклонение фактического фонда заработной платы от планового, общее и в т.ч. за счет изменения удельного расхода заработной платы и за счет изменения объема производства.
	\item сопоставить темпы изменения средней заработной платы и производительности труда. Рассчитать коэффициент опережения.
	\item сделать выводы по результатам анализа
\end{enumerate}

\begin{table}[ht!]\centering
\begin{tabular}{|*{7}{c|}}
\hline
Показатели 	& Прошлый & План & Факт & \% 					& \% плана к & темпы \\
						& год 		& 		 & 			& выполнения	& прошлому 	 & роста,\\
						&					&			 &			& плана 			& году			 & \%
\\ \hline
ВП, тыс. руб. & 22720 & 22960 & 23910 & & & \\ \hline
Фонд заработной платы ППП, тыс. руб. & 7290 & 7060 & 7320 & & & \\ \hline
Численность персонала, чел. & 47 & 45 & 46 & & & \\ \hline
Расход заработной платы  & & & & & & \\
	на 1 рубль объема производства, руб. & & & & & & \\ \hline
Средняя заработная плата  & & & & & & \\
	1 рабочего ППП за год, тыс.руб.  & & & & & & \\ \hline
Годовая выработка, тыс. руб. & & & & & & \\ \hline
\end{tabular}
\end{table}

\begin{center}\lowGE * \end{center}
\newpage



% 6.66
\topGE * {080109.65} {Бухгалтерский учет, анализ и аудит}
\prGE

	Проанализировать обеспеченность потребности в материальных ресурсах договорами и при этом рассчитать:

\begin{enumerate}
	\item внутренние источники покрытия потребности
	\item общую сумму потребности в материалах
	\item обеспеченность потребности договорами, \%
	\item выполнение договоров, \%
	\item плановый и фактический коэффициент обеспечения потребности источниками покрытия
	\item сделать выводы по результатам анализа
\end{enumerate}


\begin{table}[ht!]\footnotesize\centering
\begin{tabular}{|c |c |c|c |c |c |c |c |c|c |}
\hline
Материал &
Плановая &
\multicolumn{2}{c|}{Источники} &
Заключено &
Поступило &
Обеспеченность &
Выполнение &
\multicolumn{2}{c|}{Коэффициент} \\
& 
потребность, &
\multicolumn{2}{c|}{покрытия,} &
договоров, &
от &
потребности &
договоров, &
\multicolumn{2}{c|}{обеспечения} \\ \cline{9-10}
&
кг &
\multicolumn{2}{c|}{кг} &
кг &
поставщиков, &
договорами, \% &
\% &
план &
факт \\ \cline{3-4}
&
&
внутр. &
внеш. &
&
кг &
&
&
&
\\ \hline 
А & 29015 & & 28815 & 26500 & 25900 & & & & \\ \hline
Б & 34017 & & 33517 & 35400 & 33700 & & & & \\ \hline 
В & 12100 & & 11700 & 11700 & 10700 & & & & \\ \hline
Итого & & & & & & & & & \\ \hline
\end{tabular}
\end{table}

\begin{center}\lowGE * \end{center}
\newpage




% 6.66
\topGE * {080109.65} {Бухгалтерский учет, анализ и аудит}
\prGE

	Оценить влияние на себестоимость отклонений по заработной плате и при этом рассчитать:

\begin{enumerate}
	\item темп роста по всем показателям
	\item процент изменения себестоимости продукции вследствие разницы в темпах роста производительности труда и темпах роста средней заработной платы.
	\item абсолютный перерасход или экономию фонда заработной платы
	\item абсолютное изменение себестоимости за счет изменений в расходе фонда заработной платы.
	\item сделать выводы по результатам анализа
\end{enumerate}

\begin{table}[ht!]\centering
\begin{tabular}{|c |c |c |c |}
\hline
Наименование показателей & Прошлый год & Отчетный год & Темп роста, \% \\ \hline
Товарная продукция, тыс. руб. & 936564 & 966251 &\\ \hline
Фонд заработной платы ППП, тыс. руб. & 612518 & 614009 & \\ \hline
Численность ППП, человек & 9816 & 9899 & \\ \hline
Себестоимость ТП, тыс. руб. & 4287626 & 4288910 & \\ \hline
Среднегодовая выработка на 1 работающего, руб. & & & \\ \hline
Среднегодовая заработная плата 1 работающего, руб. & & & \\ \hline
\end{tabular}
\end{table}

\begin{center}\lowGE * \end{center}
\newpage




% 6.66
\topGE * {080109.65} {Бухгалтерский учет, анализ и аудит}
\prGE

	Проанализировать прибыль от реализации важнейших видов продукции и при этом рассчитать:

\begin{enumerate}
	\item процент выполнения плана
	\item себестоимость единицы продукции, руб.
	\item прибыль на единицу продукции, руб.
	\item прибыль от реализации, тыс. руб.
	\item влияние объема продаж, отпускной цены и себестоимости на прибыль от реализации 
	\item сделать выводы по результатам анализа
\end{enumerate}

\begin{table}[ht!]\centering
\begin{tabular}{|c|c|c|c|}
\hline
Показатели &
План &
Факт &
\% выполнения плана \\ \hline
Объем реализации, тыс. шт. & 700 & 800 & \\ \hline
Цена за единицу, руб. & 35 & 39 & \\ \hline
Переменные затраты на единицу, руб. & 20 & 21 & \\ \hline
Постоянные затраты на весь объем реализации, тыс.руб. & 2100 & 3200 & \\ \hline
Себестоимость единицы продукции, руб. & & & \\ \hline
Прибыль на единицу продукции, руб. & & & \\ \hline
Прибыль от реализации, тыс. руб. & & & \\ \hline
\end{tabular}
\end{table}

\begin{center}\lowGE * \end{center}
\newpage




% 6.66
\topGE * {080109.65} {Бухгалтерский учет, анализ и аудит}
\prGE

	Рассчитайте показатели ликвидности и финансовой устойчивости. Сделайте выводы по результатам анализа.

\begin{table}[ht!]\centering
\begin{tabular}{|c|c|}
\hline
Показатели & \makebox[3cm]{Значение} \\ \hline
Среднегодовая стоимость внеоборотных активов, тыс. руб. & 84835 \\ \hline
Среднегодовая стоимость оборотных активов, тыс. руб. & 25400 \\ \hline
Собственный капитал, тыс. руб. & 94592 \\ \hline
Долгосрочные обязательства, тыс. руб. & 1726 \\ \hline
Краткосрочные обязательства, тыс. руб. & 13917 \\ \hline
Кредиторская задолженность, тыс. руб. & 5365 \\ \hline
Среднегодовая стоимость запасов, тыс. руб. & 19794 \\ \hline
Дебиторская задолженность , тыс. руб. & 2957 \\ \hline
Денежные средства и краткосрочные финансовые вложения, тыс. руб. & 1975 \\ \hline
\end{tabular}
\end{table}

\begin{center}\lowGE * \end{center}
\newpage













\end{document}